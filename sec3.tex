\section{Geometrical Interpretation of the Dircichlet and Hermitian forms}

The possibility of \textcolor{red}{here} the Picard group and the polyhedral division belonging to it for the theory of the repeatedly mentioned forms, rests on a geometrical interpretation of these forms, which ties itself in directly to the interpretation of the Gaussian forms in the $\zeta$-halfplane(cf. p443). \textcolor{red}{here}\textcolor{red}{here} quotient \textcolor{red}{here} of the \textcolor{red}{here} indeterminate quantities x:y, denote this by $\zeta$ and bring in the $\zeta$-plane resp. the positive $\zeta$-halfplane(p.53).
If now, first of all, we have a Hermitian form (\textcolor{red}{here}), then by setting it equal to zerok with utilization of $\zeta$, one finds the equation:
\textcolor{red}{here}
or (if we set $\zeta$ = \textcolor{red}{here} as earlier):
(1)\textcolor{red}{here}
if, in particular, \textcolor{red}{here} is a definite Hermitian form, i.e, one such with \textcolor{red}{here}, then equation (1) represents an “imaginary circle” of the $\zeta$-plane. Since the form \textcolor{red}{here} process from the left side of (1) by multiplication by the non-negative number \textcolor{red}{here}, it follows, that a definite form \textcolor{red}{here}, for any  numbers \textcolor{red}{here} of \textcolor{red}{here}, is able to represent either only positive or negative real numbers. On this account, one seperates the forms in question into “positive” and “negative” forms and, moreover, notices immediately, that for a positive(negative) form the cofficients a and c are \textcolor{red}{here}. \textcolor{red}{here}, positive forns are able to be equivalent again only with positive ones, and negative ones with negative ones, and one easily writes from (y) on p.453.

We are obviously able to restrict \textcolor{red}{here} to the conversation of the positive forms, for which, now, a real geometrical interpretation is to be obtained in the $\zeta$-halfplane. For this purpose, we form, in the $\zeta$-plane, the pencil of all “spheres through the imaginary circle(1)”. If \textcolor{red}{here} is a real parameter, then this pencial represents itself by :
(2)
where \textcolor{red}{here} are the coordinates of the $\zeta$-plane already used above(p.54). The pencil(2) now contains two real boundary points, i.e., spheres infinitely small radius; here are the points with the coordinates:
(3)
that one of these two points, which belongs to the positive halfplane, we let \textcolor{red}{here} utilized for the geometric interpretation of the indefinite form \textcolor{red}{here}.
In order, first of all, to dispose of the Dirichlet forms at the same time, then to the form \textcolor{red}{here} corresponds the equation:
(4)
where roots are the following:
(5)

The square root of the discriminant is here to be so defined, that \textcolor{red}{here} is either to have a positive real component, or that, in case that vanishes, the imaginary component carries the positive sign. On the strength of this, call $\zeta$, the fuirstm \textcolor{red}{here} The second rot of equation(4). We then interpret the Dirichlet form by that semicircle belonging to the $\zeta$-halfpane, which stands particular to the $\zeta$-plane at the points \textcolor{red}{here} and \textcolor{red}{here} and is provided with an arrow pointing from \textcolor{red}{here} to \textcolor{red}{here}.

In the case of an indefinite Hermitian form, equation (1) represents a real circle of the $\zeta$-plane, corresponding to the name “indefinite”, the form has positive numerical values on the one side of this circle, and negative ones on the others; and we distinguish, in this case, on the periphery of the circle, a “positive” and “negative” \textcolor{red}{here}(bank?). Run through the periphery in such a direction, that the positive \textcolor{red}{here} is to beleft, and provide the circle with an arrow indicating the sense of this circuit. Finally, we erect in the $\zeta$-halfpane as hemisphere, which rises up or \textcolor{red}{here} top the $\zeta$-plane at the circle in question. This hemisphere, whose boundary(1) belonging to the $\zeta$-plane is provided with an arrow in the manner \textcolor{red}{here}, let be the geometrical image of the indefinite form \textcolor{red}{here}

In all three cases, now, the theorem holds, that the individual form is uniquely defined by the determinant D and the representing geometric image. The representing image, namely, furnishes, first of all, the quotients of the coefficients a,b,c. The addition of D permits the determination of the coefficients up to a common sign change, i.e, \textcolor{red}{here}, in each case, two forms “inverse: to one another \textcolor{red}{here},\textcolor{red}{here} resp.

 \textcolor{red}{here}\textcolor{red}{here}. Finally, for a definite Hermitian form, the sign is to be so \textcolor{red}{here}, that the form is positive; in the two other cases, however, the direction of the \textcolor{red}{here} of the representing image be \textcolor{red}{here} concerning the sign.

Further, the following important theorem holds: If, from any proposed form, one goes over by exercise of a \textcolor{red}{here} substitution to an equivalent form, then by the correspoinding \textcolor{red}{here} $\zeta$-substitutions, the representing model(point, semicircle, \textcolor{red}{here}) of the first form is transformed into that of the second. This theorem is an immediate result of the method followed in the geometrical interpretation of the forms, which to me individual form assigned its geometrical \textcolor{red}{here} in a manner invariant with respect to $\zeta$-substitutions, One now only still wants to convince oneself, that for a semicircle resp. hemisphere, the transformation by a unimodular $\zeta$-substitution furnishes the appropriate direction of the arrow of the \textcolor{red}{here} model. This yields itself from the circumstance, that in the fixing of the direction of the arrow, we proceeded according to the method, which is incariant with respect to the unimodular $\zeta$-subsitutions of the first kind(notations of the $\zeta$-halfsoace resp. the $\zeta$-plane). One also equally verifies, however, e.g., for the Dirichlet forms, by direct calculation and utilization of (3) on p.451, that the “first” root $\zeta$, of (a,b,c), by exercise of a unimodular $\zeta$-substitution of \textcolor{red}{here}, goes over into \textcolor{red}{here}, i.e, into the first root of the transformed form.


