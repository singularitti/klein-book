\section{Theory of the Hermitian and Dirichlet forms in projective-geometric form}

As, in the last paragraph, we could \textcolor{red}{here} the theory of the Gaussian forms on the basis of the “rectilinear modular figure”, so it is now correspondingly possible to base the theory of the Hermitian and Dirichlet forms, on the projective form of the Picard group. According to the representations of the previous paragraphs, it thereby appears indicated, to \textcolor{red}{here} equivalence simultaneously in the narrower and in the extended sense (cf. p.453). Then the division of the interior of the sphere of the hyperbolic space into plane-surfaced tetrahedral resp. double tetrahedral becomes the proper foundation of the theory.

Hereby, naturally, it is again a question only of a more external metamorphosis of our theory laid out above. But this metamorphosis is remarkable, since the geometric interpretation of the forms in the hyperbolic space fashions itself even more simply and naturally than in the $\zeta$-space.

In fact, we had first interpreted a definite Hermitian form $(a,b_1,b_2,c)$ by that point of trhe $\zeta$-halfspace, which had the coordinates:\textcolor{red}{text}
$$\zeta=\frac{-b_1}{a}, n=\frac{b_2}{a}, \mathcal{G}=\pm \frac{\sqrt{-D}}{a}.$$
Now that mapping between the $\zeta$-halfspace and the interior of the sphere of the hyperbolic space is given by:
$$\zeta=\frac{x_2+\ii x_3}{x_4},=\frac{\sqrt{x_1x_4-x_2^2-x_3^2}}{x_4}.$$
We accordingly arrive at the following conception: We shall interpret a definite Hermitian form $(a,b_1,b_2,c)$ geometrically by that point situated in the interior of the sphere of the hyperbolic space, whose coordinates are the following:
\begin{align}
x_1=c,x_2=-b_1,x_3=b_2,x_4=a.
\end{align}

It is now a great simplification (compared to the $\zeta$-space), that the interpretation hereby given also carries itself over immediately to the indefinite forms too. In fact, we shall also interpret an indefinite Hermitian form $(a,b_1,b_2,c)$ by the point of the hyperbolic space, whose coordinates are given by (1); then, however, this representing point lies outside the absolute space.

By grouping both cases together, one can consequently state the theorem, that the Hermitian forms corresponding to all the “rational points” of the hyperb0olic space, filling it out everywhere densely. Thereby, the points outside the absolute sphere furnish the indefinite forms, those inside the sphere the definite forms, while the rational situated on the spherical surface itself given the Hermitian forms of vanishing determinant (not \textcolor{red}{here} above).
In carrying through the equivalence theory on the basis of the preceding interpretation of the forms, the circumstance is again decisive, that our group is indeed properly discontinuous inside, but not outside the absolute surface of the projective space. Analogously, as for the Gaussian forms, we shall rep;ace the representing point of an indefinite form by its polar plane with respect to the absolute sphere, pushing through the interior of the sphere: 
\begin{align}
ax_1+\alpha b_1x_2-2b_2x_3+cx_4=0.
\end{align}
This polar plane then corresponds exactly to the representing hemisphere of \textcolor{red}{here}, as we utilized it above in the $\zeta$-plane.——

Also, the interpretation of the Dirichlet forms attaches itself here effortlessly: The individual Dirichlet form $(a,b,c)$ is interpreted by that rectilinear secant of the absolute sphere, which joins the two “zeroes” of the form situated on the surface of the sphere. In contrast to the Hermitian forms, however, it is not always a question here of “rational” lines of the hyperbolic space. Namely, if we set $p_{ik}=x_ix_k'-x_i'x_k$ for the introduction of line coordinantes, then the Dirichlet form $(a,b,c)$ of the determinant D is given by the line with the following coordinates:
\begin{align}
\left\{\begin{array}{rclrcl}
p_{12}&=& c\sqrt{\bar{D}}+\bar{c}\sqrt{D},& p_{13}&=& -\ii\left(c\sqrt{\bar{D}}-\bar{c}\sqrt{D}\right),\\
p_{14}&=& -2\left(b\sqrt{\bar{D}}+\bar{b}\sqrt{D}\right),& p_{23}&=& \ii\left(b\sqrt{\bar{D}}-\bar{b}\sqrt{D}\right),\\
p_{24}&=& a\sqrt{\bar{D}}+\bar{a}\sqrt{\bar{D}},& p_{34}&=& \ii\left(a\sqrt{\bar{D}}+\bar{a}\sqrt{\bar{D}}\right)\\
\end{array}\right.
\end{align}
the \textcolor{red}{ratios} of these coordinates, as one sees, always still contain the sphere toot $\sqrt{D\bar{D}}$, which is irrational in general.

Furthermore, in connection with this, one can state the special theorem, that the hyperbolic axes of the Picard group are, in only case, always “rational” lines of the projective space. Namely, a hyperbolic substitution furnishes, in the sense on p.465, the Dirichlet form $(\alpha\gamma,\delta-\alpha,-\alpha\beta)$, whose determinant $D=(\alpha+\delta)^2-4$. Now since, for that hyperbolic substitution, $(\alpha+\delta)$ is real and absolute \textcolor{red}{here}, then one has a real positive D. Here, therefore, $\sqrt{D\bar{D}}$ is a rational whole number, from which our assertion emerges.

From this, we may now easily succeed in proving a supplementary theorem, concerning the indefinite Hermitian forms: Not only to each such form does there belong a hyperbolic rotation subgroup inside the Picard group (as proved above), but also, conversely, “each” such subgroup is the reproducing group of a determinate indefinite Hermitian form associated to it, so that, with the theory of the latter, that of those subgroups is also exhausted at the same time. In the polar plane of the \textcolor{red}{here} of an arbitrarily chosen hyperbolic rotation subgroup of the Picard group, namely, always lie infinitely many hyperbolic axes, i.e. , rational lines. Two of these lines suffice, in order to recognize, in the polar plane in question, a “rational” plane; to it, consequently, actually belongs a Hermitian form.
To the study of the indefinite Hemitian forms, finally, the following important argument also attaches itself.

If $(a,b_1,b_2,c)$ is an arbitrary such form, then the representing “plane” of this is given by (2). The greatest common factor of \textcolor{red}{here}, we imagine removed, whereby, let the equation reduce itself to:
\begin{align}
a_{41}x_1+a_{42}x_2+a_{42}x_3+a_{44}x_4=0.
\end{align}
In consequence of simple considerations, one can then forma quaternary integral unimodular substitution:
\begin{align}
\tau_i=a_{i1}x_1+a_{i2}x_2+a_{i3}x_3+a_{i4}x_4, i=1,2,3,4\ldots
\end{align}
whereby, for $i=4$, the coefficients are exactly the whole numbers coming up on (4).

By means of the transformation (5), the form $(x_2^2+x_3^2-x_1x_4)$ goes over into a new integral \textcolor{red}{here} quadratic form $F(\tau_1,\tau_2,\tau_3,\tau_4)$; the operations of the (projective) Picard group, however, furnish integral unimodular $\tau_i$-substitutions, which will clearly form the “reproducing group” of the form $F(\tau_i)$.

In particular, the substitutions of the reproducing group belonging to $(a,b_1,b_2,c)$ will furnish $\tau_i$-substituions, for which the plane $\tau_i=0$ goes over into itself, and which accordingly were the form:
\begin{align}
\left\{\begin{array}{rcl}
\tau_i'&=&\alpha_{i1}\tau_1+\alpha_{i2}\tau_2+\alpha_{i3}\tau_3+\alpha_{i4}\tau_4, i=1,2,3\ldots,\\
\tau_4'&=&\alpha_{44}\tau_4.
\end{array}\right.
\end{align}

The whole number $\alpha{44}$, as divisor of the determinant of the substitution, i.e., as a divisor of 1, is equal to $\pm1$, and one therefore has:
\begin{align}
\left|a_{ik}\right|=\pm1, i,k=1,2,3\ldots.
\end{align}

If one now sets $\tau_4=0$, then $F(\tau_i)$ goes over in to an integral ternary form \textcolor{red}{here} “representable by the quaternary form $(x_2^2+x_3^2-x_1x_4)$” . The substitution of $\tau_4=0$, with this, $\tau_4'=0$, in the equation $F(\tau_i')=F(\tau_i)$, however, furnishes the result, that the ternary group:
\begin{align}
\tau_i'=\alpha_{i1}\tau_1+\alpha_{i2}\tau_2+\alpha_{i3}\tau_3, i=1,2,3\ldots,
\end{align}
to the obtained in the \textcolor{red}{here} manner from the reproducing group of $(a,b_1,b_2,c)$, represents either the entire reproducing group of the ternary form $f(\tau_1,\tau_2,\tau_3)$ or a subgroup contained in this group.


The theory of the Hermitian forms has led us in such a way to those groups to whose mo0re detailed investigation the following chapter is to be dedicated. Thereby, we also come back to the two examples considered above with $D=5$ and $D=7$; both groups, corresponding to our theorem, will be subgroups inside the reproducing groups of two particular ternary forms.


