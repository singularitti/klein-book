\section{Theory of the Gaussian forms in projective-geometrical form}

The division of the $\zeta$-halfplane into circular-arc triangles, as it belongs to the modular group, underlies the theory of Gaussian forms sketched in $\S 1$. Now considered above (p.74 ff.) the triangle net of the modular group in the “interior of the elliptic of the hyperbolic plane” and already indicated briefly there, that this “projective form” of the modular group also appears very suitable for the geometric theory of the Gaussian forms. A few further discussions concerning this matter will now be in place.

In order to give the structure of the theory as independently as possible, we proceed in the following manner: We interpret the three coefficients a,b,c of a Gaussian from directly, as homogeneous point coordinates in the plane and work ourselves the conic section, which is given by $D=0$, i.e, in detail, by $b^2-ac=0$. By suitable figure of the coordinate system, we are able to invest this conic section with the form of an ellipse; in the interior of this $D<0$, outside, $D>0$. The individual point with integral coordinates a,b,c will be able to be designed briefly as a “rotational” point of the plane.

Now the following determination yields itself immediately: The individual rotational point with the coordinates a,b,c in the interior of the ellipse id directly the representation of the definite Gaussian form $(a,b,c)$, and similarly, a rotational point in the exterior of the ellipse represents an indefinite Gaussian form $(a,b,c)$.

Two properly or improperly equivalent Gaussian forms $(a,b,c)$ and $(a',b',c')$ are connected with one another by means of the equation system:
\begin{equation}
\left\{
\begin{split}
a'&=\alpha\alpha^2+2b\alpha\gamma+c\gamma^2\\
b'&=a\alpha\beta+b(\alpha\delta+\beta\gamma)+c\gamma\delta\\
c'&=a\beta^2+2b\beta\delta+c\delta^2\\
\end{split}\right.
\end{equation}
where $\alpha$, $\beta$, $\gamma$, $\delta$ are for rational whole numbers of the determinant $1$ resp. $-1$. Thus equation system represents an integral unimodular collineation of the ellipse $D=0$ into itself; and if we collect all the substitutions of this kind, then, as was already determined on p.75, we are led back precisely to the projective form of the extended modular group. It follows: The equivalence of the Gaussian forms coincides precisely with the equivalence of the representing points with respect to the modular group.

Now, to the projective form of the modular group belongs the division of the interior of the ellipse into a net of rectilinear triangles given in figure 13 on p.75. This net is reproduced and \textcolor{red}{here} in figure 170; 
%\begin{figure}
%\end{figure}
and thereby, the usual initial space is first worked heavily for the group of the “first kind, and then also, \textcolor{red}{here} particular symmetry lines of the net, of which one, as joining-line of the points $\zeta=0$ and $\zeta=\infty$, symmetrically divides the initial space just mentioned, while the other represents a side of that double triangle. We shall immediately have to make use of these two symmetry lines for the indefinite forms. Outside the ellipse, in \textcolor{red}{here} distinction to this, as we already concluded on p.76 from the theory of indefinite Gaussian forms, the modular group was improperly discontinuous; here, therefore, there is no division into finitely extended discontinuity domains.

These geometrical relations become decisive for the projective form of the theory of the Gaussian forms. 

The representing point of a definite form lies in the interior of the ellipse. We call the form “reduced” in case the associated point belongs to the initial space (cf. figure 170). The reduction conditions in arithmetic form presented in (3) on p.449, in homogeneous coordinates, immediately yield the definition of the initial space. One only wants to notice, that the sides $a=0$ and $c=0$ of the coordinate triangle are the ellipse tangents at the points $\zeta=\infty$ resp. $\zeta=0$, while the joining-line of these two points furnishes the third side $b=0$. The quotients of the coordinates, however, are to be fixed more closely, so that the sides of the initial space are given by
\begin{equation}
a+\alpha\beta=0,\; a-\alpha b=0,\;a-c=0.
\end{equation}
From here, the arithmetic reduction conditions (3), (4) on p.449 arise directly, if one only yet odds, which boundary points of the initial space are to be counted as part of this space, as well as, that those conditions refer themselves to “positive” forms. The further development of the theory of the definite forms now fashions itself exactly as far the use as the $\zeta$-halfplane.

Things lie completely differently for the indefinite forms, since, outside the ellipse, no division into finite domains exists belonging to the group. Here, from the projective character of the consideration, it appears as the indicated path, to bing in, in place of the points outside the ellipse, its polar in the interior of the ellipse, as the representation of the individual indefinite form. With this, however, we are led back to the projective form of the Smith semicircles. An indefinite form is now called “reduced”, in case its representing line cuts the initial space; and just this requirement finds its expression in the arithmetic reduction condition (5) on p.449, namely, that:
\begin{equation}
a(a\pm b+c)<0
\end{equation}
is to hold either for one or both signs. The further continuation of the study then naturally also fashions itself here essentially as for the use of the $\zeta$-halfplane.

Moreover, it is yet to be added here, that the reduction condition (2) deviated from the original Gaussian reduction condition for indefinite forms. For the geometrical conception of this latter condition, compare the work of Hurwitc “\textcolor{red}{here}”\footnote{*}. There, a synthetic construction of the rectilinear triangle net belonging to the modular group is furnished, and indeed, the development is based on the so-called “elementary chords of the first and second kinds”, which are defined arithmetically, but are moreover identical with the symmetry lines of the net of Fig. 170 consisting of two resp. four triangle sides.

The Gaussian reduction conditions for an indefinite form now requires (stabled geometrically), that the representing line of the form $(a,b,c)$ intersect (in the interior of the ellipse) the two elementary chords heavily accentuated in figure 170, namely, first the chord of the first kind which joins the points $\zeta=0$ and $\zeta=\infty$, then the elementary chord of the second kind given by $a=c$, which joins the points $\zeta=\pm 1$. To this, furthermore, is added one more condition regarding the “direction of the arrow” of the representing lines; we are able, in connection to the positioning in figure 170, to express this condition thus: The direction of the arrow of the representing line is to pass over the elementary chord $a=c$ in the direction from below to above. The analytic expression of this condition is:
\begin{equation}
ac<0,\; \abs{\dfrac{-b+\sqrt{D}}{c}}<1,\;\abs{\dfrac{-b-\sqrt{D}}{c}}>1,
\end{equation}
where in the lost two inequalities, on the left sides are meant the absolute values of the numbers \textcolor{red}{here} in vertical strokes; and these are just the conditions, which Gauss sets up in Article 133 of the “Disquisitions \textcolor{red}{here}”\footnote{*}. For the further development of the geometric theory of the indefinite forms upon this foundation, we refer to the work of named, as well as to the (hand written) lectures of Klein \textcolor{red}{here}\footnote{**}.