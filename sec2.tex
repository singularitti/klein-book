\section{Introduction of the Dirichlet and Hermitian quadratic forms}
A binary quadratic form named after Dirichlet was the same form as a Gaussian form:
\begin{align}\label{eq:21}
ax^2+abxy+cy^2
\end{align}
but here, the coefficients a,b,c are to be whole complex numbers of the form $(m+ni)$ with rational whole numbers $m,n$ and correspondingly, let $x$ and $y$ be indeterminate whole complex numbers of this kind.
There are those forms, whose theory Dirichlet was laid at in his substantial paper \textcolor{red}{here}\footnote{\textcolor{red}{here}}. Hereby, as was already emphasized above(p. 92), we took the \textcolor{red}{here} for a model are built up the theory of the forms(1) purely arithmetic.

The coefficients $a,b,c$ of the Dirichlet form belong to the domain of those whole complex numbers, which are introduced by Gauss for the purpose of the theory of biquadratic residues \footnote{\textcolor{red}{here}}.
In this domain, the calculation rules of the ordinary or rotational remain in force without restriction, concerning which one wants to compare the first part of the paper of Dirichlet's just named or also the eleventh supplementary in Dirichlet-Dedekind, \textcolor{red}{here} \footnote{\textcolor{red}{here}}.
According to the language of Dedekind's, all whole complex numbers $(m+ni)$, in conjunction with all quotients of such numbers, form a "number field" or briefly "field" of the second degree; we will denote this number field from now on symbolically by $\Omega^{(2)}$ or $\Omega$.
As abbreviation for the Dirichlet form \ref{eq:21}, we utilize the notation $(a,b,c)$; the determinant of the form (a,b,c) is $D=b^{2}-ac$.
Now transform the Dirichlet form by means of the substitution:
\begin{align}\label{eq:22}
x=\alpha x' + \beta y' , y=\gamma x' + \delta y'
\end{align}
when $\alpha, \beta, \gamma, \delta$ are any fair whole numbers from $\Omega$ with a non-vanishing determinant. As result, there again arises a Dirichlet form $(a',b',c')$, whose coefficients are:
\begin{align} \label{eq:23}
\left\{\begin{array}{rcl}
a'&=& \alpha^2 a+2\alpha\gamma b+\gamma^2 c,\\
b'&=&\alpha\beta a+(\alpha\delta+\beta\gamma)b+\gamma\delta c,\\
c'&=&=\beta^2 a+2\beta\delta b+\delta^2 c,
\end{array}\right.
\end{align}
and whole determinant $D'$ calculates itself as:
\begin{align}\label{eq:24}
D'=(\alpha\delta-\beta\gamma)^2 D
\end{align}
the transformed form $(a',b',c')$ to be obtained thus is called "contained under the form $(a,b,c)$".
Of the two forms $(a,b,c)$ and $(a',b',c')$, should each be contained under the other, then each of the two  numbers D and D' must go into the other. Therefore, $(\delta\delta-\beta\gamma)$ is such a whole number from $\Omega$, whose reciprocal value is likewise a whole number of this number field. According to this, $(\delta\delta-\beta\gamma)$ must be equal to $\pm 1$ or $\pm i$ \footnote{\textcolor{red}{here}}. One now calls the two forms $(a,b,c)$ and $(a',b',c')$ properly or improperly equivalent, if the determinant $(\delta\delta-\beta\gamma)$ is equal $+1$ or $-1$\footnote{\textcolor{red}{here}}. From equation \ref{eq:24}, it then yields itself, that equivalent forms always have the same determinant; on the other hand, for transformation by as substitution of the determinant $\pm i$, a sign change of $D$ enters. However, we will agree, thaty by equivalence is always to be meant from now on simply the "proper" equivalent.
The substitutions, which effect the equivalence of the Dirichlet forms, in case we apply the non-homogeneous terminology, now lead immediately to the Picard group investigated in detail above (p. 76 ff). But we also arrive at just this group for the Hermitian forms, to which, first of all, use \textcolor{red}{new} return.
With inessential deviation from the original Hermitian notation\footnote{Compare, for this notation, the original paper, named on p. 92, of Hermite's in Crelle's Journal Bd. 47, p.343 ff.}, we had already(p.92) put the so-called Hermitian forms into the form:
\begin{align} \label{eq:25}
D=b\bar{b}-ac=b_{1}^{2}+b_{2}^2-ac
\end{align}
Here, for the individual form, $a$ and $c$ are to be real whole number, but $b$ and $\bar{b}$ conjugate complex numbers from $\Omega$; correspondingly $x$ and $\bar{x}$, as well as $y$ and $\bar{y}$, are to represent conjugate complex, but not more closely determined whole numbers of $\Omega$. A Hermitian form is thus conjugate with itself.
The whole number $b$ we write explicitly $b=b_{1} +i b_{2}$and utilize the schema $(a,b_1,b_2,c)$ of the far rational whole numbers $a,b_1,b_2,c$ as an abbreviation for the form \ref{eq:25}.
The determinant of the \ref{eq:25} is
\begin{align} \label{eq:26}
D=b\bar{b}-ac=b_{1}^{2}+b_{2}^2-ac
\end{align}
As one sees the determinant of a Hermitian form is always real, in contradistinction to that of a Dirichlet form, which is any complex whole number of the number filed $\Omega$. This situation causes us, (as for the Gaussian forms) to distinguish between definite and indefinite Hermitian forms, according as the determinant 1) is negative or positive. The significance of this core distinction we investigate further below.
Now transform the form $(a,b_1,b_2,c)$ by means of the simultaneous substitutions:
\begin{equation}
\begin{split} \label{eq:27}
x &= \alpha x' + \beta y' , y= \gamma x' + \delta y'\\
\bar{x} &= \bar{\alpha}    \bar{x}' + \bar{\beta}, \bar{y}=\bar{\gamma}\bar{x}'+\bar{\delta}\bar{y}'
\end{split}
\end{equation}
where $\alpha,\beta,\gamma, \delta$ are whole number from $\Omega$ of non-vanishing determinant, and $bar{\alpha}$ is conjugate to $\alpha$, $bar{\beta}$ is conjugate to $\beta$,,?. The result is again a Hermitian form $(a',b_{1}',b_{2}',c')$ with the coefficients:
\begin{align} \label{eq:28}
\left\{\begin{array}{rcl}
a' &= a\alpha\bar{\alpha} + b\alpha\bar{\gamma} + \bar{b}\gamma\bar{\alpha} + c\gamma\bar{\gamma}\\
b' &= a\alpha\bar{\beta}  + b\alpha\bar{\delta} + \bar{b}\gamma\bar{\gamma} + c\gamma\bar{\delta}\\
\bar{b}' &= a\beta\bar{\alpha}  + b\beta\bar{\gamma} + \bar{b}\delta\bar{\alpha} + c\delta\bar{\gamma}\\
c' &= a\beta\bar{\beta}  + b\delta\bar{\beta} + \bar{b}\delta\bar{\gamma} + c\delta\bar{\delta}
\end{array}\right.
\end{align}
and of the following determinant:
\begin{align} \label{eq:29}
D' = (\alpha\delta-\beta\gamma)(\bar{\alpha}\bar{\gamma}-\bar{\beta}\bar{\gamma})D
\end{align}
The transformed form $(a',b_{1}',b_{2}',c')$ thus obtained is called ?contained under the form $(a,b_1,b_2,c)$?.
Here 1) and 1)' always have the same sigh, so that a definite(indefinite) form always can \textcolor{red}{here} be contained under a definite(indefinite) \textcolor{red}{here} again.
If, of the two forms $(a,b_{1},b_{2},c)$ and $(a',b_1',b_2',c')$, each is to be contained under the other, then for this it is necessary and sufficient, that $\alpha\delta-\beta\gamma$ is either equal to $\pm 1$ or $\pm i$. Thereby, there is a distinction between the Dirichlet forms, insofar as the determinant $D'=D$ for $\alpha\delta-\beta\gamma= \pm i$ as well as for $\alpha\delta-\beta\gamma = 1$, in consequence of \ref{eq:29}. However, one also speaks \textcolor{red}{here} of proper resp. improper equivalence only then, if $\alpha\delta-\beta\gamma = 1$ resp. $=-1$. If $\alpha\delta-\beta\gamma = \pm i$, then let the forms be called "equivalent in the extended sense"\footnote{Hermite distinguishes l.c. on p.350 \textcolor{red}{here} orders of improper equivalence, corresponding to the cases $\alpha\delta-\beta\gamma = -1, +i, -i$}.
As one sees, we are also led to the Picard group here again. We shall consequently make corresponding use of the latter group in the treatment of the fundamental problems at the theories of the Dirichlet and Hermitian forms, as of the modular group for the Gaussian forms. The problems to be treated, however, are the following: if two of the same determinant are given, then it is to be decided, whether they are equivalent or not; and in the former case, all substitutions are to be given, which transform the one form into the other. It will be possible, in the treatment of these tasks, to embark on a lone of development, which finds its model in the investigation of the Gaussian forms on the basis of the modular group\footnote{See, moreover, for the historical development of the theory of the Dirichlet and Hermitian forms, the references given on p.92. ff.}.