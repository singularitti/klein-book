\section{the representing group of the Hermitian forms belonging to the determinant $D=7$}

In general approaches developed are also to be carried through for the case of the determinant $D=7$, since, in many aspects, the previous examples still fashion themselves too elementary, while for $D=7$, e.g., in the determination of the generators of the reproducing group, the general rules on p.472 will came into force.

Since for $D=7$ parabolic substitutions are not able to come up, then, as easily seen, only the reduction conditions (3) on p.463 come into force. By discussion of these, one enumerates, in all, 46 pairwise inverse reduced forms.
 
If we now attach to the principal form $(1, 0, 0, -7)$ are apply to it the algorithm of continuous reduction, then there arises a net, which is already composed of all 66 partial polygons. Thus follows: There is only one single class of indefinite forms of the determinant $D=7$, which, as such, is self-inverse.
%\begin{figure}
%\centering
%\includegraphics[width=0.7\linewidth]{}
%\end{figure}

In figure 166 half of the net, with 33 partial polygons, is schematically produced, and, as well, the reduced forms are inscribed everywhere, as too, the bounding \textcolor{red}{tires} of the individual partial polygons and the boundary curves of the entire net are most closely depicted(?) in the manner agreed upon above (p.473Z). In order, however, to fashion the figure symmetrically, the two partial polygons of the reduced forms $(-2,0,-1,3)$ and $(-2,0,1,3)$ have \textcolor{red}{here} cut up into two pieces (and indeed, by means of the symmetry planes of the double pentahedron serving as initial space). One finds these four pieces, which represent quadrangles, on the right one and left sides of the figure; they are coordinated to one another by the arrow denoted by $V_2$.

The sides of the polygon of the second kind obtained, for the most part, belong t the first kind. We have there, first, be two sides already mentioned coordinated to one another by the arrow $V_2$, and then still three further pairs of boundary curves, which are mapped onto one another by the arrows denoted by $V_1$, $V_3$ and $V_4$. One will easily verify the results of the figure everywhere here; thus, e.g., one arrives, from the partial polygon of the form $(-2,1,-2,1)$, by passing over the side 2, i.e, by exercise of the associated transformation given on p.479, in fact, into the partial polygon of the form $(-2,-1,-2,-1)$, \textcolor{red}{here} by $V_3$.

However, there must now necessarily be sides of the second kind, and these are the boundary curves bounded by $\overline{e_1e_2}$, $\overline{e_6e_4}$ and $\overline{e_3e_5}$ in the figure, whereby the Katter is divided by its midpoint $\overline{e_4}$ into the two pieces $\overline{e_3e_4}$ and $\overline{e_4e_5}$. Namely, we are not concerned here with symmetry lines; rather, the sides $\overline{e_1e_2}$ and $\overline{e_3e_4}$ are to be mapped onto $\overline{e_4e_5}$ and $\overline{e_6e_4}$. Thereby, corresponding to the character of the operation of the second kind, in each case a reversal of the individual side enters, so that, e.g., the vertex $e_1$ is mapped onto $e_3$ and $e_2$ onto $e_4$. One will very easily verify these remarks with the figure in hand; e.g., the form $(2,-1,0,-3)$, positioned in figure 166 below on the right, is carried over into the form $(2,-1,0,-3)$ by exercise of the transformation $(2,4)$ given on p.479, which in fact is inverse from situated at the vertex $e_3$.

The net of the 33 partial polygons is now to be positioned on the representing hemisphere of the principal form $(1,0,0,-7)$. This will yield there a symmetrically shaped circular-arc polygon with ten vertices; and one easily determines, that all \textcolor{red}{here} angle, are right. Thus, one needs only notice, e.g., for the vertex situated at $e_1$, that the edge $(2,4)$ stands orthogonally to the side $\mathrm{1}$ of the pentahedron etc.

The operations $V_1, V_2, V_3, V_4$ now receive a uniquely determined interpretation as substitutions which transform the form $(1,0,0,-7)$ into itself. We also add, not the same time, the substitutions $V_5$ and $V_6$, not drawn in the figure, which, in the manner just described, transform $\overline{e_1e_2}$  into $\overline{e_3e_4}$ resp. $\overline{e_6e_4}$ into $\overline{e_4e_5}$, and which carry $(1,0,0,-7)$ over into $-1,0,0,-1)$. In order then, to read off the interpretation of $V_1, \ldots, V_6$ from the net of partial polygons, we must, beginning from the quadrangle of the form $(1,0,0,-7)$, lay the “closed” path, corresponding to the individual substitution $V_i$, through the net and combine the generators $S,T,\ldots$, hereby arranged next to one another, according to the pattern of their sequencing. One reads off directly form figure 166:
\begin{align}
\begin{array}{cc}
V_1=S & V_2=VT^{-2}UT^2UT^{-2}V\\
V_3=VT^{-1}SVT^{-1}UVT^2SV & V_4=VSVSTUVSVUTSVSV\\
V_5=T^2(T,U)VUT^{-1}SVSV & V_6=T^{-2}(T^{-1},U)VUTSVSV
\end{array}
\end{align}
Hereby, the substitutions corresponding to the edges $(2,4)$ and $(3,4)$ are denoted $(T,U)$ and $(T^{-1},U)$ for smart. The result of actually calculating out the generators $V_1,\ldots,V_6$ together with the remaining results, we clothe in the following theorem: The group of all substitutions contained in the unimodular Picard group of the first kind, which either transform the principal form $(1,0,0,-1)$ of the determinant 1 into itself, or into its inverse form $(-1,0,0,7)$, is a principal circle group of the second type (considered in the $\zeta$-plane), which has the following six substitutions as a system of generators:
\begin{align}
\left\{
\begin{array}{ccc}
V_1=\left(\begin{array}{cc}
\ii & 0\\
0 & -\ii\\
\end{array}\right) &
V_2=\left(\begin{array}{cc}
5+2\ii & 14\\
2 & 5-2\ii\\
\end{array}\right) & 
V_3=\left(\begin{array}{cc}
2+2\ii & 7\\
1 & 2-2\ii\\
\end{array}\right) \\
V_4=\left(\begin{array}{cc}
2+5\ii & 14\\
2 & 2-5\ii\\
\end{array}\right) & 
V_5=\left(\begin{array}{cc}
-3+2\ii & 7+7\ii\\
-1+\ii & 3+2\ii\\
\end{array}\right) & 
\\
V_6=\left(\begin{array}{cc}
3+2\ii & 7-7\ii\\
-1-\ii & -3+2\ii\\
\end{array}\right) & &\\
\end{array}
\right.
\end{align}
the substitutions $V_2,V_3,V_4$ are hyperbolic, and their fixpoints lie at:
\begin{align}
\zeta=\pm \sqrt{6}+\ii, \zeta=\pm\sqrt{3}+2\ii, \zeta=\frac{\sqrt{3}+5\ii}{2}
\end{align}
on the principal circle; $V_5$ and $V_6$ are loxodromic, and their fixpoints (naturally situated similarly on the principal circle) are :
\begin{align}
\zeta=\frac{3\pm\sqrt{5}+\ii(3\mp\sqrt{5})}{2}, \zeta=-\frac{(3\pm\sqrt{5})+\ii(3\mp\sqrt{5})}{2}
\end{align}
With this, at the same tine, all means are obtained, in order to geometrically construct, in the $\zeta$-plane, the polygon of the present principal circle group. Namely, this polygon has a symmetry line in the imaginary $\zeta$-axis. But if we extend with the reflection $\bar{V}$ in that axis, then $V^{\pm 1}_2\bar{V}$, $V^{\pm 1}_3\bar{V}$, $V^{\pm 1}_4\bar{V}$ become reflactions, where symmetry circles are directly the boundaries belonging to $V_2$, $V_3$, $V_4$. On the strength of this form (1), one concludes, as the equations of the boundary curves in question:
\begin{align}
\xi^2+\eta^2\mp 5\xi-2\eta+7=0, \xi^2+\eta^2\mp 4\xi-4\eta+7=0, \xi^2+\eta^2\mp 2\xi-5\eta+7=0.
\end{align}
The circles still remaining, however, are likewise easily determinable; \textcolor{red}{here}, for the individual circle, one can always give three circles, to which it runs orthogonally: the equations of the circle in question are:
\begin{align}
5(\xi^2+\eta^2)\mp 28\xi+35=0, 5(\xi^2+\eta^2)-28\eta+35=0.
\end{align}
On the strength of this, the precise shape of the discontinuity domain “ of the second type” is given in figure 167. One is concerned with a discontinuity domain of the “first kind” consisting of two separately situated circular-arc polygons, symmetric with respect to the principal circle. The coordination of the boundary curves is altered with respect to figure 166, insofar as the side $\overline{e_1e_2}$ is now mapped by the loxodramic substitution $V_5$ onto the side $\overline{e_3'e_4'}$ at the outer polygon, and insofar as the corresponding \textcolor{red}{ring} hold for all other sides, which were of the second kind on the hemisphere.

In order to obtain the discontinuity domain of the reproducing group, we first go back to the representing hemisphere of the principal form, divide the decangle there into two septangles by the symmetry plane  (cf. figure 163) and exercise the substitutions $V_5$ resp. $V_6$ on these two hept-angles. We attach the hept-angles to be thus obtained onto the previous ones, as figure 163 immediately carries rgus out, again in the $\zeta$-plane. There arises a completely rectangular sixteen-angle as discontinuity domain, whereby, it is to be noticed, that at the upper part of figure 163, for the sake of clarity, the boundary curves were drawn imprecisely, namely too large.
%\begin{figure}
%\end{figure}

The coordination of the boundary curves is only first partially brought to expression in the figure. For the sides not considered, the common prescription holds, that they belong together with their sides symmetric with respect to the imaginary $\zeta$-axis (centerline of the figure). The generators let themselves be easily produced from (1); thus, e.g., $V_5\phantom{}'=V_5\phantom{}^2$, $V_6\phantom{}'=V_6\phantom{}^2$. For the sake of brevity, we ignore here the calculation of the complete system of generators.

In schematic arrangement, the discontinuity domain of the reproducing group of the form $(1,0,0,-7)$ is reproduced by figure 169. If we ignore the two fixpoints of the elliptic substitutions $V_1$ and $V_1'$ of period two (cf. figure 163), then the remaining sixteen vertices arrange themselves into six cycles, as one will easily determines by means of the arrow in figure 169. Since it is a question here exclusively of rectangular vertices, then, as one likewise rends off easily from the figure, the angle sum of two of the vertex cycles in equal to $\alpha \pi$, that of the four others, $\pi$. On the strength of this, one immediately determines the theorem: The reproducing groups of the single class of indefinite Hermitian forms entering for $D=7$ belong to the family of the signature:

%\begin{figure}
%
%\end{figure}
\[
(1,6;2,2,2,2,2,2)---
\]

The approach of the immediate arithmetic definition of the reproducing group of the principal form $(1,0,0,-7)$ leads us to entirely similar considerations, as we carried them out above (p.432) for the principal form of the determinant $D=5$. Form (8) on p.453 follows, that the substitutions, which transform the principal form $(1,0,0,-7)$ into itself, are characterizible by:
\[
\alpha\overline{\alpha}-7\gamma\overline{\gamma}=1,\;\alpha\overline{\beta}=7\gamma\overline{\gamma},\;\alpha\delta-\beta\gamma=1.
\]

By discussion of these equations, we get the theorem: The reproducing group of the principal form $(1,0,0,-7)$ consists of all substitutions of the form:
\begin{equation}
\zeta'=\dfrac{\alpha\zeta+7\overline{\gamma}}{\gamma\zeta+\overline{\alpha}}
\end{equation}
contained in the unimodular Picard group. That these substitutions, all together, form a group, is immediately evident. Furthermore, one finds just as easily, that those substitutions, which transform $(1,0,0,-7)$ into $(-1,0,0,7)$, are characterized by $\delta=-\overline{\alpha}$ and $\beta=-7\overline{\gamma}$. The generating system (1) verifies these remarks.---

We would now also be able to treat the possible extensions of out present group, and indeed, as well as reflections, as by substitutions of determinant: However, one affects\textcolor{red}{?} these extensions very easily on the basis of the pertinent general approaches. Furthermore, we shall later obtain, from another aspect, the most comprehensive properly discontinuous principal circle group, in which the groups discussed here are contained.
