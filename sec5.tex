\section{Reduction theory of the Dirichlet forms}

If the determinant $D$ of a Dirichlet form $(a,b,c)$ is a perfect square inside the number field $\Omega$, then the roots $\zeta_1, \zeta_2$(cf p.455) belonging to $(a,b,c)$ are likewise numbers of $\Omega$ and accordingly furnish two points of the $\zeta$-plane, which each cusps of $\infty^2$ pentahedra of the halfspace division. The representing semicircle of the form $(a,b,c)$, under these circumstances, runs through only finitely many pentahedra, The treatment of these forms on the basis of the Picard group, which offers no difficulty, was no interest for us, since here, relations to the subgroups of the Picard group do not appear. We accordingly exclude, from now on, the forms with quadratic determinants.\footnote{*}

The equivalence theory again bses itself on the concept of the reduced forms: The Dirichlet form $(a,b,c)$ is to be called “reduced”, in case its representing semicircle was a segment of non-vanishing arc-length in common with the initial space of the Picard group characterized by (1) on p.457.

Since $D=0$, as a square is excluded, the representing semicircle of each form $(a,b,c)$ pushes into the $\zeta$-halfspacehalfspace. It follows immediately from this, that each Dirichlet form $(a,b,c)$ is equivalent with a reduced form. However, in order to arrive at the complete solution of the equivalence problem, we must first develop the theory of the reduced forms even further.

It is now very remarkable, that the presentation of the reduction conditions in arithmetic form leads to quite unsurveyable formulas, which we shall consequently not set up completely here.

Dirichlet treats the reduction theory in $\S$16 of his repeatedly named work and gives the following reduction conditions:
\begin{align}
	\abs{b}\sqrt{2}\leq \abs{a}\leq\abs{c},
\end{align}
understanding by $\abs{a},\ldots$ The absolute value of $a, \ldots$. These inequalities, \textcolor{red}{here} after the reduction conditions of the definite Gaussian forms (cf.(3) p.449), however, do not express the reduction condition just given in geometrical form. Consider, e.g., that the center of the representing semicircle of the form $(a,b,c)$, in consequence of (5) on p.455, is situated at $\zeta=-\frac{b}{a}$. This center, according to (1), would have at most the distance $\dfrac{1}{\sqrt{2}}$ from the origin, which by no means holds for the forms reduced in any sense.

Furthermore, the theory of Dirichlet’s beaed on the condition(1), at this points, \textcolor{red}{here} considerably behind the treatment of the Gaussian forms in the “Disquistions Arithmetic\textcolor{red}{?}”. The doctrine of the “periods of reduced indefinite forms”, which Gauss develops l.c.Art.136, immediately lets itself be correspondingly \textcolor{red}{here} at for the Dirichlet forms, as will emerge from the reduction condition \textcolor{red}{here} by us. It appears, that Dirichlet has not noticed this possibility.

Also in the paper of \textcolor{red}{here}’s normed on p.93, to which the representation here essentially attraches, the setting up of the “complete” reduction conditions in arithmetic form is ignored.

In order to see the finiteness of the number of reduced forms for given $D$, are furthermore need not know the complete arithmetic reduction conditions. For this, the two conditions:
\begin{align}
	\abs{a}<\sqrt{2\abs{D}},\abs{b}<\frac{\abs{a}}{\sqrt{2}}+\sqrt{\abs{D}}
\end{align}
for a reduced form $(a,b,c)$, indeed necessary, but not yer sufficient, already suffice. The correctness of this proceeds from the circumstance, that the representing semicircle of $(a,b,c)$ was not the center $\zeta=-\dfrac{b}{a}$ and radius $\dfrac{\sqrt{\abs{D}}}{\abs{a}}$. The vertex of the double pentahedron (1) on p.451 situabled closest to the $\zeta$-plane has the coordinate $\nu=\dfrac{1}{\sqrt{2}}$; the first condition (2) consequently brings to expression, that the radius $\dfrac{\sqrt{\abs{D}}}{\abs{a}}$ for a reduced form must be greater than this $\nu$. The projection of the double pentahedron (1) on p.457 into the $\zeta$-plane lies inside a circle with the radius $\dfrac{1}{\sqrt{2}}$ about $\zeta=0$. The center of the representing semicircle of a reduced form may consequently not attain the disturbance $\dfrac{1}{\sqrt{2}}+\dfrac{\abs{D}}{\abs{a}}$, since this semicircle could otherwise have no point in common with double pentahedron. Form this requirement, one will immediately hand off the second inequality(2). Notice moreover, that on account of the exclusion of purely quadratic $D$, the number always has a value different from \textcolor{red}{$\mathbf{Q}$}.

Since the number field $\Omega$ only furnishes a bonded number of whole numbers, whose absolute values do not exceed a fixed finite bond, \textcolor{red}{here} in consequence of (2), for given $D$, first $a$, then also $b$, and with this, $c$ too, are restricted to a finite number of values: For given determinant $D$, there is only a finite number of reduced forms and, with this, only a finite number of form classes.

The next development now ties itself in precisely with Stephen Smith’s theory of the indefinite Gaussian forms laid at in “M” I on p.250ff. Since the endpoints $\zeta_1, \zeta_2$ of the representing semicircle of a Dirichlet form $(a,b,c)$ of non-quadratic determinant are not parabolic points, then this semicircle runs through infinitely many double pentahedra toward $\zeta_1$, as well as $\zeta_2$, and appears cut up in such a way \textcolor{red}{here} a chain of infinitely many segments. The transformation of \textcolor{red}{here} are of these segments into the initial space furnishes a substitution, which carries $(a,b,c)$ over into a reduced form.

If we transfer all segments into the initial space, then we are also able here to hold first to their original equivalence. Namely, if we pursue the individual segment, as the semicircle of a reduced form, from the initial space into a neighingboring double pentahedron, then the segment of this circle lying in the latter is thrown back into the initial space by a determinate one of the fine substitutions known here from p.87:
\begin{align}
    S=\left(\begin{array}{cc}
    \ii & 0\\
    0 & -\ii\\
    \end{array}\right), 
    T^{\pm 1}=\left(\begin{array}{cc}
    1 & \mp 1\\
    0 & 1\\
    \end{array}\right), 
    U=\left(\begin{array}{cc}
    0 & -1\\
    1 & 0\\
    \end{array}\right), 
    V=\left(\begin{array}{cc}
    \ii & 1\\
    0 & -\ii\\
    \end{array}\right),
\end{align}
and furnishes here the immediately following term in the chain of the segments running through the initial space. Foe the forms, this signifies, that the substitution (3) coming into application transforms the reduced form belonging to the first segment into a “neighboring” similarly reduced form. By the algorithm thus obtained, one obviously obtains all the reduced forms equivalent with the proposed form and finds for the latter, at the same time, a determinate sequence. We designate this algorithm as the “process of continuous reduction”.

The previous consideration, moreover, requires an exposition, in case the representing semicircle passes through edges or vertices of the double pentahedra at our division of the $\zeta$-halfspace. In this case, among the segments in the initial space such will occur, which end in edges resp. Vertices of this domain, and whose associated semicircles next pass over into double pentahedra, which are connected with the initial space only in edges resp. vertices. In order to continue the chain of reduced forms beyond such a place, one obviously cannot apply one of the substitutions(3); rather, one at twelve particular additional substitutions is now to be exercised, which one will easily define from the eight types and the four non-parabolic vertices at the initial space.

We now join the result obtained with the finiteness of the number of all reduced forms of gived determinant $D$ already verified. It yields itself, exactly as for the indefinite Gaussian forms in “M” I on p.260, that in the chain of the reduced forms of non-quadratic determinant $D$ belongs a “finite-termed” so called “period of reduced forms”; and one can generate the entire period by means of the process of continuous reduction from an individual form of the period.

It will now scarcely still be necessary, to say, now, on the strength of this, we shall decide concerning the equivalence of two proposed forms $(a,b,c)$ and $(a',b',c')$. One must produce the periods of reduced forms belonging to the two forms. These form periods must be identical, in case we are to have equivalence of $(a,b,c)$ and $(a',b',c')$.

As an example, we first consider the determinant $D=1+\ii$, which represents a prime number of the field $\Omega$. Here $\abs{a}<\sqrt{s\sqrt{2}}$, so that, for a, only the eight values, $\pm 1, \pm\ii, \pm(1+\ii), \pm(1-\ii)$, are admissible, of we would otherwise have to do with a reduced form. If now, $a=\pm 1$ or $\pm\ii$, then the second reduction condition (2) on p.460 yields, that b can have the nine values $0, \pm 1, \pm\ii, \pm(1+\ii), \pm(1-\ii)$. For $a=\pm(1\pm\ii)$, the values $b=\pm 2$ and $\pm 2\ii$ are also admissible; however, on account of $b^2-ac=1+\ii$, b must now be divisible by $(1+\ii)$, so that the values $b=\pm 1, \pm\ii$ are to be excluded. All together, one thus obtains a system of 72 forms, pairwise inverse to one another.

Of these 72 forms, however, only eight show themselves to be reduced, which latter form a single self-inverse form period. The eight forms of the period are, in the correct order, the following:
\begin{align}
    (1,0,-1-\ii),(1,1,-\ii),(-\ii,-1,1),(\ii,0,-1+\ii),\\
    (-\ii,0,1-\ii),(i,1,-1),(-1,-1,\ii),(-1,0,1+\ii)
\end{align}
the substitutions to be exercised here are, in order, $T^{-1}, U, V, S, V, U, T$.

As one sees, the form period consists here of two symmetric values, expect that each two forms of the period symmetric to one another furnish, not identical, but inverse forms. One recognizes immediately, \textcolor{red}{here} this always and only then occurs, if the representing semicircle meets one, and with this, infinitely many, edges of the pentahedral division perpendicularly, which belongs to elliptic substitutions of the period 2; we come back to this in the next paragraph.

By utilization of this circumstance, moreover, one can quite drastically shorten the setting up of the form periods for latter valued $D$. If, e.g., we choose $D=2+\ii$, then the first condition (2) on p.460 yields the inequality $\abs{a}<\sqrt{2\sqrt{5}}$, so that only the values $0, \pm 1, \pm\ii, \pm(1+\ii), \pm(1-\ii), \pm 2, \pm 2\ii$ are admissible for $a$. Now the center of the individual representing semicircle lies at $\zeta=-\dfrac{b}{a}$; and since the values of a just given are all divisors of 2, then it I exlusively a matter of points, at which rectilinear edges of the pentahedral division should perpendicular to the $\zeta$-plane (cf.Fig.19 on p.81). The individual semicircle consequently cuts the associated edge perpendicularly.

Further, one can now so transport the representing semicircle in question by exercise of a substitution $\zeta'=\pm\zeta+m+n\ii$ , that the rectilinear edge of the pentahedral division intersected perpendicularly by it becomes one of the six edges of this kind belonging to the initial space. Since the coefficient a herby experiences at most assign change, the form remains reduced, as easily seen. In such a way, one has \textcolor{red}{here} to one ore the twelve following terms:
\begin{align}
    \pm(1,0,-2-\ii),\pm(\ii,0,-1+2\ii),\pm(1+\ii,1,-1),\\
    \pm(1-\ii,-\ii,-1+2\ii),\pm(1-\ii,1,-\ii),\pm(1+\ii,\ii,-2+\ii),
\end{align}
as one easily determines by discussion of the six edges of the initial space in question.
The form $(1+\ii,1,-1)$ goes over into the form $-(1+\ii,\ii,-2+\ii)$ by the substitution $\left(\begin{array}{cc}
\ii & \ii\\
0 & -\ii\\
\end{array}\right)$, and one correspondingly recognizes the equivalence of the forth and fifth forms of the system just given. Since here, however, each form is equivalent with its inverse, one needs only yet to exercise the “algorithm of continuous reduction” individually to the four forms:
\begin{align}
    (1,0,-2-\ii),(\ii,0,-1+2\ii),(1+\ii,1,-1),(1-\ii,1,-\ii).
\end{align}

The effectuation at the calculation furnishers two form periods “equivalent in the extended sense”(cf, p.453), each six termed. It will suffice to present one of these periods:
\begin{align}
    (1,0,-2-\ii),(1,1,-1-\ii),(-1-\ii,-1,1),\\
    (1+\ii,1,-1),(-1,-1,1+\ii),(-1,0,2+\ii).
\end{align}
The substitutions coming into application here are
\begin{align}
    T^{-1},U,\begin{array}{cc}
    \ii & 1+\ii\\
    0 & -\ii\\
    \end{array}, U, T,
\end{align}
in the Third place, we therefore have an example of those twelve substitutions not belonging to the system (3), which are also to be applied in the continuous of the present case, according to p.462(above).