\section{The Transformation of the Dirichlet forms and themselves}

The second problem of the theory of equivalence, which amounts to giving all substitutions which transform a proposed Dirichlet form $(a,b,c)$ into itself, is also immediately solvable on the basis of the process of “continuous reduction”.

A substitution of the unimodular Picard group, which carries $(a,b,c)$ over into itself, will transform the representing semicircle into itself with maintenance of the direction of the arrow. By pushing aside the particular case, that the semicircle is on elliptic axis; it can only be a question of loxodromic or hyperbolic substitutions, which have the feet $\zeta_1, \zeta_2$ or the semicircle as fixpoints.

But now, the semicircle consists of a chain of segments which we build up from infinitely many equivalent finite-termed periods. By reserving particular consideration of the elliptic axes, we thus have the following result: A Dirichlet form $(a,b,c)$ of non-quadratic D always and only lets itself be transformed into itself by the substitutions of a cyclic loxodromic or hyperbolic subgroup, which has the semicircle of the form as orbit curve.

This theorem, however, is also immediately reversible: to each cyclic loxodromic or hyperbolic subgroup belongs a Dirichlet form of non-quadratic determinant. With the treatment of the Dirichlet forms, therefore, the theory of the cyclic subgroups in question is disposed of at the same time.

The process of continuous reduction now also furnishes the means of calculating the generating substitution of the cyclic subgroup belonging to the form $(a,b,c)$. To this end, one has only to combine these substitutions, which came into application in setting up the individual terms of the \textcolor{red}{here} period. Thus, e.g., the reduced form $(1,0,-1-\ii)$ of the determinant $(1+\ii)$, considered in the previous paragragh, is transformed into itself by the substitution:
\begin{align}\label{eq:61}
    T^{-1}UVSVUTS=\left(\begin{array}{cc}
    1-2\ii & 2-2\ii\\
    -2\ii & 1-2\ii\\
    \end{array}\right),
\end{align}
as generator of the associated cyclic group; and in the same sense, to the form $(1,0,-20\ii)$ belongs the substitution:
\begin{align}\label{eq:62}
    T^{-1}U\cdot\left(\begin{array}{cc}
    \ii & 1+\ii\\
    0 & -\ii\\
    \end{array}\right)\cdot UTS=\left(\begin{array}{cc}
    2-\ii & 3-\ii\\
    1-\ii & 2-\ii\\
    \end{array}\right).
\end{align}

Only notice, that the last substitution S, which in each case engineers the transition to the first term of the following period, was not given above.

Furthermore, the substitutions, which carry a Dirichlet form $(a,b,c)$ over into itself, can also be produced in an arithmetic way, essentially as this was done for the Gaussian forms in “M” I on p.252ff. The substitutions are then to be put into the form:
\begin{align}
    \left(\begin{array}{cc}
    \frac{t-bu}{\sigma} & -\frac{cu}{\sigma}\\
    \frac{au}{\sigma} & \frac{t+bu}{\sigma}\\
    \end{array}\right).
\end{align}
Thereby, $\sigma$ denotes the “divisor” of the form $(a,b,c)$, i.e,m the greatest common divisor of $a, 2b, c$, which, e.g, for “primitive” forms is either 1 or $(1+\ii)$ or finally 2. For $t$ and $u$, however, are to be substituted all solutions of the generalized “\textcolor{red}{here} equation”:
\begin{align}\label{eq:64}
    t^2-Du^2=\sigma^2
\end{align}
in whole complex numbers of the field $\Omega$. From the previous theory follows immediately, that the Pellian equation \ref{eq:64}, for non-quadratic $D$, is always solvable by $\infty'$ pairs of while complex numbers $t, u$, and that all these solutions let themselves be calculated from the “smallest” one of them according the logarithm \textcolor{red}{here} here from the Gaussian forms. Thus, e.g., one finds from \ref{eq:61} and \ref{eq:62},
\begin{align}
\begin{array}{ccc}
    \text{for }D=1+\ii & t=1-2\ii &u=2\ii\\
    \text{for }D=2+\ii & t=2-\ii & u=1-\ii\\
\end{array}
\end{align}
in this specific situation, therefore, “smallest” solutions. The small have to make an important application in the next \textcolor{red}{here} of the solvability of the Pellian equation thus obtained\footnote{*}.

Now, those forms are to be investigated, for which the representing semicircle is an elliptic edge of the pentahedral(?) division. Of the eight edges to be counted as part of the double pentahedral (1) on p.457, however, six \textcolor{red}{here} upon parabolic points in the $\zeta$-plane, and thus do not come into consideration for us. The two remaining edges, on the other hand, have the feet:
\begin{align}
    \zeta_1,\zeta_2=\frac{-1\pm\ii\sqrt{3}}{2} \text{ and } \zeta_1,\zeta_2=\frac{\pm\sqrt{3}+\ii}{2}.
\end{align}
Here, therefore, in case we restrict ourselves to primitive forms, we are led to the two forms $(2,1,2)$ and $(2,-\ii,2)$ belonging to $D=-3$ resp. $D=+3$. These forms ( and \textcolor{red}{here} those equivalent with them) are the only Dirichlet forms of non-quadratic determinant, which, besides the substitutions of each of the cyclic groups already obtained above as associated, we also transformed into themselves by elliptic substitutions, Hereby, it is a question , in the individual case, of a cyclic group $G_3$ of order three. This $G_3$ unite itself with the cyclic group $G_\infty$, if we retain the terminology used on p.234; the latter group then furnishes all the substitutions of the present form into itself.——

Finally, a few more particular species of forms of interest are to be mentioned.

First of all, we consider the case, that the representing semicircle of $(a,b,c)$ cuts an elliptic edge, of the pentahedral division, belonging to the period to the period two, perpendicularly. The individual pentahedron has six edges of this kind; they are those edges which end in parabolic points. By exercise of the associated elliptic substitution, the representing semicircle is carried over into itself with reversal of the direction of its arrow. One recognizes immediately, in the strength of this, that we are concerned here with all those form classes, which are self-inverse. The representing semicircle carries infinitely many elliptic fixpoints here, among which every two following are another \textcolor{red}{here} “half-period”. All the substitutions of the Picard group, which transform the present form $(a,b,c)$ into itself or $(-a,-b,-c)$, form a group, which one has to design ate as a “hyperbolic” or “loxodromic dihedral group”(cf. p.346). For all non-quadratic determinant D, we get form classes at this kind; e.g., the “principal classes” furnished by $(1,0,-D)$ always belong here, Compare, moreover, the examples of the previous paragraph.

Secondly, we must still consider the case, that the representing semicircle of $(a,b,c)$ is either \textcolor{red}{here} in a symmetry semisphere orthogonally. Here we are led to the concept of the ambiguous form, and thus obtains two different species of ambiguous forms $(a,b,c)$. The cyclic group belonging to the individual ambiguous form is hyperbolic and, inside the Picard group of the second kind, is itself capable of extension by reflections. The indefinite Gaussian forms furnish examples of the first species of ambiguous Dirichlet forms, and the definite ones the second species.
