\section{Treatment of the equivalence problem for the definite Hermitian forms}

The “unimodular Picard group”, entering for the (proper) equivalence of the Hermitian and Dirichlet forms, is more closely investigated on p.35 ff. as a group $\Gamma_2$. There yielded itself, as “initial space” of this group, a double pentahedron of the $\zeta$-halfspace, which was defined by the conditions:
\begin{align}\label{eq:41}
	-\dfrac{1}{2}\leq \zeta <\frac{1}{2}, 0\leq\eta\leq\frac{1}{2}, \zeta^2+\eta^2+\nu^2\geq 1,
\end{align}
which the addition, that in the last two formulas, the equality signs are only then to hold, if $\zeta\leq 0$.

On the double pentahedron, we now establish the treatment of the equivalence problem of our forms.
First, if we were a positive Hermitian form, then the following definition is to be given a preeminient position: The positive Hermitian form $(a,b_1,b_2,c)$ is to be called “reduced”, if its representing point lies in the initial space at the unimodular Picard group. By substituting the coordinates(3) on p.455 of the representing point into \ref{eq:41}, here followes: The positive form $(a,b_1,b_2,c)$ is always and only then a reduced form, if the conditions hold:
\begin{align}\label{eq:42}
	-a\leq -ab_1\geq a, 0\leq 2b_2\leq a, a\leq c,
\end{align}
with the addition, that in the second and third formulas, equalitu signs only hold then, in case $b_1\geq 0$.

From the concept of the discontinuity domain, there yields itself: Each positive Hermitian form of two proposed forms, go, in both cases, to the reduced form (say with use of the pentahedral division). The latter must identical, if those forms are to be equivalent.

The request for all the substitutions, which effect the equivalence in the individual case, one reduces, in known fashion, to the task of ascertaining all substitutions, which transform a proposed form into itself. We answer this request, say for the reduced forms, in the following way: A reduced form $(a,b_1,b_2,c)$ will be transformed into itself by none of \textcolor{red}{here} $\zeta$-substitutions apart from the identity, incase its representing point is not situabled right on an edge of the elementary pentrohedron. On the other hand, one has two resp. three transformations, forming a cyclic group, of the form into itself, in case the representing point belongs to an edge, but to no vertex of the elementary pentahedron; and finally, we have for, six or twelve formations, forming a \textcolor{red}{here} resp. dihedral or tetrahedral group, in case the representing point is a vertex of the elementary pentahedron. All these remarks follow immediately from the character of the pentahedral division of the $\zeta$-halfspace belonging to the Picard group.

In order, after this, e.g., for the edge of the elementary pentahedron given by:
\begin{align}
	\zeta=-\frac{1}{2}, \eta=\frac{1}{2}, \nu>\frac{1}{\sqrt{2}},
\end{align}
to characterize all associated reduced forms, we have to set:
\begin{align}
	\frac{b_1}{a}=\frac{1}{2}, \frac{b_2}{a}=\frac{1}{2}, \frac{\sqrt{-D}}{a}>\frac{1}{\sqrt{2}}.
\end{align}
All the positive forms corresponding to this condition, we are able to collect together into the form
\begin{align}
	\alpha m x \bar{x}+m(1+\ii)x\bar{y}+m(1-\ii)\bar{x}y+(2m+n)y\bar{y},
\end{align}
where $m$ and $n$ are only two rational whole positive numbers. --- Naturally, to each edge of the elementary pentahedron belong infinitely many form classes; in contrast to this, the individual pentahedral vertex always yields only one class, as long as we restrict ourselves to “primitive” forms, i.e, to such, for which $(a,b_1,b_2,c)$, have no common factor \textcolor{red}{here} $\mathbf{1}$. The latter vertex or the vertex just discussed, e.g., furnishes the class of the determinant $D=-2$ belonging to the reduced form:
\begin{align}
	2\alpha\bar{\alpha}+(1+\ii)x\bar{y}+(1-\ii)\bar{x}y+2y\bar{y}.
\end{align}

For a given value of the determinant $D$, the \textcolor{red}{here} of the number of reduced forms and, with this, the finiteness of the class number of positive Hermitian forms is a simple consequence of the complete reduction conditions \ref{eq:42}. Namely, from these conditions, one easily derives the inequality:
\begin{align}
	a\leq \sqrt{-2D},
\end{align}
So that $a$, and with this, also $b_1$, and $b_2$, for given $D$, are restricted to a finite number of values. With $a, b, D$, however, $c$ is uniquely determined, After this, no difficulty stands in the way of the enumeration of all the reduced forms in the individual case $D$. Thus, e.g, for $D=-5$, one finds three closses with the reduced forms:
\begin{align}
	(1,0,0,5),(2,0,1,3),(2,1,0,3).
\end{align}
The representing point of the last form proceeds from the second one by the substitution $\zeta'=\ii \zeta$; the two forms, therefore, according to the language agreed upon on p.453, are equivalent in the extended sense.

