`\section{The reproducing groups of the Hermitian forms belonging to the determinant $D=5$}

The simplest example after the illustration of the developments of the last two paragraphs would be furnished \textcolor{red}{here} by those form classes, whose representing hemisphere are the symmetry hemispheres of the pentahedral division. But it reveals itself, that we arrive here only first to very simple relations: namely, one is led to only two different reproducing groups, of which one is the ordinary modular group, and the other the group of the circular-arc triangle of the angles $\dfrac{\pi}{2}, \dfrac{\pi}{4}, 0$. We shall later find occasion, to investigate these latter groups directly and accordingly, \textcolor{red}{here} not longer here by the examples in questions.

For the treatment of more instructive example, we first make a few more remarks regarding the numerical calculations to be carried through here.

First of all, for given $D$, one can set up for oneself a table of the reduced forms following the prescriptions on p.466.

For the individual hemisphere hereby entering:
\begin{align}\label{eq:91}
    a(\xi^2+\eta^2+\vartheta^2)+2b_1\xi-2b_2\eta+c=0,
\end{align}
the intersection with the initial space of the unimodular Picard group of the first kind is then to be determined. The first facial surfaces of the initial space correspond to the first generators $\textcolor{red}{here}, T, T^{-1},U,V$ and are, in this order, to be distinguished by numbers $\mathrm{1}$ to 5. The equations of these surfaces are, in order:
\begin{align}\label{eq:92}
    \eta=0, \xi=\frac{1}{2}, \xi=-\frac{1}{2}, \xi^2+\eta^2+\vartheta^2=1, \eta=\frac{1}{2}.
\end{align}

The boundary curves of the segment of the sphere \ref{eq:91} situated in the initial space is correspondingly to be described by numerals 1 through 5, according as they are furnished by the first, etc. sided \ref{eq:92}.

The hemisphere \ref{eq:91} can, however, in particular, run through one of the eight edges of the initial space. By then placing the numerals of the two participating facial surfaces next to one another in parametres, we use the symbols $(1,2), (1,3)$ for the designation of the edges and, with this, at the same time, of the respective sides of the intersection polygon \textcolor{red}{here} on the hemisphere \ref{eq:91}.

For the edges in question, in place of the generators $S, T, \ldots$, the substitutions given in (4) on p.470 appear. Furthermore, these eight edges are not conjugate. Three of them, namely $(2,4), (3,4)$ and $(4,5)$, belong to elliptic substitutions of period three; here we have ordinary sides of the partial polygons. The five other edges $(1,2),(1,3),(1,4),(2,5),(3,5)$ belong to the period two and consequently furnish symmetry circles of the polygon net to be constructed on the hemisphere (1) (cf. p.473). One need not continue the net of the partial polygons over and beyond such a side; for on the other side, indeed, the forms inverse to the previous ones will again find themselves, in the symmetric arrangement.

In the case occurring, these three edges of the “elementary pentahedron”, which symmetrically \textcolor{red}{here} the faces 1,4 and 5, obtain just this same character of symmetry by circles. To such symmetry lines of our polygon net we assign the symbols (1), (4), (5).

In order to make these prescriptions easily surveyable to our vision, in figure 160 is drawn the perpendicular projection of the initial space onto the $\zeta$-plane, whereby, as grand \textcolor{red}{here} appears a rectangle with the vertices $\zeta=\pm\dfrac{1}{2},\pm\dfrac{1}{2}+\dfrac{\ii}{2}$. The facial surfaces 1, 2, 3, 5 project themselves hereby into lines (which carry the corresponding numerals), the surface 4, on the other hand, furnishes the interior of the rectangle of the edges, $(1), (5), (1,2), (1,3), (2,5), (3,5)$ will project into points, while the fine remaining one furnish lines.
%\begin{figure}
%\centering
%\includegraphics[width=0.7\linewidth]{}
%\end{figure}

In one would now determine the intersection polygons of the hemisphere with the initial space, then next discuss the question, whether this polygon has a side 4. The projection of such a side on the $\zeta$-plane has the equation:
\begin{align}\label{eq:93}
    2b_1\xi-2b_1\eta+(a+c)=0,
\end{align}
as one sees immediately by combination of the relevant equations \ref{eq:91} and \ref{eq:92}. A side 4 will accordingly always and only then occur, if the line represented in \ref{eq:93} cuts the rectangle of figure 160. If this is not the case, then the intersection polygon sought is a quadrangle which four sides 1, 2, 5, 3. In each case, however, by addition of the position of the center and radius of the sphere \ref{eq:91}, one will easily inform oneself further on the carse of the intersection polygon.

It is here also to be given in general, at which “neighboring” forms one arrives, if one should leave a partial polygon over a side $1, 2, \ldots$. If the first form is $(a,b_1,b_2,c)$, and the neighboring one $(a',b_1',b_2',c)$, then according to (3) on p.461 and (3) on p.453, one has the following transformation formulas in the five cases in question:
\begin{enumerate}
    \item $a'=a, b_1'=-b_1, b_2'=-b_2, c'=c$,
    \item $a'=a, b_1'=a+b_1, b_2'=b_2, c'=a+2b_1+c$,
    \item $a'=a, b_1'=-a+b_1, b_2'=b_2, c'=a-2b_1+c$,
    \item $a'=c, b_1'=-b_1, b_2'=b_2, c'=a$,
    \item $a'=a, b_1'=-b_1, b_2'=a-b_2, c'=a-2b_2+c$,
\end{enumerate}
Next to these there arrange themselves the three further transofrmations referring themselves to the edges $(2,4), (3,4)$, and $(4,5)$:
\begin{align}
    \begin{array}{ccccc}
    (2,4) & a'=a+2b_1+c, & b_1'=b_1+c, & b_2'=b_2, & c'=c,\\
    (3,4) & a'=a-2b_1+c, & b_1'=b_1-c, & b_2'=b_2, & c'=c,\\
    (4,5) & a'=a-2b_2+c, & b_1'=-b_1, & b_2'=-b_2+c, & c'=c.\\
    \end{array}
\end{align}
We now go over to the proper subject of the present paragraph, namely, by means of the prescriptions just given, to investigate the reproducing groups of the form classes belonging to the determinant $D=5$.

One will designate the form $(1,0,0,-5)$ as the principal form of the determinant $D=5$. The “principal class” belonging to it was, as the effectuation of the preceding methods of investigation generally described shows, 48 reduced forms, pairwise inverse to one another. One of the two symmetric halves of the net is schematically reproduced in figure 161. In the 34 polygons, in each case, the associated reduced form is inscribed and the sides are characterized by their numerals everywhere. A side 2 is thereby always a side 3 for the neighboring polygons and conversely.  The remaining three sides, however, in this specific situation, retain their numerals for the neighboring partial polygon, corresponding to the circumstance, that the associated substitutions $S, U, V$ are of period two. Therefore, for these sides, in each case, only one numeral is inscribed. The boundary curves of the net remaining open will mostly be furnished by pentahedral edges; they lay themselves together into five symmetry lines, which are distinguished by the numerals $\Rmnum{1}$ through \Rmnum{4} in the figure. Of the boundary curves still remaining, one half is mapped onto the other by a generator of the reproducing group, which is indicated by the arrow appended in the figure. The four vertices surrounded with small circles are parabolic; the occurrence of such vertices was to be expected according to the criterion p.475.
%\begin{figure}
%\centering
%\includegraphics[width=0.7\linewidth]{}
%\end{figure}

The half-polygon furnished in such a way by the process of continuous reduction is a discontinuity domain of the second kind with five sides of the second kind, namely symmetry circles, and two sides of the first kin, which join together at a recumbent angle. Onto which hemisphere belonging to the form class we imagine the net positioned is \textcolor{red}{here} in itself; however, will\textcolor{red}{?} will utilize, most expediently, the representing hemisphere of the principal form $(1,0,0,-5)$ itself. Figure 161 then informs us, that the two non-parabolic vertices of the half-polygon are \textcolor{red}{here} at $\zeta=\pm 2,\vartheta=1$, and the four parabolic ones at $\zeta=\pm 2+\ii$ and $\zeta=\pm 1+2\ii$. The projection by orthogonal semicircles furnishes, in the $\zeta$-plane, the two polygons represented in figure 162, which comprise the discontinuity domain of that principal circle group of the second type contained in the unimodualr Picard group of the first kind, where substitutions transform $(1,0,0,-5)$ either into itself or into $(-1,0,0,5)$. The generators of this group, which were characterized more closely in figure 162 by arrows, are the following:
\begin{align}\label{eq:95}
    \left\{\begin{array}{ccc}
    S=\left(\begin{array}{cc}
    \ii & 0\\
    0 & -\ii\\
    \end{array}\right), & V_1=\left(\begin{array}{cc}
    2 & -5 \\
    1 & -2 \\
    \end{array}\right), & V_2=\left(\begin{array}{cc}
    3 & -5-5\ii\\
    1-\ii & -3\\
    \end{array}\right),\\
    V_3=\left(\begin{array}{cc}
    2 & -5\ii\\
    -\ii & -2\\
    \end{array}\right), & V_4=\left(\begin{array}{cc}
    3 & 5-5\ii\\
    -1-\ii & -3\\
    \end{array}\right), & V_5=\left(\begin{array}{cc}
    2 & 5\\
    -1 & -2\\
    \end{array}\right).
    \end{array}\right.
\end{align}
It is a question here, as one sees, exclusively of elliptic substitutions of period two, of which the last five in \textcolor{red}{here} the interior of the principal circle with the exterior.
%\begin{figure}
%\centering
%\includegraphics[width=0.7\linewidth]{}
%\end{figure}

In order to obtain the discontinuity domain of the reproducing group of $(1,0,0,-5)$ itself, we are able to restrict ourselves to the interior of the principal circle and position, beside the previous half-polygon, say along the side denoted by \Rmnum{3} in figure 161, a half-polygon symmetric with it, as figure 163 indicates this. The system of the generators described more closely in the figure is:
\begin{align}
    \left\{\begin{array}{cc}
    S=\left(\begin{array}{cc}
    \ii & 0\\
    0 & -\ii\\
    \end{array}\right) & 
    V_3'= V_3SV_3=\left(\begin{array}{cc}
    9\ii & 20\\
    4 & -9\ii\\
    \end{array}\right)
    \\
    V_1'=V_1V_2=\left(\begin{array}{cc}
    4+5\ii & 10-10\ii\\
    2+2\ii & 4-5\ii\\
    \end{array}\right) & 
    V5'= V_ 5V_3=\left(\begin{array}{cc}
    4-5\ii & -10-10\ii\\
    -2+2\ii & 4+5\ii\\
    \end{array}\right)
    \\
    V_2'=V_2V_3=\left(\begin{array}{cc}
    1+5\ii & 10-5\ii\\
    2+\ii & 1-5\ii\\
    \end{array}\right) &
    V_4'= V_4V_3=\left(\begin{array}{cc}
    1-5\ii & -10-5\ii\\
    -2+\ii & 1+5\ii\\
    \end{array}\right)
    \end{array}\right.
\end{align}
Above all else, we take note of the theorem: the reproducing groups of the principal class of the determinant $D=5$ are principal circle groups from the family of the signature:
\begin{align}
    (0,6,2,2,\infty,\infty,\infty,\infty)
\end{align}
One will easily carry out the transformation of the domain of figure 163 into a “canonical” polygon.---

The arithmetic law of formation of the principal-circle group was only first given indirectly so far, and indeed, that it consists of all substitutions of the unimodular Picard group, which, \textcolor{red}{here} into the form (1) on p.453, transform the form $(x\bar{x}-Sy\bar{y})$ into itself.

Now, however, we can also directly bring into account the properties of our substitutions hereby defined, and thus find, that we have to do with all substitutions, whose coefficients are whole numbers of $\Omega$ satisfying the three conditions:
\begin{align}\label{eq:96}
    \alpha\delta-\beta\gamma=1, \alpha\bar{\alpha}-5\gamma\bar{\gamma}=1, \alpha\bar{\beta}=5\delta\bar{\delta}.
\end{align}
In case none of the numbers $\alpha, \beta, \gamma, \delta$ vanishes, we conclude from the last equation \ref{eq:96}, that $\delta=\varepsilon\times\cdot\bar{\alpha}$, $\beta=\varepsilon\cdot 5 \bar{\gamma}$, understanding by $\varepsilon$ a \textcolor{red}{here} or fractional number of $\Omega$. The first equation \ref{eq:96} then furnishes $\varepsilon(\alpha\bar{\alpha}-5\gamma\bar{\gamma})=1$, so that, in consequence of the second, $\varepsilon=1$. The case of the \textcolor{red}{here} of $\beta$ (on account of the last and first equation \ref{eq:96}) are then not able to \textcolor{red}{here} too.
 
By this, the following theorem is proved: The reproducing group of the principal form $(1,0,0,-5)$ consist of all unimodular substituions:
\begin{align}
    \zeta'=\frac{\alpha\zeta+5\bar{\gamma}}{\gamma\zeta+\bar{\alpha}}
   \end{align} 
with whole complex coefficients $\alpha, \gamma$ belonging to the field $\Omega$ with this is obtained the full insight into the arithmetic law of formation of our groups; for that the substitutions in question form a group in their totality, is evident from a mere glance at the schema:
\begin{align}
    \left(\begin{array}{cc}
    \alpha & 5 \bar{\gamma}\\
    \gamma & \bar{\alpha}\\
    \end{array}\right)\cdot\left(\begin{array}{cc}
    \alpha' & 5 \bar{\gamma}\\
    \gamma' & \bar{\alpha}'\\
    \end{array}\right)=\left(\begin{array}{cc}
    \alpha\bar{alpha}'+5 \bar{\gamma}\gamma' & 5(\alpha\bar{\gamma}'+\bar{\alpha}'+\bar{\gamma})\\
    \bar{\alpha}\gamma'+\alpha'\gamma & \bar{\alpha}\bar{\alpha}'+5 \gamma\bar{\gamma}'\\ 
    \end{array}\right).
\end{align}
Furthermore, the law set up emerges directly for the generators \ref{eq:95}.---

We could now also treat the possible extensions of the principal-circle group in question. Since $(x\bar{x}-b'y\bar{y})$ is transformed into itself by the substitution $\left(\begin{array}{cc}
\ii & 0\\
0 & 1\\
\end{array}\right)$, our group permits the extension by addition of $\zeta'=\ii\zeta$ and one easily presents, say on the representing hemisphere, a polygon of the extended group by solving the polygon corresponding to the net of figure 161 discussed above.

However, at the same time, one can go one more step further, and also add the reflection in the imaginary $\zeta$-axis, which is, in fact, contained in the Picard group at the second kind. The polygon last mentioned then falls into four \textcolor{red}{here} symmetric and congruent circular-arc quadrangles with the angles $\dfrac{\pi}{4}, \dfrac{\pi}{2}, 0, \dfrac{\pi}{2}$, as this is represented by figure 164 in the projection onto the interior of the principal circle. We shall meet this group once more $(0,4,2,2,4,\infty)$ in the next chapter in another connection; we shall be able to show there, that it cannot, at the same time, we combined in a more comprehensive, similarly, properly discontinuous principal circle group as subgroup.---
%\begin{figure}
%\centering
%\includegraphics[width=0.7\linewidth]{}
%\end{figure}

With the principal class, now, all reduced forms of the determinant $D=5$ are not yet exhausted. There remain, in fact, yet two further form classes, which, however, show themselves to the equivalent in the extended sense. Let its office, that we give the net of the reduced forms for one of the two classes; half of this is drawn in figure 165. Here, in fact we have to do again with a self-inverse class. But, corresponding to the circumstance, that the present class is equivalent in the extended sense to the third class belonging to $D=5$, the associated reproducing group does not let itself be extended by addition of substitution of the determinant i (cf. p.474).
%\begin{figure}
%\centering
%\includegraphics[width=0.7\linewidth]{}
%\end{figure}

The appended values of $\zeta$ and $\vartheta$ refer themselves to the case, that we position the net on the representing hemisphere of the form $(2,-1,0,-2)$. Thereby, as is \textcolor{red}{here} in the figure, two parabolic vertices.

The immediate arithmetic definition of our group, in the form belonging to $(2,-1,0,-2)$, one can so formulate, say in connection, with (8) on p.453, that it is a question of all substitutions of the unimodular Picard group, which satisfy the two relations:
\begin{align}
    2\alpha\bar{\alpha}-\alpha\bar{\gamma}-\gamma\bar{\alpha}-2\gamma\bar{\gamma}=2,\\
    2\alpha\bar{\beta}-\alpha\bar{\delta}-\gamma\bar{\beta}-2\gamma\bar{\delta}=-1.
\end{align}
But it does not appear, that one fashion the law thereby given, by further development, into an essentially more transparent form.