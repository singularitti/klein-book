% % % % % % % % % % % % % % % % % % % % % % % % % % % % % % % % % % %
% 请用XeTeX引擎编译, 编译之前请确保安装相应字体!
% 原文中出现的"", 全部使用English quotes(unicode), 如果看到?*?说明
% 原来是"*"没有替换过来. 如果需要其他格式的引号请自行替换
% 原文中矩阵内的元素以","分隔, 这里以空格代替
% 行内的分式一般采用\dfrac命令, 如果需要改成\frac, 请自行替换
% 由于手稿不清楚, 可能会发生把$\xi$堪称$\zeta$, 把$\alpha$堪称2(或反过来),
% 把$\cup$看成U, 把$\vartheta$看成$\nu$的情况, 请见谅
% % % % % % % % % % % % % % % % % % % % % % % % % % % % % % % % % % %
\documentclass[10pt,a4paper,leqno]{article}
\usepackage{amsmath}
\usepackage{amsfonts}
\usepackage{amssymb}
\usepackage{graphicx}
\usepackage{xcolor}
\usepackage{unicode-math} % 配合 \setmathfont 命令

\newcommand{\ii}{\ensuremath{\mathrm{i}}} % 虚数单位
\newcommand{\abs}[1]{\left|#1\right|} % 绝对值
\makeatletter
\newcommand{\Rmnum}[1]{\expandafter\@slowromancap\romannumeral #1@} % roman number
\makeatother

\begin{document}
\tableofcontents
\section{The Gaussian forms and the modular group}
We write the integral binary quadratic forms (by relation of the terminology utilized in "M" I):
\begin{align}\label{eq:11} %第一章第一个公式
ax^2+2bxy+cy^2
\end{align}
and use, as abbreviation for the individual such form, by symbol $(a,b,c)$ .The coefficients $(a,b,c)$ are rational intergral numbers, and $D=b^{2}-ac$ is the determinant of the form $(a,b,c)$. Let it be allowed, to designate the form \ref{eq:11}  as "\textit{Gaussian forms}"; for even if very valuable fundamental theorems concerning these were discarded by Lagrange, a complete theory of the forms in question is first contained in the \textcolor{red}{there}
The foundations of this theory now permit an especially brief and transparent representation, if one avails oneself of the geometric tools, which the modular group and the discussion of the $\zeta$-halfplane into circular-arc triangles associated to it given ready at hand. This geometric treatment of the Gaussian forms is given in "M" I on p.243.; and here are found a few, not unimportant expressions concerning the ambiguous forms as well as the sefl-inverse indefinite forms, in "M" II on p.161 ff.
In order to briefly remember to principal points of view of this subject, above all, the individual form(I) received a geometric interpretation.
For the definite forms, i.e., for those with $D<0$, one may restrict oneself to the so-called “positive” forms, for which $a$ and $c$ are positive numbers. A positive form, however, is interpreted geometrically by those points belonging to the positive $\zeta$-halfplane, which one obtains by solution of the quadratic equation:
\begin{align}\label{eq:12}
a\zeta^2+2b\zeta+c=0
\end{align}
one shows immediately, that by the giving of the representing point and the numerical solve $D$, the positive form $a,b,c$ is uniquely determined.
For an indefinite form, i.e., for one such with $D<0$, one also lets the restriction enter, that $D$ is different from a square. For the geometrical interpretation of the indefinite form, one marks the two points on the $\zeta$-axis, which now correspond to real and distinct roots of the equation(I). One then represents the form by that semicircle of the positive halfplane orthogonal to be the $\zeta$-axis, which has the two planes just marked as feet. The circle was also to be provided with a certain directional arrow, in order that the two forms $(a,b,c)$, $(-a,-b,-c)$, inverse to one another, could be separated from one another. The circle thus outfilled, in conjunction with the numerical solve $D$, furnished the form uniquely.
Now it was the theory of the equivalence and reduction of the Gaussian forms which, on the basis of the triangle division of the $\zeta$-halfpane, assumed a surveyable form by means of geometrical interpretation of the forms. For equivalent forms, the two representing points resp. semicircles are equivalent to the modular group of the “first” kind; a form is called reduced, in case its representing point belongs to the initial space (i.e., to the discontinuity domain continually formed in "M" I) of the group mentioned resp., its representing semicircle cuts through this domain. From this, there yielded itself, as reduction conditions for the definite positive forms:
\begin{align}\label{eq:13}
a\zeta^2+2b\zeta+c=0
\end{align}
with the condition, that for $a=c$, in more detail:
\begin{align}\label{eq:14}
b\geq 0
\end{align}
is to hold; for the reduced indefinite forms, besides this, there is the condition, that:
\begin{align}
a(a\pm b+c)<0
\end{align}
is to hold either for one or for both sighs. Yet it is to be noticed, that this latter inequality wears off from the Gaussian reduction condition for indefinite forms; we come back to this at the end of the chapter.
From the reduction conditions followed, moreover, by the arithmetic consideration, the finiteness od the number of reduced forms and with this, the finiteness of the class number for a given determinant $D$.
For the goal of our present representation, the theory of the indefinite forms is more important than that of the definite arcs. On the representing semicircle of an indefinite form, the double triangle of the modular division cuts off the infinitely termed chain of segments. We then obtain, in each case, a reduced form equivalent with the proposed form, if we transfer any one of those segments into the initial space and exercise thee corresponding transformation on the form. On account of the finiteness of the number or reduced forms, however, we obtain in this way any finitely many segments in the initial space(cf. “M" I on p.257 ff.); and from this the fact yielded itself, that an indefinite Gaussian form can always be transformed into itself by infinitely many substitutions. These substitutions form a cyclic hyperbolic group, which has the representing semicircle of the form as orbit curve.
This group could be extended inside the modular group in two particular cases, namely, first to a hyperbolic dihedral group of the first kind(cf. p346), in case $(a,b,c)$ is equivalent with its inverse form $(-a,-b,-c)$, secondly, to a(?) cyclic group of the second kind , in case $(a,b,c)$ is ambiguous (cf. "M" II p. 161)
The chain of the segments in the initial space, on which we laid at the representing semicircle of an indefinite form, furnished the associated “period of reduced forms”. These latter forms thereby appear arranged into a closed chain; and, in this chain, the transition from one form to a neighboring one, the so-called ?process of continuous reduction(?)?, effects itself in this specific situation by exercise of one of the two substitutions $\left(\begin{array}{cc}1 & 1\\0 & 1\end{array}\right)$ or $\left(\begin{array}{cc}0 & 1\\-1 & 0\end{array}\right)$, i.e., one of the two substitutions, which , for our discontinuity domain, furnish the generators or the modular group.
With this are given the principal points of view, according to which the geometrical treatment of the Gaussian forms was carried through in "M" I, we come back once more below to the ?projective form? of this theory, which one obtains, if one replace. The $\zeta$-halfplane by the interior of the ellipse of the hyperbolic plane. The conception hereby yielding itself attaches itself to the classical theory the Gaussian forms in many respects even directly. % 引入 sec1.tex
\section{Introduction of the Dirichlet and Hermitian quadratic forms}
A binary quadratic form named after Dirichlet was the same form as a Gaussian form:
\begin{align}\label{eq:21}
ax^2+abxy+cy^2
\end{align}
but here, the coefficients a,b,c are to be whole complex numbers of the form $(m+ni)$ with rational whole numbers $m,n$ and correspondingly, let $x$ and $y$ be indeterminate whole complex numbers of this kind.
There are those forms, whose theory Dirichlet was laid at in his substantial paper \textcolor{red}{here}\footnote{\textcolor{red}{here}}. Hereby, as was already emphasized above(p. 92), we took the \textcolor{red}{here} for a model are built up the theory of the forms(1) purely arithmetic.

The coefficients $a,b,c$ of the Dirichlet form belong to the domain of those whole complex numbers, which are introduced by Gauss for the purpose of the theory of biquadratic residues \footnote{\textcolor{red}{here}}.
In this domain, the calculation rules of the ordinary or rotational remain in force without restriction, concerning which one wants to compare the first part of the paper of Dirichlet's just named or also the eleventh supplementary in Dirichlet-Dedekind, \textcolor{red}{here} \footnote{\textcolor{red}{here}}.
According to the language of Dedekind's, all whole complex numbers $(m+ni)$, in conjunction with all quotients of such numbers, form a "number field" or briefly "field" of the second degree; we will denote this number field from now on symbolically by $\Omega^{(2)}$ or $\Omega$.
As abbreviation for the Dirichlet form \ref{eq:21}, we utilize the notation $(a,b,c)$; the determinant of the form (a,b,c) is $D=b^{2}-ac$.
Now transform the Dirichlet form by means of the substitution:
\begin{align}\label{eq:22}
x=\alpha x' + \beta y' , y=\gamma x' + \delta y'
\end{align}
when $\alpha, \beta, \gamma, \delta$ are any fair whole numbers from $\Omega$ with a non-vanishing determinant. As result, there again arises a Dirichlet form $(a',b',c')$, whose coefficients are:
\begin{align} \label{eq:23}
\left\{\begin{array}{rcl}
a'&=& \alpha^2 a+2\alpha\gamma b+\gamma^2 c,\\
b'&=&\alpha\beta a+(\alpha\delta+\beta\gamma)b+\gamma\delta c,\\
c'&=&=\beta^2 a+2\beta\delta b+\delta^2 c,
\end{array}\right.
\end{align}
and whole determinant $D'$ calculates itself as:
\begin{align}\label{eq:24}
D'=(\alpha\delta-\beta\gamma)^2 D
\end{align}
the transformed form $(a',b',c')$ to be obtained thus is called "contained under the form $(a,b,c)$".
Of the two forms $(a,b,c)$ and $(a',b',c')$, should each be contained under the other, then each of the two  numbers D and D' must go into the other. Therefore, $(\delta\delta-\beta\gamma)$ is such a whole number from $\Omega$, whose reciprocal value is likewise a whole number of this number field. According to this, $(\delta\delta-\beta\gamma)$ must be equal to $\pm 1$ or $\pm i$ \footnote{\textcolor{red}{here}}. One now calls the two forms $(a,b,c)$ and $(a',b',c')$ properly or improperly equivalent, if the determinant $(\delta\delta-\beta\gamma)$ is equal $+1$ or $-1$\footnote{\textcolor{red}{here}}. From equation \ref{eq:24}, it then yields itself, that equivalent forms always have the same determinant; on the other hand, for transformation by as substitution of the determinant $\pm i$, a sign change of $D$ enters. However, we will agree, thaty by equivalence is always to be meant from now on simply the "proper" equivalent.
The substitutions, which effect the equivalence of the Dirichlet forms, in case we apply the non-homogeneous terminology, now lead immediately to the Picard group investigated in detail above (p. 76 ff). But we also arrive at just this group for the Hermitian forms, to which, first of all, use \textcolor{red}{new} return.
With inessential deviation from the original Hermitian notation\footnote{Compare, for this notation, the original paper, named on p. 92, of Hermite's in Crelle's Journal Bd. 47, p.343 ff.}, we had already(p.92) put the so-called Hermitian forms into the form:
\begin{align} \label{eq:25}
D=b\bar{b}-ac=b_{1}^{2}+b_{2}^2-ac
\end{align}
Here, for the individual form, $a$ and $c$ are to be real whole number, but $b$ and $\bar{b}$ conjugate complex numbers from $\Omega$; correspondingly $x$ and $\bar{x}$, as well as $y$ and $\bar{y}$, are to represent conjugate complex, but not more closely determined whole numbers of $\Omega$. A Hermitian form is thus conjugate with itself.
The whole number $b$ we write explicitly $b=b_{1} +i b_{2}$and utilize the schema $(a,b_1,b_2,c)$ of the far rational whole numbers $a,b_1,b_2,c$ as an abbreviation for the form \ref{eq:25}.
The determinant of the \ref{eq:25} is
\begin{align} \label{eq:26}
D=b\bar{b}-ac=b_{1}^{2}+b_{2}^2-ac
\end{align}
As one sees the determinant of a Hermitian form is always real, in contradistinction to that of a Dirichlet form, which is any complex whole number of the number filed $\Omega$. This situation causes us, (as for the Gaussian forms) to distinguish between definite and indefinite Hermitian forms, according as the determinant 1) is negative or positive. The significance of this core distinction we investigate further below.
Now transform the form $(a,b_1,b_2,c)$ by means of the simultaneous substitutions:
\begin{equation}
\begin{split} \label{eq:27}
x &= \alpha x' + \beta y' , y= \gamma x' + \delta y'\\
\bar{x} &= \bar{\alpha}    \bar{x}' + \bar{\beta}, \bar{y}=\bar{\gamma}\bar{x}'+\bar{\delta}\bar{y}'
\end{split}
\end{equation}
where $\alpha,\beta,\gamma, \delta$ are whole number from $\Omega$ of non-vanishing determinant, and $bar{\alpha}$ is conjugate to $\alpha$, $bar{\beta}$ is conjugate to $\beta$,,?. The result is again a Hermitian form $(a',b_{1}',b_{2}',c')$ with the coefficients:
\begin{align} \label{eq:28}
\left\{\begin{array}{rcl}
a' &= a\alpha\bar{\alpha} + b\alpha\bar{\gamma} + \bar{b}\gamma\bar{\alpha} + c\gamma\bar{\gamma}\\
b' &= a\alpha\bar{\beta}  + b\alpha\bar{\delta} + \bar{b}\gamma\bar{\gamma} + c\gamma\bar{\delta}\\
\bar{b}' &= a\beta\bar{\alpha}  + b\beta\bar{\gamma} + \bar{b}\delta\bar{\alpha} + c\delta\bar{\gamma}\\
c' &= a\beta\bar{\beta}  + b\delta\bar{\beta} + \bar{b}\delta\bar{\gamma} + c\delta\bar{\delta}
\end{array}\right.
\end{align}
and of the following determinant:
\begin{align} \label{eq:29}
D' = (\alpha\delta-\beta\gamma)(\bar{\alpha}\bar{\gamma}-\bar{\beta}\bar{\gamma})D
\end{align}
The transformed form $(a',b_{1}',b_{2}',c')$ thus obtained is called ?contained under the form $(a,b_1,b_2,c)$?.
Here 1) and 1)' always have the same sigh, so that a definite(indefinite) form always can \textcolor{red}{here} be contained under a definite(indefinite) \textcolor{red}{here} again.
If, of the two forms $(a,b_{1},b_{2},c)$ and $(a',b_1',b_2',c')$, each is to be contained under the other, then for this it is necessary and sufficient, that $\alpha\delta-\beta\gamma$ is either equal to $\pm 1$ or $\pm i$. Thereby, there is a distinction between the Dirichlet forms, insofar as the determinant $D'=D$ for $\alpha\delta-\beta\gamma= \pm i$ as well as for $\alpha\delta-\beta\gamma = 1$, in consequence of \ref{eq:29}. However, one also speaks \textcolor{red}{here} of proper resp. improper equivalence only then, if $\alpha\delta-\beta\gamma = 1$ resp. $=-1$. If $\alpha\delta-\beta\gamma = \pm i$, then let the forms be called "equivalent in the extended sense"\footnote{Hermite distinguishes l.c. on p.350 \textcolor{red}{here} orders of improper equivalence, corresponding to the cases $\alpha\delta-\beta\gamma = -1, +i, -i$}.
As one sees, we are also led to the Picard group here again. We shall consequently make corresponding use of the latter group in the treatment of the fundamental problems at the theories of the Dirichlet and Hermitian forms, as of the modular group for the Gaussian forms. The problems to be treated, however, are the following: if two of the same determinant are given, then it is to be decided, whether they are equivalent or not; and in the former case, all substitutions are to be given, which transform the one form into the other. It will be possible, in the treatment of these tasks, to embark on a lone of development, which finds its model in the investigation of the Gaussian forms on the basis of the modular group\footnote{See, moreover, for the historical development of the theory of the Dirichlet and Hermitian forms, the references given on p.92. ff.}.
\section{Geometrical Interpretation of the Dircichlet and Hermitian forms}

The possibility of \textcolor{red}{here} the Picard group and the polyhedral division belonging to it for the theory of the repeatedly mentioned forms, rests on a geometrical interpretation of these forms, which ties itself in directly to the interpretation of the Gaussian forms in the $\zeta$-halfplane(cf. p443). \textcolor{red}{here}\textcolor{red}{here} quotient \textcolor{red}{here} of the \textcolor{red}{here} indeterminate quantities x:y, denote this by $\zeta$ and bring in the $\zeta$-plane resp. the positive $\zeta$-halfplane(p.53).
If now, first of all, we have a Hermitian form (\textcolor{red}{here}), then by setting it equal to zerok with utilization of $\zeta$, one finds the equation:
\textcolor{red}{here}
or (if we set $\zeta$ = \textcolor{red}{here} as earlier):
(1)\textcolor{red}{here}
if, in particular, \textcolor{red}{here} is a definite Hermitian form, i.e, one such with \textcolor{red}{here}, then equation (1) represents an “imaginary circle” of the $\zeta$-plane. Since the form \textcolor{red}{here} process from the left side of (1) by multiplication by the non-negative number \textcolor{red}{here}, it follows, that a definite form \textcolor{red}{here}, for any  numbers \textcolor{red}{here} of \textcolor{red}{here}, is able to represent either only positive or negative real numbers. On this account, one seperates the forms in question into “positive” and “negative” forms and, moreover, notices immediately, that for a positive(negative) form the cofficients a and c are \textcolor{red}{here}. \textcolor{red}{here}, positive forns are able to be equivalent again only with positive ones, and negative ones with negative ones, and one easily writes from (y) on p.453.

We are obviously able to restrict \textcolor{red}{here} to the conversation of the positive forms, for which, now, a real geometrical interpretation is to be obtained in the $\zeta$-halfplane. For this purpose, we form, in the $\zeta$-plane, the pencil of all “spheres through the imaginary circle(1)”. If \textcolor{red}{here} is a real parameter, then this pencial represents itself by :
(2)
where \textcolor{red}{here} are the coordinates of the $\zeta$-plane already used above(p.54). The pencil(2) now contains two real boundary points, i.e., spheres infinitely small radius; here are the points with the coordinates:
(3)
that one of these two points, which belongs to the positive halfplane, we let \textcolor{red}{here} utilized for the geometric interpretation of the indefinite form \textcolor{red}{here}.
In order, first of all, to dispose of the Dirichlet forms at the same time, then to the form \textcolor{red}{here} corresponds the equation:
(4)
where roots are the following:
(5)

The square root of the discriminant is here to be so defined, that \textcolor{red}{here} is either to have a positive real component, or that, in case that vanishes, the imaginary component carries the positive sign. On the strength of this, call $\zeta$, the fuirstm \textcolor{red}{here} The second rot of equation(4). We then interpret the Dirichlet form by that semicircle belonging to the $\zeta$-halfpane, which stands particular to the $\zeta$-plane at the points \textcolor{red}{here} and \textcolor{red}{here} and is provided with an arrow pointing from \textcolor{red}{here} to \textcolor{red}{here}.

In the case of an indefinite Hermitian form, equation (1) represents a real circle of the $\zeta$-plane, corresponding to the name “indefinite”, the form has positive numerical values on the one side of this circle, and negative ones on the others; and we distinguish, in this case, on the periphery of the circle, a “positive” and “negative” \textcolor{red}{here}(bank?). Run through the periphery in such a direction, that the positive \textcolor{red}{here} is to beleft, and provide the circle with an arrow indicating the sense of this circuit. Finally, we erect in the $\zeta$-halfpane as hemisphere, which rises up or \textcolor{red}{here} top the $\zeta$-plane at the circle in question. This hemisphere, whose boundary(1) belonging to the $\zeta$-plane is provided with an arrow in the manner \textcolor{red}{here}, let be the geometrical image of the indefinite form \textcolor{red}{here}

In all three cases, now, the theorem holds, that the individual form is uniquely defined by the determinant D and the representing geometric image. The representing image, namely, furnishes, first of all, the quotients of the coefficients a,b,c. The addition of D permits the determination of the coefficients up to a common sign change, i.e, \textcolor{red}{here}, in each case, two forms “inverse: to one another \textcolor{red}{here},\textcolor{red}{here} resp.

 \textcolor{red}{here}\textcolor{red}{here}. Finally, for a definite Hermitian form, the sign is to be so \textcolor{red}{here}, that the form is positive; in the two other cases, however, the direction of the \textcolor{red}{here} of the representing image be \textcolor{red}{here} concerning the sign.

Further, the following important theorem holds: If, from any proposed form, one goes over by exercise of a \textcolor{red}{here} substitution to an equivalent form, then by the correspoinding \textcolor{red}{here} $\zeta$-substitutions, the representing model(point, semicircle, \textcolor{red}{here}) of the first form is transformed into that of the second. This theorem is an immediate result of the method followed in the geometrical interpretation of the forms, which to me individual form assigned its geometrical \textcolor{red}{here} in a manner invariant with respect to $\zeta$-substitutions, One now only still wants to convince oneself, that for a semicircle resp. hemisphere, the transformation by a unimodular $\zeta$-substitution furnishes the appropriate direction of the arrow of the \textcolor{red}{here} model. This yields itself from the circumstance, that in the fixing of the direction of the arrow, we proceeded according to the method, which is incariant with respect to the unimodular $\zeta$-subsitutions of the first kind(notations of the $\zeta$-halfsoace resp. the $\zeta$-plane). One also equally verifies, however, e.g., for the Dirichlet forms, by direct calculation and utilization of (3) on p.451, that the “first” root $\zeta$, of (a,b,c), by exercise of a unimodular $\zeta$-substitution of \textcolor{red}{here}, goes over into \textcolor{red}{here}, i.e, into the first root of the transformed form.



\section{Treatment of the equivalence problem for the definite Hermitian forms}

The “unimodular Picard group”, entering for the (proper) equivalence of the Hermitian and Dirichlet forms, is more closely investigated on p.35 ff. as a group $\Gamma_2$. There yielded itself, as “initial space” of this group, a double pentahedron of the $\zeta$-halfspace, which was defined by the conditions:
\begin{align}\label{eq:41}
	-\dfrac{1}{2}\leq \zeta <\frac{1}{2}, 0\leq\eta\leq\frac{1}{2}, \zeta^2+\eta^2+\nu^2\geq 1,
\end{align}
which the addition, that in the last two formulas, the equality signs are only then to hold, if $\zeta\leq 0$.

On the double pentahedron, we now establish the treatment of the equivalence problem of our forms.
First, if we were a positive Hermitian form, then the following definition is to be given a preeminient position: The positive Hermitian form $(a,b_1,b_2,c)$ is to be called “reduced”, if its representing point lies in the initial space at the unimodular Picard group. By substituting the coordinates(3) on p.455 of the representing point into \ref{eq:41}, here followes: The positive form $(a,b_1,b_2,c)$ is always and only then a reduced form, if the conditions hold:
\begin{align}\label{eq:42}
	-a\leq -ab_1\geq a, 0\leq 2b_2\leq a, a\leq c,
\end{align}
with the addition, that in the second and third formulas, equalitu signs only hold then, in case $b_1\geq 0$.

From the concept of the discontinuity domain, there yields itself: Each positive Hermitian form of two proposed forms, go, in both cases, to the reduced form (say with use of the pentahedral division). The latter must identical, if those forms are to be equivalent.

The request for all the substitutions, which effect the equivalence in the individual case, one reduces, in known fashion, to the task of ascertaining all substitutions, which transform a proposed form into itself. We answer this request, say for the reduced forms, in the following way: A reduced form $(a,b_1,b_2,c)$ will be transformed into itself by none of \textcolor{red}{here} $\zeta$-substitutions apart from the identity, incase its representing point is not situabled right on an edge of the elementary pentrohedron. On the other hand, one has two resp. three transformations, forming a cyclic group, of the form into itself, in case the representing point belongs to an edge, but to no vertex of the elementary pentahedron; and finally, we have for, six or twelve formations, forming a \textcolor{red}{here} resp. dihedral or tetrahedral group, in case the representing point is a vertex of the elementary pentahedron. All these remarks follow immediately from the character of the pentahedral division of the $\zeta$-halfspace belonging to the Picard group.

In order, after this, e.g., for the edge of the elementary pentahedron given by:
\begin{align}
	\zeta=-\frac{1}{2}, \eta=\frac{1}{2}, \nu>\frac{1}{\sqrt{2}},
\end{align}
to characterize all associated reduced forms, we have to set:
\begin{align}
	\frac{b_1}{a}=\frac{1}{2}, \frac{b_2}{a}=\frac{1}{2}, \frac{\sqrt{-D}}{a}>\frac{1}{\sqrt{2}}.
\end{align}
All the positive forms corresponding to this condition, we are able to collect together into the form
\begin{align}
	\alpha m x \bar{x}+m(1+\ii)x\bar{y}+m(1-\ii)\bar{x}y+(2m+n)y\bar{y},
\end{align}
where $m$ and $n$ are only two rational whole positive numbers. --- Naturally, to each edge of the elementary pentahedron belong infinitely many form classes; in contrast to this, the individual pentahedral vertex always yields only one class, as long as we restrict ourselves to “primitive” forms, i.e, to such, for which $(a,b_1,b_2,c)$, have no common factor \textcolor{red}{here} $\mathbf{1}$. The latter vertex or the vertex just discussed, e.g., furnishes the class of the determinant $D=-2$ belonging to the reduced form:
\begin{align}
	2\alpha\bar{\alpha}+(1+\ii)x\bar{y}+(1-\ii)\bar{x}y+2y\bar{y}.
\end{align}

For a given value of the determinant $D$, the \textcolor{red}{here} of the number of reduced forms and, with this, the finiteness of the class number of positive Hermitian forms is a simple consequence of the complete reduction conditions \ref{eq:42}. Namely, from these conditions, one easily derives the inequality:
\begin{align}
	a\leq \sqrt{-2D},
\end{align}
So that $a$, and with this, also $b_1$, and $b_2$, for given $D$, are restricted to a finite number of values. With $a, b, D$, however, $c$ is uniquely determined, After this, no difficulty stands in the way of the enumeration of all the reduced forms in the individual case $D$. Thus, e.g, for $D=-5$, one finds three closses with the reduced forms:
\begin{align}
	(1,0,0,5),(2,0,1,3),(2,1,0,3).
\end{align}
The representing point of the last form proceeds from the second one by the substitution $\zeta'=\ii \zeta$; the two forms, therefore, according to the language agreed upon on p.453, are equivalent in the extended sense.


\section{Reduction theory of the Dirichlet forms}

If the determinant $D$ of a Dirichlet form $(a,b,c)$ is a perfect square inside the number field $\Omega$, then the roots $\zeta_1, \zeta_2$(cf p.455) belonging to $(a,b,c)$ are likewise numbers of $\Omega$ and accordingly furnish two points of the $\zeta$-plane, which each cusps of $\infty^2$ pentahedra of the halfspace division. The representing semicircle of the form $(a,b,c)$, under these circumstances, runs through only finitely many pentahedra, The treatment of these forms on the basis of the Picard group, which offers no difficulty, was no interest for us, since here, relations to the subgroups of the Picard group do not appear. We accordingly exclude, from now on, the forms with quadratic determinants.\footnote{*}

The equivalence theory again bses itself on the concept of the reduced forms: The Dirichlet form $(a,b,c)$ is to be called “reduced”, in case its representing semicircle was a segment of non-vanishing arc-length in common with the initial space of the Picard group characterized by (1) on p.457.

Since $D=0$, as a square is excluded, the representing semicircle of each form $(a,b,c)$ pushes into the $\zeta$-halfspacehalfspace. It follows immediately from this, that each Dirichlet form $(a,b,c)$ is equivalent with a reduced form. However, in order to arrive at the complete solution of the equivalence problem, we must first develop the theory of the reduced forms even further.

It is now very remarkable, that the presentation of the reduction conditions in arithmetic form leads to quite unsurveyable formulas, which we shall consequently not set up completely here.

Dirichlet treats the reduction theory in $\S$16 of his repeatedly named work and gives the following reduction conditions:
\begin{align}
	\abs{b}\sqrt{2}\leq \abs{a}\leq\abs{c},
\end{align}
understanding by $\abs{a},\ldots$ The absolute value of $a, \ldots$. These inequalities, \textcolor{red}{here} after the reduction conditions of the definite Gaussian forms (cf.(3) p.449), however, do not express the reduction condition just given in geometrical form. Consider, e.g., that the center of the representing semicircle of the form $(a,b,c)$, in consequence of (5) on p.455, is situated at $\zeta=-\frac{b}{a}$. This center, according to (1), would have at most the distance $\dfrac{1}{\sqrt{2}}$ from the origin, which by no means holds for the forms reduced in any sense.

Furthermore, the theory of Dirichlet’s beaed on the condition(1), at this points, \textcolor{red}{here} considerably behind the treatment of the Gaussian forms in the “Disquistions Arithmetic\textcolor{red}{?}”. The doctrine of the “periods of reduced indefinite forms”, which Gauss develops l.c.Art.136, immediately lets itself be correspondingly \textcolor{red}{here} at for the Dirichlet forms, as will emerge from the reduction condition \textcolor{red}{here} by us. It appears, that Dirichlet has not noticed this possibility.

Also in the paper of \textcolor{red}{here}’s normed on p.93, to which the representation here essentially attraches, the setting up of the “complete” reduction conditions in arithmetic form is ignored.

In order to see the finiteness of the number of reduced forms for given $D$, are furthermore need not know the complete arithmetic reduction conditions. For this, the two conditions:
\begin{align}
	\abs{a}<\sqrt{2\abs{D}},\abs{b}<\frac{\abs{a}}{\sqrt{2}}+\sqrt{\abs{D}}
\end{align}
for a reduced form $(a,b,c)$, indeed necessary, but not yer sufficient, already suffice. The correctness of this proceeds from the circumstance, that the representing semicircle of $(a,b,c)$ was not the center $\zeta=-\dfrac{b}{a}$ and radius $\dfrac{\sqrt{\abs{D}}}{\abs{a}}$. The vertex of the double pentahedron (1) on p.451 situabled closest to the $\zeta$-plane has the coordinate $\nu=\dfrac{1}{\sqrt{2}}$; the first condition (2) consequently brings to expression, that the radius $\dfrac{\sqrt{\abs{D}}}{\abs{a}}$ for a reduced form must be greater than this $\nu$. The projection of the double pentahedron (1) on p.457 into the $\zeta$-plane lies inside a circle with the radius $\dfrac{1}{\sqrt{2}}$ about $\zeta=0$. The center of the representing semicircle of a reduced form may consequently not attain the disturbance $\dfrac{1}{\sqrt{2}}+\dfrac{\abs{D}}{\abs{a}}$, since this semicircle could otherwise have no point in common with double pentahedron. Form this requirement, one will immediately hand off the second inequality(2). Notice moreover, that on account of the exclusion of purely quadratic $D$, the number always has a value different from \textcolor{red}{$\mathbf{Q}$}.

Since the number field $\Omega$ only furnishes a bonded number of whole numbers, whose absolute values do not exceed a fixed finite bond, \textcolor{red}{here} in consequence of (2), for given $D$, first $a$, then also $b$, and with this, $c$ too, are restricted to a finite number of values: For given determinant $D$, there is only a finite number of reduced forms and, with this, only a finite number of form classes.

The next development now ties itself in precisely with Stephen Smith’s theory of the indefinite Gaussian forms laid at in “M” I on p.250ff. Since the endpoints $\zeta_1, \zeta_2$ of the representing semicircle of a Dirichlet form $(a,b,c)$ of non-quadratic determinant are not parabolic points, then this semicircle runs through infinitely many double pentahedra toward $\zeta_1$, as well as $\zeta_2$, and appears cut up in such a way \textcolor{red}{here} a chain of infinitely many segments. The transformation of \textcolor{red}{here} are of these segments into the initial space furnishes a substitution, which carries $(a,b,c)$ over into a reduced form.

If we transfer all segments into the initial space, then we are also able here to hold first to their original equivalence. Namely, if we pursue the individual segment, as the semicircle of a reduced form, from the initial space into a neighingboring double pentahedron, then the segment of this circle lying in the latter is thrown back into the initial space by a determinate one of the fine substitutions known here from p.87:
\begin{align}
    S=\left(\begin{array}{cc}
    \ii & 0\\
    0 & -\ii\\
    \end{array}\right), 
    T^{\pm 1}=\left(\begin{array}{cc}
    1 & \mp 1\\
    0 & 1\\
    \end{array}\right), 
    U=\left(\begin{array}{cc}
    0 & -1\\
    1 & 0\\
    \end{array}\right), 
    V=\left(\begin{array}{cc}
    \ii & 1\\
    0 & -\ii\\
    \end{array}\right),
\end{align}
and furnishes here the immediately following term in the chain of the segments running through the initial space. Foe the forms, this signifies, that the substitution (3) coming into application transforms the reduced form belonging to the first segment into a “neighboring” similarly reduced form. By the algorithm thus obtained, one obviously obtains all the reduced forms equivalent with the proposed form and finds for the latter, at the same time, a determinate sequence. We designate this algorithm as the “process of continuous reduction”.

The previous consideration, moreover, requires an exposition, in case the representing semicircle passes through edges or vertices of the double pentahedra at our division of the $\zeta$-halfspace. In this case, among the segments in the initial space such will occur, which end in edges resp. Vertices of this domain, and whose associated semicircles next pass over into double pentahedra, which are connected with the initial space only in edges resp. vertices. In order to continue the chain of reduced forms beyond such a place, one obviously cannot apply one of the substitutions(3); rather, one at twelve particular additional substitutions is now to be exercised, which one will easily define from the eight types and the four non-parabolic vertices at the initial space.

We now join the result obtained with the finiteness of the number of all reduced forms of gived determinant $D$ already verified. It yields itself, exactly as for the indefinite Gaussian forms in “M” I on p.260, that in the chain of the reduced forms of non-quadratic determinant $D$ belongs a “finite-termed” so called “period of reduced forms”; and one can generate the entire period by means of the process of continuous reduction from an individual form of the period.

It will now scarcely still be necessary, to say, now, on the strength of this, we shall decide concerning the equivalence of two proposed forms $(a,b,c)$ and $(a',b',c')$. One must produce the periods of reduced forms belonging to the two forms. These form periods must be identical, in case we are to have equivalence of $(a,b,c)$ and $(a',b',c')$.

As an example, we first consider the determinant $D=1+\ii$, which represents a prime number of the field $\Omega$. Here $\abs{a}<\sqrt{s\sqrt{2}}$, so that, for a, only the eight values, $\pm 1, \pm\ii, \pm(1+\ii), \pm(1-\ii)$, are admissible, of we would otherwise have to do with a reduced form. If now, $a=\pm 1$ or $\pm\ii$, then the second reduction condition (2) on p.460 yields, that b can have the nine values $0, \pm 1, \pm\ii, \pm(1+\ii), \pm(1-\ii)$. For $a=\pm(1\pm\ii)$, the values $b=\pm 2$ and $\pm 2\ii$ are also admissible; however, on account of $b^2-ac=1+\ii$, b must now be divisible by $(1+\ii)$, so that the values $b=\pm 1, \pm\ii$ are to be excluded. All together, one thus obtains a system of 72 forms, pairwise inverse to one another.

Of these 72 forms, however, only eight show themselves to be reduced, which latter form a single self-inverse form period. The eight forms of the period are, in the correct order, the following:
\begin{align}
    (1,0,-1-\ii),(1,1,-\ii),(-\ii,-1,1),(\ii,0,-1+\ii),\\
    (-\ii,0,1-\ii),(i,1,-1),(-1,-1,\ii),(-1,0,1+\ii)
\end{align}
the substitutions to be exercised here are, in order, $T^{-1}, U, V, S, V, U, T$.

As one sees, the form period consists here of two symmetric values, expect that each two forms of the period symmetric to one another furnish, not identical, but inverse forms. One recognizes immediately, \textcolor{red}{here} this always and only then occurs, if the representing semicircle meets one, and with this, infinitely many, edges of the pentahedral division perpendicularly, which belongs to elliptic substitutions of the period 2; we come back to this in the next paragraph.

By utilization of this circumstance, moreover, one can quite drastically shorten the setting up of the form periods for latter valued $D$. If, e.g., we choose $D=2+\ii$, then the first condition (2) on p.460 yields the inequality $\abs{a}<\sqrt{2\sqrt{5}}$, so that only the values $0, \pm 1, \pm\ii, \pm(1+\ii), \pm(1-\ii), \pm 2, \pm 2\ii$ are admissible for $a$. Now the center of the individual representing semicircle lies at $\zeta=-\dfrac{b}{a}$; and since the values of a just given are all divisors of 2, then it I exlusively a matter of points, at which rectilinear edges of the pentahedral division should perpendicular to the $\zeta$-plane (cf.Fig.19 on p.81). The individual semicircle consequently cuts the associated edge perpendicularly.

Further, one can now so transport the representing semicircle in question by exercise of a substitution $\zeta'=\pm\zeta+m+n\ii$ , that the rectilinear edge of the pentahedral division intersected perpendicularly by it becomes one of the six edges of this kind belonging to the initial space. Since the coefficient a herby experiences at most assign change, the form remains reduced, as easily seen. In such a way, one has \textcolor{red}{here} to one ore the twelve following terms:
\begin{align}
    \pm(1,0,-2-\ii),\pm(\ii,0,-1+2\ii),\pm(1+\ii,1,-1),\\
    \pm(1-\ii,-\ii,-1+2\ii),\pm(1-\ii,1,-\ii),\pm(1+\ii,\ii,-2+\ii),
\end{align}
as one easily determines by discussion of the six edges of the initial space in question.
The form $(1+\ii,1,-1)$ goes over into the form $-(1+\ii,\ii,-2+\ii)$ by the substitution $\left(\begin{array}{cc}
\ii & \ii\\
0 & -\ii\\
\end{array}\right)$, and one correspondingly recognizes the equivalence of the forth and fifth forms of the system just given. Since here, however, each form is equivalent with its inverse, one needs only yet to exercise the “algorithm of continuous reduction” individually to the four forms:
\begin{align}
    (1,0,-2-\ii),(\ii,0,-1+2\ii),(1+\ii,1,-1),(1-\ii,1,-\ii).
\end{align}

The effectuation at the calculation furnishers two form periods “equivalent in the extended sense”(cf, p.453), each six termed. It will suffice to present one of these periods:
\begin{align}
    (1,0,-2-\ii),(1,1,-1-\ii),(-1-\ii,-1,1),\\
    (1+\ii,1,-1),(-1,-1,1+\ii),(-1,0,2+\ii).
\end{align}
The substitutions coming into application here are
\begin{align}
    T^{-1},U,\begin{array}{cc}
    \ii & 1+\ii\\
    0 & -\ii\\
    \end{array}, U, T,
\end{align}
in the Third place, we therefore have an example of those twelve substitutions not belonging to the system (3), which are also to be applied in the continuous of the present case, according to p.462(above).
\section{The Transformation of the Dirichlet forms and themselves}

The second problem of the theory of equivalence, which amounts to giving all substitutions which transform a proposed Dirichlet form $(a,b,c)$ into itself, is also immediately solvable on the basis of the process of “continuous reduction”.

A substitution of the unimodular Picard group, which carries $(a,b,c)$ over into itself, will transform the representing semicircle into itself with maintenance of the direction of the arrow. By pushing aside the particular case, that the semicircle is on elliptic axis; it can only be a question of loxodromic or hyperbolic substitutions, which have the feet $\zeta_1, \zeta_2$ or the semicircle as fixpoints.

But now, the semicircle consists of a chain of segments which we build up from infinitely many equivalent finite-termed periods. By reserving particular consideration of the elliptic axes, we thus have the following result: A Dirichlet form $(a,b,c)$ of non-quadratic D always and only lets itself be transformed into itself by the substitutions of a cyclic loxodromic or hyperbolic subgroup, which has the semicircle of the form as orbit curve.

This theorem, however, is also immediately reversible: to each cyclic loxodromic or hyperbolic subgroup belongs a Dirichlet form of non-quadratic determinant. With the treatment of the Dirichlet forms, therefore, the theory of the cyclic subgroups in question is disposed of at the same time.

The process of continuous reduction now also furnishes the means of calculating the generating substitution of the cyclic subgroup belonging to the form $(a,b,c)$. To this end, one has only to combine these substitutions, which came into application in setting up the individual terms of the \textcolor{red}{here} period. Thus, e.g., the reduced form $(1,0,-1-\ii)$ of the determinant $(1+\ii)$, considered in the previous paragragh, is transformed into itself by the substitution:
\begin{align}\label{eq:61}
    T^{-1}UVSVUTS=\left(\begin{array}{cc}
    1-2\ii & 2-2\ii\\
    -2\ii & 1-2\ii\\
    \end{array}\right),
\end{align}
as generator of the associated cyclic group; and in the same sense, to the form $(1,0,-20\ii)$ belongs the substitution:
\begin{align}\label{eq:62}
    T^{-1}U\cdot\left(\begin{array}{cc}
    \ii & 1+\ii\\
    0 & -\ii\\
    \end{array}\right)\cdot UTS=\left(\begin{array}{cc}
    2-\ii & 3-\ii\\
    1-\ii & 2-\ii\\
    \end{array}\right).
\end{align}

Only notice, that the last substitution S, which in each case engineers the transition to the first term of the following period, was not given above.

Furthermore, the substitutions, which carry a Dirichlet form $(a,b,c)$ over into itself, can also be produced in an arithmetic way, essentially as this was done for the Gaussian forms in “M” I on p.252ff. The substitutions are then to be put into the form:
\begin{align}
    \left(\begin{array}{cc}
    \frac{t-bu}{\sigma} & -\frac{cu}{\sigma}\\
    \frac{au}{\sigma} & \frac{t+bu}{\sigma}\\
    \end{array}\right).
\end{align}
Thereby, $\sigma$ denotes the “divisor” of the form $(a,b,c)$, i.e,m the greatest common divisor of $a, 2b, c$, which, e.g, for “primitive” forms is either 1 or $(1+\ii)$ or finally 2. For $t$ and $u$, however, are to be substituted all solutions of the generalized “\textcolor{red}{here} equation”:
\begin{align}\label{eq:64}
    t^2-Du^2=\sigma^2
\end{align}
in whole complex numbers of the field $\Omega$. From the previous theory follows immediately, that the Pellian equation \ref{eq:64}, for non-quadratic $D$, is always solvable by $\infty'$ pairs of while complex numbers $t, u$, and that all these solutions let themselves be calculated from the “smallest” one of them according the logarithm \textcolor{red}{here} here from the Gaussian forms. Thus, e.g., one finds from \ref{eq:61} and \ref{eq:62},
\begin{align}
\begin{array}{ccc}
    \text{for }D=1+\ii & t=1-2\ii &u=2\ii\\
    \text{for }D=2+\ii & t=2-\ii & u=1-\ii\\
\end{array}
\end{align}
in this specific situation, therefore, “smallest” solutions. The small have to make an important application in the next \textcolor{red}{here} of the solvability of the Pellian equation thus obtained\footnote{*}.

Now, those forms are to be investigated, for which the representing semicircle is an elliptic edge of the pentahedral(?) division. Of the eight edges to be counted as part of the double pentahedral (1) on p.457, however, six \textcolor{red}{here} upon parabolic points in the $\zeta$-plane, and thus do not come into consideration for us. The two remaining edges, on the other hand, have the feet:
\begin{align}
    \zeta_1,\zeta_2=\frac{-1\pm\ii\sqrt{3}}{2} \text{ and } \zeta_1,\zeta_2=\frac{\pm\sqrt{3}+\ii}{2}.
\end{align}
Here, therefore, in case we restrict ourselves to primitive forms, we are led to the two forms $(2,1,2)$ and $(2,-\ii,2)$ belonging to $D=-3$ resp. $D=+3$. These forms ( and \textcolor{red}{here} those equivalent with them) are the only Dirichlet forms of non-quadratic determinant, which, besides the substitutions of each of the cyclic groups already obtained above as associated, we also transformed into themselves by elliptic substitutions, Hereby, it is a question , in the individual case, of a cyclic group $G_3$ of order three. This $G_3$ unite itself with the cyclic group $G_\infty$, if we retain the terminology used on p.234; the latter group then furnishes all the substitutions of the present form into itself.——

Finally, a few more particular species of forms of interest are to be mentioned.

First of all, we consider the case, that the representing semicircle of $(a,b,c)$ cuts an elliptic edge, of the pentahedral division, belonging to the period to the period two, perpendicularly. The individual pentahedron has six edges of this kind; they are those edges which end in parabolic points. By exercise of the associated elliptic substitution, the representing semicircle is carried over into itself with reversal of the direction of its arrow. One recognizes immediately, in the strength of this, that we are concerned here with all those form classes, which are self-inverse. The representing semicircle carries infinitely many elliptic fixpoints here, among which every two following are another \textcolor{red}{here} “half-period”. All the substitutions of the Picard group, which transform the present form $(a,b,c)$ into itself or $(-a,-b,-c)$, form a group, which one has to design ate as a “hyperbolic” or “loxodromic dihedral group”(cf. p.346). For all non-quadratic determinant D, we get form classes at this kind; e.g., the “principal classes” furnished by $(1,0,-D)$ always belong here, Compare, moreover, the examples of the previous paragraph.

Secondly, we must still consider the case, that the representing semicircle of $(a,b,c)$ is either \textcolor{red}{here} in a symmetry semisphere orthogonally. Here we are led to the concept of the ambiguous form, and thus obtains two different species of ambiguous forms $(a,b,c)$. The cyclic group belonging to the individual ambiguous form is hyperbolic and, inside the Picard group of the second kind, is itself capable of extension by reflections. The indefinite Gaussian forms furnish examples of the first species of ambiguous Dirichlet forms, and the definite ones the second species.

\section{Reduction theory of the indefinite Hermitian forms}

We interpreted on indefinite Hermitian form $(a,b_1,b_2,c)$ where (p.455 ff.) by the hemisphere:
\begin{align}\label{eq:71}
    a(\xi^2+\eta^2+\vartheta)+2b_1\xi-2b_2\eta+c=0.
\end{align}
standing perpendicularly to the \zeta-plane and provided with an appropriate arrow.

In the definition of the “reduced” form, we proceed here analogously, as above: The indefinite Hermitian form $(a,b_1,b_2,c)$ is called “reduced”, in case its representing hemisphere has a finitely extended surface portion in common with the initial space of the unimodular Picard group. The representing hemisphere is thus either one of those facial surfaces of the double pentahedron(1) on p.457, which latter is to be counted, or it intersects this pentahedron in a surface portion of non-vanishing content.

Since, because of $D>0$, it is a question in (1) of a sphere of non-vanishing radius, then one immediately concludes: Each indefinite form $(a,b_1,b_2,c)$ is transformable into a reduced form equivalent to it.

In carving out the reduction theory further, it is often expedient, to treat  by themselves those reduced forms, whose representing hemisphere are facial surfaces of the initial space. There are only two inequivalent hemispheres(halfplanes) of this kind; and we are able to the halfplane $\eta=0$ and the hemisphere of radius $\mathrm{1}$ about $\zeta=0$ as such. The two associated “primitive” forms are $(0,0,1,0)$ and $(1,0,0,-1)$; and self-evidently, all “inprimitive” forms also belong here, which proceed from those two by addition of factors. The determinant of the primitive forms is $D=1$ in both cases.

Now apart form the cases just mentioned, a proper penetration of the initial space is always present for the representing hemisphere of a reduced form $(a,b_1,b_2,c)$. But in order, on the strength of this, to formulate the reduction conditions, arithmetically, one has to distinguish, whether the first coefficient a vanishes or not.

If $a=0$, then \ref{eq:71} represents a plane. In order that this penetrate the initial space, among the four non-parabolic vertices of this domain, two must let themselves be given, which are \textcolor{red}{situable} on different sides of the plane \ref{eq:71} and not on it. If we bring in the coordinates of more vertices, then follows: An indefinite form $(a,b_1,b_2,c)$ with $a=0$ is always and only then reduced, if among the far whole numbers:
\begin{align}\label{eq:72}
    b_1+c, -b_1+c, b_1-b_2+c, -b_1-b_2+c,
\end{align}
at least one is $<0$, and, at the same time, at least one is $>0$.

If, however, $a\gtrless0$, then one has a proper hemisphere in \ref{eq:71}. If this is to penetrate the initial space, then, of the four vertices of the initial space just utilized, at least one must lie in the interior of the hemisphere. There consequently yields itself: An indefinite Hermitian from $(a,b_1,b_2,c)$ with $a\gtrless 0$ is always and only reduced, if, of the four whole numbers:
\begin{align}\label{eq:73}
    a^2\pm ab_1+ac, a^2\pm ab_1-ab_2+ac
\end{align}
at least one is $<0$.

The arithmetic formulation of the reduction conditions play the same role here again, as far the forms considered before. We are able, by means of those conditions, to show, that for given determinant $D$, the number of reduced indefinite forms, and consequently, also the class number, is a bounded one.

If, namely, $a=0$, then an account of $D=b_1^2+b_2^2$, we have, for given $D$, a bounded number pairs $b_1, b_2$. For the individual pair $b_1, b_2$, then, $c$ is included within finite bounds, since, among the expressions \ref{eq:72}, in each case, one must be $>0$ and one $<0$.

If, on the other hand, $a\gtrless 0$ and one of the first two numbers \ref{eq:73} is negative, then one replaces as by $(b_1^2+b_2^2-D)$ and has, for at least one of the two signs in question:
\begin{align}
    a^2\pm ab_1+b_1^2+b_2^2<D.
\end{align}
One easily transforms this inequality into the form:
\begin{align}
    (2a\pm b_1)^2+3b_1^2+4b_2^2<4D,
\end{align}
and recognized form this, that for given $D$, the whole numbers $a, b_1, b_2$ are only able to be chosen in finitely many ways. But, with $D, a, b_1, b_2, c$, is uniquely determined. If, finally, one of the last two numbers \ref{eq:73} is negative, then one of the inequalities holds:
\begin{align}
    (2a-b_2\pm b_1)^2+(b_1\pm b_2)^2+3b_1^2+2b_2^2<4D.
\end{align}
the discussion of this again leads to the desired result, we are now also able to develop an “algorithm of continuous reduction” for the indefinite Hermitian forms.

If here too, for the sake of more fluent language, the forms belonging to the facial surfaces of the initial space remain initially excluded, then one representing semisphere of an individual indefinite form $(a, b_1, b_2, c)$ will intersect infinitely many double pentahedra of the halfspace division. The surface of the hemisphere, for its part, thereby appears covered by a net of infinitely many circular-arc polygons with three, four or five sides, whereby these circular-arc polygons press together infinite number toward each point of the boundary of the hemisphere substituted in the $\zeta$-plane.

Each of these circular-arc polygons furnishes, by means of its transfer backs into the initial space, a substitution, which transforms the proposed form into a reduced form equivalent with it; and we arrive in this way, at the same time, at all reduced forms of the proposed class. But we recognize here, above all, a “netlike connection” of all these reduced forms; for one such is immediately given by the character of the positioning of the circular-arc polygons onto the representing hemisphere of the form $(a,b_1,b_2,c)$.

The process of continuous reduction will now consist in our now actually producing the connection of the reduced forms thus recognized resp. their spherical segments substituted in the initial space. One will obviously continues the individual such spherical surface-portion into the three resp. four or five neighboring double pentahedra, and transfer \textcolor{red}{here} into the initial space the spherical segments to the net here. In this way, one obtains, as “neighboring” worth that first reduced form, three resp. four or five particular additional reduced forms, and, by continuation of this operation, can arrive at each reduced form of the class.

The substitutions to be applied in this process, in general, \textcolor{red}{here}, only the five generators of  of the unimodular Picard group named in \ref{eq:73} on p.461. only in case the hemisphere of an individual reduced form runs through one of the tight edges of the initial space, will one of the following eight substitutions have to be exercised for the calculation of the appropriate neighboring form:
\begin{align}
    \left\{\begin{array}{ccc}
    ST^{\pm 1}\left(\begin{array}{cc}
    \ii & \mp \ii\\
    0 & -\ii\\
    \end{array}\right) & VT^{\pm 1}=\left(\begin{array}{cc}
    \ii & 1\mp \ii\\
    0 & -\ii\\
    \end{array}\right) & SU=\left(\begin{array}{cc}
    0 & \ii\\
    \ii & 0\\
    \end{array}\right)\\
    T^{\pm 1}UT^{\pm 1}=\left(\begin{array}{cc}
    1 & 0\\
    \mp 1 & 1\\
    \end{array}\right) & VUV=\left(\begin{array}{cc}
    \ii & 0\\
    1 & -\ii\\
    \end{array}\right) & 
    \end{array}\right.
\end{align}
these eight are among the twelve substitutions already mentioned at a corner \textcolor{red}{here} place for the Dirichlet forms(p.462).

We now combine the perceptions thus obtained with the theorem of the finiteness of the number of reduced forms for given D. There immediately arises the important theorem: the process of continuous reduction always ends, from a first reduced form, the only “finitely many” further reduced forms of the same class. There thus yields itself, for each class of indefinite Hermitian forms, a determinate, completely \textcolor{red}{here} “net of reduced forms”, which obviously represents the analogy of the “period of reduced Dirichlet forms”.

One now recognizes without difficulty, that the results obtained also remain in force for those forms, which beong to the facial surfaces of the initial space. There arises here too, without further \textcolor{red}{ado}, a division of the individual hemisphere into circular-arc triangle resp. quadrangles, which have the same signature or the continuous reduction of the forms now intended, as we just become acquainted with them in general. Thereby, one has only, in each case, to make use of the precise prescriptions, which decide the coordination of the facial surfaces of the individual double pentahedron.---

The question of the equivalence of two indefinite Hermitian forms of the same determinant now disposes of itself in known fashion, Form each of the two given forms, say by mediation of the halfspace division, one must go over to a reduced form, and form here, by the process of continuous reduction, produce the entire net of reduced forms. The given forms are always and only then equivalent, if these nets are identical.


\section{The reproducing groups of the indefinite Hermitian forms}

Now secondly, we have to give all those substitutions, which transform a proposed indefinite Hermitian form $(a,b_1,b_2,c)$ into itself. Hereby, it is a question of the group of all these substitutions of the unimodular Picard group, which carry the representing hemisphere of the form $(a,b_1,b_2,c)$, together with its directional arrow, over into itself. This subgroup, which in the sequel will be heavily referred to, we designate, in abbreviation, as the “reproducing group” at the Hermitian form $(a,b_1,b_2,c)$. We recognize in it a properly discontinuous principal-circle group which refers itself to the representing hemisphere of the form $(a,b_1,b_2,c)$, and which has the base circle of this hemisphere as principal circle. With this, we have obtained a first one among the approaches to the “arithmetic” definition of principal-circle groups considered in the introduction (p.447).

The more precise theory of rthe reproducing groups of the indefinite forms $(a,b_1,b_2,c)$ is established through the developments of the previous paragraph.

On the representing hemisphere, a net of polygons was cut out by the pentahedral division of the halfspace, \textcolor{red}{here} furnished is the reduced forms of the class. But the number of those latter forms is finite, let us sayt equal to $\gamma$. It follows immediately: The discontinuity domain $P_0$ of the reproducing group lets itself be represented as a complex of $\nu$ circular-arc polygons of the kind just mentioned. We still call the latter the “partial polygon” of the domain $P_0$. The boundary curves of the paritail polygon, and, with this, of the domain $P_0$, are circular-arcs, which are erected perpendicularly to the principal circle. Naturally, $P_0$, as all discontinuity domains of principal-circle groups, represents a simply connected domain\footnote{*}.

From the finiteness of the number $\gamma$ of the partial polygons follows, that $P_0$ has only finitely many sides. Since, further, the partial polygons press themselves together in infinite number toward each point of the principal circle, then the same thing will hold for the discontinuity domains $P_0$, $P_1$, $P_2,\;\ldots$ With this follows the theorem: The reproducing group of an indefinite Hermitian form is a principal-circle group of “finite” character $(p,n)$, which is “properly” discontinuous on the principal circle itself.

Equivalent forms $(a,b_1,b_2,c)$ and $(a',b_1',b_2',c')$ furnish reproducing groups of the same “class”, which, as transformable into one another, do not count as distinct from the standpoint of invariant theory(p.355 ff.). We shall accordingly \textcolor{red}{here} investigate reduced forms for their reproducing groups. The groups of the remaining forms are to be obtained from here simply by transformation.

The latter remark is important for settling up a system of generating substitutions of the individual reproducing group. Here too, the building up of the polygon $P_0$ from its $\gamma$ partial polygons is fundamental. The process of continuous reduction furnishes the substitutions, which furnish the progression from the individual partial polygon to the neighboring ones. By combining these substitutions according to the prescription of the arrangement of the partial polygons, we arrive at the generators of the reproducing group.---

The equivalence theory of the Hermitian forms of positive determinant is hereby brought to an end in its principal points. Nevertheless, one more series of further discussions of interest might be added here.

First, we remark, that we are able to position the polygon net to the reproducing group, instead of on the representing hemisphere of the form, \textcolor{red}{here} in the $\zeta$-plane, and indeed, outside or inside the base circle of that hemisphere. It is a question hereby of a projection of the hemisphere onto the $\zeta$-plane by semicircles, which \textcolor{red}{here} perpendicularly to the hemisphere as well as to the $\zeta$-plane. The individual point of the hemisphere consequently furnishes two points of the $\zeta$-plane, which are symmetric to one another with respect to the base circle of the hemisphere, i.e., principal circle of the group.

It is not entirely immaterial, whether we utilize the polygon net in the original or the new form. A distinction emerges, e.g., for the “self-inverse” form classes. Should an indefinite form $(a,b_1,b_2,c)$ be equivalent with its inverse form $(-a,-b_1,-b_2,-c)$, then there is, in the unimodular Picard group (of the first kind), one, and with this, at the same time, infinitely many, substitutions, which carry the principal circle over into itself with reversal of the direction of the arrow, The individed one of these substitutions will, in the $\zeta$-plane, interchange the interior of the principal circle with the exterior; however, it transforms the representing hemisphere into itself, and indeed, with reversal of the angles.

We have the simplest example belonging here, in case the representing hemisphere runs through an elliptic axis belonging to the period two, Here than, the fact of the axis are the fixpoints of the elliptic substitution in question. For the hemisphere, this substitution obviously obtains the character of operation of the same kind, since it represents a reflection in the axis just named. 

From this study follows the theorem : For an indefinite Hermitian form equivalent with its inverse form, all substitutions contained in the unimodular Picard group of the first kind, which transform the form either into itself or its inverse, form a group, in which the reproducing group of the form is a distinguished subgroup of index two. The more comprehensive group, following the language on p.132, is to be designated as a group of the “second type”; but on the representing hemisphere it has the character of a group of the “second kind.” That these two kinds of groups are not essentially distinct from one another, was directly remarked on p.141.——

Besides the self-inverse classes, we also have here the ambiguous form classes resp. forms. In this respect, the following definition is to be placed in a position of reeminence: If the representing hemisphere of an indefinite definition is to be placed in a position of preeminence: If the representing hemisphere of an indefinite form $(a,b_1,b_2,c)$ runs orthogonally to one (and, with this, to infinitely many) symmetry hemisphere of the pentahedral division of the halfspace, then the form in  question is called “ambiguous”. The special character of the associated reproducing group is immediately evident: The reproducing group of an ambiguous indefinite form $(a,b_1,b_2,c)$ permits an extension by reflections to a group of the second kind\footnote{*}. This character of the group is then present on the representing hemisphere as well as in the $\zeta$-plane.---
 
A similarity important group extension arises itself on the addition of substitutions of the determinant $i$, which furthermore \textcolor{red}{here} the structure of the $\zeta$-substitutions applied so far. The “extended” Picard group of the first kind thus arising, in which the “unimodular” one is a distinguished subgroup of index two, was likewise investigated in detail above(p.77 ff.) and has as(?) discontinuity domain a “double tetrahedron” determined by the conditions (6) on p.35.

The substitution to the added here, according to the agreement on p.453, furnish Hermitian forms equivalent “in the extended sense”; the determinants of two such forms reveal themselves as equal. One now obviously has here the following alternative: An individual form class of the determinant $D$, upon exercise of one of the new transformations, is either permuted with a second class of the same determinant or is hereby transformed into itself. In the first case, the reproducing groups belonging to those two classes are conjugate inside the extended Picard group, in the second case, the reproducing group itself is capable of extension by addition of substitutions of the determinant $i$. We will then distinguish between an “extended” and a “unimodular” reproducing group (of the first kind); naturally, this one is combined in that one as a distinguished subgroup of index two.——

The question of the accurnence of parabolic substitutions inside the individual one of our reproducing groups has a special significance. One calls two groups “commensurable” if, either directly, or often suitable transformation of one group, they have a subgroup in common, which has finite index inside both groups\footnote{**}.  We shall later be able to show, that the reproducing group of an indefinite Hermitian firm is always and only then commensurable with the ordinary modular group, if it contains parabolic substitutions.

Now parabolic substitutions will occur or not, according as the discontinuity domain of the reproducing group positioned on the representing hemisphere \textcolor{red}{here} down with one or more cusps to the $\zeta$-plane or remains entirely distant from the boundary of the hemisphere. But the first or second of these cases enters, according as the base circle passes through parabolic, i.e., rational points of the $\zeta$-plane or not.

On the strength of this, one sets $\xi$ and then, in consequence of (1) on p.454, has as the equation of the base circle:
\begin{equation}\label{eq:81}
ax^2+ay^2+cz^2-2b_2yz+2b_1xz=0.
\end{equation}
Parabolic substitution will always and only then occur, if this equation lets itself by solved by a triple of rational whole numbers $x$,$y$,$z$. For $a=0$, the existence of such a solution is immediately evident. If $a\gtrless0$, carry the last equation, nu multiplication by a,over into the form:
\begin{equation}
(zx+b_1z)^2+(ay-b_2z)^2-Dz^2=0.
\end{equation}
If $d^2$ is now the greatest \textcolor{red}{here} of a whole rational number going into $D$, then write $D=d^2-D_0$ and moreover set:
\[
ax+b_1z=X,\; ay-b_2z=Y,\; dz=Z.
\]
then with $x$, $y$, $z$, $X$, $Y$, $Z$ are also whole rational numbers, and the latter satisfty the equation:
\begin{equation}\label{eq:82}
X^2+Y^2-D_0Z^2=0.
\end{equation}
Conversely, an integral solution of this equation always furnishes a triple of rational, with this, however, also(?) one such of whole rational, numbers $x$, $y$, $z$ which satisfy equation\ref{eq:81}.

The question of the solubility of equation \ref{eq:82} in whole, not simultaneously vanishing, numbers is now answered by a know theorem of number theory\footnote{*}. The condition, easily recognizable as necessary, namely that -1 is a quadratic residue of $D_0$, is also sufficient for the existence of an integral solution. But -1 will always and only then quadratic a quadratic residue of $D_0$, if no prime number of the form $(4n+3)$ is contained in $D_0$. The theorem therefore hold: Whether, in an individual reproducing group, parabolic substitutions occur or not, depend solely on the numerical value of the associated determinant; the group will always, but also only then, be free of parabolic substitutions, if in $D$ at least one prime number of the form $(4n+3)$ is contained in an odd (highest) power.

Furthermore, we notice, that the indefinite Hermitian forms are \textcolor{red}{exhausted} for the theory of the principal-circle groups inside circle group does not only belonging to each indefinite form $(a,b_1,b_2,c)$; but we shall be able to show at the end of the chapter, that also, conversely, to each such subgroup an indefinite Hermitian form.

In closing, we return once more to the historical development of the theory developed here.

Here, first of all,, besides the references to the historical given on p.92 ff., one more paper of Picard’s on indefinite Hermitian forms is to be named*. This work is to be considered  as a direct continuation of Hermite’s original investigations(in Bd.47 of Credle's Journal), which latter learns undisposed of the indefinite forms $(a,b_1,b_2,c)$. The reduction of the indefinite forms is reduced by Picard 1.c.), by means of a principal originating from Hermite, to that of the definite forms. This principle, which comes into force in even more detail in the next chapter, lets itself be force, by using our geometrical language, in the following manner: A positive Hermitian form is to be designated as “associated: to a given indefinite form, it the representing point of that form is situated on the hemisphere of this latter\footnote{ **}. An indefinite form is then called reduced, if there lets itself be given a definite form associated to it, which is reduced. How, on the basis or this approach, the process of continuous reduction lets itself be introduced, we shall likewise have to explain in the next chapter. This process, then, also plays a very important role for Hermite and Picard 1.c.; however, let it be remembered, that the source of the process in question is to be sought for in "Disquisitions arithmetica"\footnote{a}

It is now immediately evident, that the method of the definition of reduced indefinite forms followed here, and going back to St. Smith, lead materially to the same results, as the Hermitian approach (equal reduction conditions of the definite forms assumed). All the same, the Smith principle, of basing the definitions of the reduced indefinite forms, independently of the definite ones, on a developed theory of groups and discontinuity domains, must be considered as a vey important step vis-à-vis Hemite, which furnished a far more transparent structure of the equivalence theory of indefinite forms. Furthermore, the service is due to Bionchi, to have extended the Smith method to the Dirichlet and Hemitian form\footnote{*}.

`\section{The reproducing groups of the Hermitian forms belonging to the determinant $D=5$}

The simplest example after the illustration of the developments of the last two paragraphs would be furnished \textcolor{red}{here} by those form classes, whose representing hemisphere are the symmetry hemispheres of the pentahedral division. But it reveals itself, that we arrive here only first to very simple relations: namely, one is led to only two different reproducing groups, of which one is the ordinary modular group, and the other the group of the circular-arc triangle of the angles $\dfrac{\pi}{2}, \dfrac{\pi}{4}, 0$. We shall later find occasion, to investigate these latter groups directly and accordingly, \textcolor{red}{here} not longer here by the examples in questions.

For the treatment of more instructive example, we first make a few more remarks regarding the numerical calculations to be carried through here.

First of all, for given $D$, one can set up for oneself a table of the reduced forms following the prescriptions on p.466.

For the individual hemisphere hereby entering:
\begin{align}\label{eq:91}
    a(\xi^2+\eta^2+\vartheta^2)+2b_1\xi-2b_2\eta+c=0,
\end{align}
the intersection with the initial space of the unimodular Picard group of the first kind is then to be determined. The first facial surfaces of the initial space correspond to the first generators $\textcolor{red}{here}, T, T^{-1},U,V$ and are, in this order, to be distinguished by numbers $\mathrm{1}$ to 5. The equations of these surfaces are, in order:
\begin{align}\label{eq:92}
    \eta=0, \xi=\frac{1}{2}, \xi=-\frac{1}{2}, \xi^2+\eta^2+\vartheta^2=1, \eta=\frac{1}{2}.
\end{align}

The boundary curves of the segment of the sphere \ref{eq:91} situated in the initial space is correspondingly to be described by numerals 1 through 5, according as they are furnished by the first, etc. sided \ref{eq:92}.

The hemisphere \ref{eq:91} can, however, in particular, run through one of the eight edges of the initial space. By then placing the numerals of the two participating facial surfaces next to one another in parametres, we use the symbols $(1,2), (1,3)$ for the designation of the edges and, with this, at the same time, of the respective sides of the intersection polygon \textcolor{red}{here} on the hemisphere \ref{eq:91}.

For the edges in question, in place of the generators $S, T, \ldots$, the substitutions given in (4) on p.470 appear. Furthermore, these eight edges are not conjugate. Three of them, namely $(2,4), (3,4)$ and $(4,5)$, belong to elliptic substitutions of period three; here we have ordinary sides of the partial polygons. The five other edges $(1,2),(1,3),(1,4),(2,5),(3,5)$ belong to the period two and consequently furnish symmetry circles of the polygon net to be constructed on the hemisphere (1) (cf. p.473). One need not continue the net of the partial polygons over and beyond such a side; for on the other side, indeed, the forms inverse to the previous ones will again find themselves, in the symmetric arrangement.

In the case occurring, these three edges of the “elementary pentahedron”, which symmetrically \textcolor{red}{here} the faces 1,4 and 5, obtain just this same character of symmetry by circles. To such symmetry lines of our polygon net we assign the symbols (1), (4), (5).

In order to make these prescriptions easily surveyable to our vision, in figure 160 is drawn the perpendicular projection of the initial space onto the $\zeta$-plane, whereby, as grand \textcolor{red}{here} appears a rectangle with the vertices $\zeta=\pm\dfrac{1}{2},\pm\dfrac{1}{2}+\dfrac{\ii}{2}$. The facial surfaces 1, 2, 3, 5 project themselves hereby into lines (which carry the corresponding numerals), the surface 4, on the other hand, furnishes the interior of the rectangle of the edges, $(1), (5), (1,2), (1,3), (2,5), (3,5)$ will project into points, while the fine remaining one furnish lines.
%\begin{figure}
%\centering
%\includegraphics[width=0.7\linewidth]{}
%\end{figure}

In one would now determine the intersection polygons of the hemisphere with the initial space, then next discuss the question, whether this polygon has a side 4. The projection of such a side on the $\zeta$-plane has the equation:
\begin{align}\label{eq:93}
    2b_1\xi-2b_1\eta+(a+c)=0,
\end{align}
as one sees immediately by combination of the relevant equations \ref{eq:91} and \ref{eq:92}. A side 4 will accordingly always and only then occur, if the line represented in \ref{eq:93} cuts the rectangle of figure 160. If this is not the case, then the intersection polygon sought is a quadrangle which four sides 1, 2, 5, 3. In each case, however, by addition of the position of the center and radius of the sphere \ref{eq:91}, one will easily inform oneself further on the carse of the intersection polygon.

It is here also to be given in general, at which “neighboring” forms one arrives, if one should leave a partial polygon over a side $1, 2, \ldots$. If the first form is $(a,b_1,b_2,c)$, and the neighboring one $(a',b_1',b_2',c)$, then according to (3) on p.461 and (3) on p.453, one has the following transformation formulas in the five cases in question:
\begin{enumerate}
    \item $a'=a, b_1'=-b_1, b_2'=-b_2, c'=c$,
    \item $a'=a, b_1'=a+b_1, b_2'=b_2, c'=a+2b_1+c$,
    \item $a'=a, b_1'=-a+b_1, b_2'=b_2, c'=a-2b_1+c$,
    \item $a'=c, b_1'=-b_1, b_2'=b_2, c'=a$,
    \item $a'=a, b_1'=-b_1, b_2'=a-b_2, c'=a-2b_2+c$,
\end{enumerate}
Next to these there arrange themselves the three further transofrmations referring themselves to the edges $(2,4), (3,4)$, and $(4,5)$:
\begin{align}
    \begin{array}{ccccc}
    (2,4) & a'=a+2b_1+c, & b_1'=b_1+c, & b_2'=b_2, & c'=c,\\
    (3,4) & a'=a-2b_1+c, & b_1'=b_1-c, & b_2'=b_2, & c'=c,\\
    (4,5) & a'=a-2b_2+c, & b_1'=-b_1, & b_2'=-b_2+c, & c'=c.\\
    \end{array}
\end{align}
We now go over to the proper subject of the present paragraph, namely, by means of the prescriptions just given, to investigate the reproducing groups of the form classes belonging to the determinant $D=5$.

One will designate the form $(1,0,0,-5)$ as the principal form of the determinant $D=5$. The “principal class” belonging to it was, as the effectuation of the preceding methods of investigation generally described shows, 48 reduced forms, pairwise inverse to one another. One of the two symmetric halves of the net is schematically reproduced in figure 161. In the 34 polygons, in each case, the associated reduced form is inscribed and the sides are characterized by their numerals everywhere. A side 2 is thereby always a side 3 for the neighboring polygons and conversely.  The remaining three sides, however, in this specific situation, retain their numerals for the neighboring partial polygon, corresponding to the circumstance, that the associated substitutions $S, U, V$ are of period two. Therefore, for these sides, in each case, only one numeral is inscribed. The boundary curves of the net remaining open will mostly be furnished by pentahedral edges; they lay themselves together into five symmetry lines, which are distinguished by the numerals $\Rmnum{1}$ through \Rmnum{4} in the figure. Of the boundary curves still remaining, one half is mapped onto the other by a generator of the reproducing group, which is indicated by the arrow appended in the figure. The four vertices surrounded with small circles are parabolic; the occurrence of such vertices was to be expected according to the criterion p.475.
%\begin{figure}
%\centering
%\includegraphics[width=0.7\linewidth]{}
%\end{figure}

The half-polygon furnished in such a way by the process of continuous reduction is a discontinuity domain of the second kind with five sides of the second kind, namely symmetry circles, and two sides of the first kin, which join together at a recumbent angle. Onto which hemisphere belonging to the form class we imagine the net positioned is \textcolor{red}{here} in itself; however, will\textcolor{red}{?} will utilize, most expediently, the representing hemisphere of the principal form $(1,0,0,-5)$ itself. Figure 161 then informs us, that the two non-parabolic vertices of the half-polygon are \textcolor{red}{here} at $\zeta=\pm 2,\vartheta=1$, and the four parabolic ones at $\zeta=\pm 2+\ii$ and $\zeta=\pm 1+2\ii$. The projection by orthogonal semicircles furnishes, in the $\zeta$-plane, the two polygons represented in figure 162, which comprise the discontinuity domain of that principal circle group of the second type contained in the unimodualr Picard group of the first kind, where substitutions transform $(1,0,0,-5)$ either into itself or into $(-1,0,0,5)$. The generators of this group, which were characterized more closely in figure 162 by arrows, are the following:
\begin{align}\label{eq:95}
    \left\{\begin{array}{ccc}
    S=\left(\begin{array}{cc}
    \ii & 0\\
    0 & -\ii\\
    \end{array}\right), & V_1=\left(\begin{array}{cc}
    2 & -5 \\
    1 & -2 \\
    \end{array}\right), & V_2=\left(\begin{array}{cc}
    3 & -5-5\ii\\
    1-\ii & -3\\
    \end{array}\right),\\
    V_3=\left(\begin{array}{cc}
    2 & -5\ii\\
    -\ii & -2\\
    \end{array}\right), & V_4=\left(\begin{array}{cc}
    3 & 5-5\ii\\
    -1-\ii & -3\\
    \end{array}\right), & V_5=\left(\begin{array}{cc}
    2 & 5\\
    -1 & -2\\
    \end{array}\right).
    \end{array}\right.
\end{align}
It is a question here, as one sees, exclusively of elliptic substitutions of period two, of which the last five in \textcolor{red}{here} the interior of the principal circle with the exterior.
%\begin{figure}
%\centering
%\includegraphics[width=0.7\linewidth]{}
%\end{figure}

In order to obtain the discontinuity domain of the reproducing group of $(1,0,0,-5)$ itself, we are able to restrict ourselves to the interior of the principal circle and position, beside the previous half-polygon, say along the side denoted by \Rmnum{3} in figure 161, a half-polygon symmetric with it, as figure 163 indicates this. The system of the generators described more closely in the figure is:
\begin{align}
    \left\{\begin{array}{cc}
    S=\left(\begin{array}{cc}
    \ii & 0\\
    0 & -\ii\\
    \end{array}\right) & 
    V_3'= V_3SV_3=\left(\begin{array}{cc}
    9\ii & 20\\
    4 & -9\ii\\
    \end{array}\right)
    \\
    V_1'=V_1V_2=\left(\begin{array}{cc}
    4+5\ii & 10-10\ii\\
    2+2\ii & 4-5\ii\\
    \end{array}\right) & 
    V5'= V_ 5V_3=\left(\begin{array}{cc}
    4-5\ii & -10-10\ii\\
    -2+2\ii & 4+5\ii\\
    \end{array}\right)
    \\
    V_2'=V_2V_3=\left(\begin{array}{cc}
    1+5\ii & 10-5\ii\\
    2+\ii & 1-5\ii\\
    \end{array}\right) &
    V_4'= V_4V_3=\left(\begin{array}{cc}
    1-5\ii & -10-5\ii\\
    -2+\ii & 1+5\ii\\
    \end{array}\right)
    \end{array}\right.
\end{align}
Above all else, we take note of the theorem: the reproducing groups of the principal class of the determinant $D=5$ are principal circle groups from the family of the signature:
\begin{align}
    (0,6,2,2,\infty,\infty,\infty,\infty)
\end{align}
One will easily carry out the transformation of the domain of figure 163 into a “canonical” polygon.---

The arithmetic law of formation of the principal-circle group was only first given indirectly so far, and indeed, that it consists of all substitutions of the unimodular Picard group, which, \textcolor{red}{here} into the form (1) on p.453, transform the form $(x\bar{x}-Sy\bar{y})$ into itself.

Now, however, we can also directly bring into account the properties of our substitutions hereby defined, and thus find, that we have to do with all substitutions, whose coefficients are whole numbers of $\Omega$ satisfying the three conditions:
\begin{align}\label{eq:96}
    \alpha\delta-\beta\gamma=1, \alpha\bar{\alpha}-5\gamma\bar{\gamma}=1, \alpha\bar{\beta}=5\delta\bar{\delta}.
\end{align}
In case none of the numbers $\alpha, \beta, \gamma, \delta$ vanishes, we conclude from the last equation \ref{eq:96}, that $\delta=\varepsilon\times\cdot\bar{\alpha}$, $\beta=\varepsilon\cdot 5 \bar{\gamma}$, understanding by $\varepsilon$ a \textcolor{red}{here} or fractional number of $\Omega$. The first equation \ref{eq:96} then furnishes $\varepsilon(\alpha\bar{\alpha}-5\gamma\bar{\gamma})=1$, so that, in consequence of the second, $\varepsilon=1$. The case of the \textcolor{red}{here} of $\beta$ (on account of the last and first equation \ref{eq:96}) are then not able to \textcolor{red}{here} too.
 
By this, the following theorem is proved: The reproducing group of the principal form $(1,0,0,-5)$ consist of all unimodular substituions:
\begin{align}
    \zeta'=\frac{\alpha\zeta+5\bar{\gamma}}{\gamma\zeta+\bar{\alpha}}
   \end{align} 
with whole complex coefficients $\alpha, \gamma$ belonging to the field $\Omega$ with this is obtained the full insight into the arithmetic law of formation of our groups; for that the substitutions in question form a group in their totality, is evident from a mere glance at the schema:
\begin{align}
    \left(\begin{array}{cc}
    \alpha & 5 \bar{\gamma}\\
    \gamma & \bar{\alpha}\\
    \end{array}\right)\cdot\left(\begin{array}{cc}
    \alpha' & 5 \bar{\gamma}\\
    \gamma' & \bar{\alpha}'\\
    \end{array}\right)=\left(\begin{array}{cc}
    \alpha\bar{alpha}'+5 \bar{\gamma}\gamma' & 5(\alpha\bar{\gamma}'+\bar{\alpha}'+\bar{\gamma})\\
    \bar{\alpha}\gamma'+\alpha'\gamma & \bar{\alpha}\bar{\alpha}'+5 \gamma\bar{\gamma}'\\ 
    \end{array}\right).
\end{align}
Furthermore, the law set up emerges directly for the generators \ref{eq:95}.---

We could now also treat the possible extensions of the principal-circle group in question. Since $(x\bar{x}-b'y\bar{y})$ is transformed into itself by the substitution $\left(\begin{array}{cc}
\ii & 0\\
0 & 1\\
\end{array}\right)$, our group permits the extension by addition of $\zeta'=\ii\zeta$ and one easily presents, say on the representing hemisphere, a polygon of the extended group by solving the polygon corresponding to the net of figure 161 discussed above.

However, at the same time, one can go one more step further, and also add the reflection in the imaginary $\zeta$-axis, which is, in fact, contained in the Picard group at the second kind. The polygon last mentioned then falls into four \textcolor{red}{here} symmetric and congruent circular-arc quadrangles with the angles $\dfrac{\pi}{4}, \dfrac{\pi}{2}, 0, \dfrac{\pi}{2}$, as this is represented by figure 164 in the projection onto the interior of the principal circle. We shall meet this group once more $(0,4,2,2,4,\infty)$ in the next chapter in another connection; we shall be able to show there, that it cannot, at the same time, we combined in a more comprehensive, similarly, properly discontinuous principal circle group as subgroup.---
%\begin{figure}
%\centering
%\includegraphics[width=0.7\linewidth]{}
%\end{figure}

With the principal class, now, all reduced forms of the determinant $D=5$ are not yet exhausted. There remain, in fact, yet two further form classes, which, however, show themselves to the equivalent in the extended sense. Let its office, that we give the net of the reduced forms for one of the two classes; half of this is drawn in figure 165. Here, in fact we have to do again with a self-inverse class. But, corresponding to the circumstance, that the present class is equivalent in the extended sense to the third class belonging to $D=5$, the associated reproducing group does not let itself be extended by addition of substitution of the determinant i (cf. p.474).
%\begin{figure}
%\centering
%\includegraphics[width=0.7\linewidth]{}
%\end{figure}

The appended values of $\zeta$ and $\vartheta$ refer themselves to the case, that we position the net on the representing hemisphere of the form $(2,-1,0,-2)$. Thereby, as is \textcolor{red}{here} in the figure, two parabolic vertices.

The immediate arithmetic definition of our group, in the form belonging to $(2,-1,0,-2)$, one can so formulate, say in connection, with (8) on p.453, that it is a question of all substitutions of the unimodular Picard group, which satisfy the two relations:
\begin{align}
    2\alpha\bar{\alpha}-\alpha\bar{\gamma}-\gamma\bar{\alpha}-2\gamma\bar{\gamma}=2,\\
    2\alpha\bar{\beta}-\alpha\bar{\delta}-\gamma\bar{\beta}-2\gamma\bar{\delta}=-1.
\end{align}
But it does not appear, that one fashion the law thereby given, by further development, into an essentially more transparent form.
\section{the representing group of the Hermitian forms belonging to the determinant $D=7$}

In general approaches developed are also to be carried through for the case of the determinant $D=7$, since, in many aspects, the previous examples still fashion themselves too elementary, while for $D=7$, e.g., in the determination of the generators of the reproducing group, the general rules on p.472 will came into force.

Since for $D=7$ parabolic substitutions are not able to come up, then, as easily seen, only the reduction conditions (3) on p.463 come into force. By discussion of these, one enumerates, in all, 46 pairwise inverse reduced forms.
 
If we now attach to the principal form $(1, 0, 0, -7)$ are apply to it the algorithm of continuous reduction, then there arises a net, which is already composed of all 66 partial polygons. Thus follows: There is only one single class of indefinite forms of the determinant $D=7$, which, as such, is self-inverse.
%\begin{figure}
%\centering
%\includegraphics[width=0.7\linewidth]{}
%\end{figure}

In figure 166 half of the net, with 33 partial polygons, is schematically produced, and, as well, the reduced forms are inscribed everywhere, as too, the bounding \textcolor{red}{tires} of the individual partial polygons and the boundary curves of the entire net are most closely depicted(?) in the manner agreed upon above (p.473Z). In order, however, to fashion the figure symmetrically, the two partial polygons of the reduced forms $(-2,0,-1,3)$ and $(-2,0,1,3)$ have \textcolor{red}{here} cut up into two pieces (and indeed, by means of the symmetry planes of the double pentahedron serving as initial space). One finds these four pieces, which represent quadrangles, on the right one and left sides of the figure; they are coordinated to one another by the arrow denoted by $V_2$.

The sides of the polygon of the second kind obtained, for the most part, belong t the first kind. We have there, first, be two sides already mentioned coordinated to one another by the arrow $V_2$, and then still three further pairs of boundary curves, which are mapped onto one another by the arrows denoted by $V_1$, $V_3$ and $V_4$. One will easily verify the results of the figure everywhere here; thus, e.g., one arrives, from the partial polygon of the form $(-2,1,-2,1)$, by passing over the side 2, i.e, by exercise of the associated transformation given on p.479, in fact, into the partial polygon of the form $(-2,-1,-2,-1)$, \textcolor{red}{here} by $V_3$.

However, there must now necessarily be sides of the second kind, and these are the boundary curves bounded by $\overline{e_1e_2}$, $\overline{e_6e_4}$ and $\overline{e_3e_5}$ in the figure, whereby the Katter is divided by its midpoint $\overline{e_4}$ into the two pieces $\overline{e_3e_4}$ and $\overline{e_4e_5}$. Namely, we are not concerned here with symmetry lines; rather, the sides $\overline{e_1e_2}$ and $\overline{e_3e_4}$ are to be mapped onto $\overline{e_4e_5}$ and $\overline{e_6e_4}$. Thereby, corresponding to the character of the operation of the second kind, in each case a reversal of the individual side enters, so that, e.g., the vertex $e_1$ is mapped onto $e_3$ and $e_2$ onto $e_4$. One will very easily verify these remarks with the figure in hand; e.g., the form $(2,-1,0,-3)$, positioned in figure 166 below on the right, is carried over into the form $(2,-1,0,-3)$ by exercise of the transformation $(2,4)$ given on p.479, which in fact is inverse from situated at the vertex $e_3$.

The net of the 33 partial polygons is now to be positioned on the representing hemisphere of the principal form $(1,0,0,-7)$. This will yield there a symmetrically shaped circular-arc polygon with ten vertices; and one easily determines, that all \textcolor{red}{here} angle, are right. Thus, one needs only notice, e.g., for the vertex situated at $e_1$, that the edge $(2,4)$ stands orthogonally to the side $\mathrm{1}$ of the pentahedron etc.

The operations $V_1, V_2, V_3, V_4$ now receive a uniquely determined interpretation as substitutions which transform the form $(1,0,0,-7)$ into itself. We also add, not the same time, the substitutions $V_5$ and $V_6$, not drawn in the figure, which, in the manner just described, transform $\overline{e_1e_2}$  into $\overline{e_3e_4}$ resp. $\overline{e_6e_4}$ into $\overline{e_4e_5}$, and which carry $(1,0,0,-7)$ over into $-1,0,0,-1)$. In order then, to read off the interpretation of $V_1, \ldots, V_6$ from the net of partial polygons, we must, beginning from the quadrangle of the form $(1,0,0,-7)$, lay the “closed” path, corresponding to the individual substitution $V_i$, through the net and combine the generators $S,T,\ldots$, hereby arranged next to one another, according to the pattern of their sequencing. One reads off directly form figure 166:
\begin{align}
\begin{array}{cc}
V_1=S & V_2=VT^{-2}UT^2UT^{-2}V\\
V_3=VT^{-1}SVT^{-1}UVT^2SV & V_4=VSVSTUVSVUTSVSV\\
V_5=T^2(T,U)VUT^{-1}SVSV & V_6=T^{-2}(T^{-1},U)VUTSVSV
\end{array}
\end{align}
Hereby, the substitutions corresponding to the edges $(2,4)$ and $(3,4)$ are denoted $(T,U)$ and $(T^{-1},U)$ for smart. The result of actually calculating out the generators $V_1,\ldots,V_6$ together with the remaining results, we clothe in the following theorem: The group of all substitutions contained in the unimodular Picard group of the first kind, which either transform the principal form $(1,0,0,-1)$ of the determinant 1 into itself, or into its inverse form $(-1,0,0,7)$, is a principal circle group of the second type (considered in the $\zeta$-plane), which has the following six substitutions as a system of generators:
\begin{align}
\left\{
\begin{array}{ccc}
V_1=\left(\begin{array}{cc}
\ii & 0\\
0 & -\ii\\
\end{array}\right) &
V_2=\left(\begin{array}{cc}
5+2\ii & 14\\
2 & 5-2\ii\\
\end{array}\right) & 
V_3=\left(\begin{array}{cc}
2+2\ii & 7\\
1 & 2-2\ii\\
\end{array}\right) \\
V_4=\left(\begin{array}{cc}
2+5\ii & 14\\
2 & 2-5\ii\\
\end{array}\right) & 
V_5=\left(\begin{array}{cc}
-3+2\ii & 7+7\ii\\
-1+\ii & 3+2\ii\\
\end{array}\right) & 
\\
V_6=\left(\begin{array}{cc}
3+2\ii & 7-7\ii\\
-1-\ii & -3+2\ii\\
\end{array}\right) & &\\
\end{array}
\right.
\end{align}
the substitutions $V_2,V_3,V_4$ are hyperbolic, and their fixpoints lie at:
\begin{align}
\zeta=\pm \sqrt{6}+\ii, \zeta=\pm\sqrt{3}+2\ii, \zeta=\frac{\sqrt{3}+5\ii}{2}
\end{align}
on the principal circle; $V_5$ and $V_6$ are loxodromic, and their fixpoints (naturally situated similarly on the principal circle) are :
\begin{align}
\zeta=\frac{3\pm\sqrt{5}+\ii(3\mp\sqrt{5})}{2}, \zeta=-\frac{(3\pm\sqrt{5})+\ii(3\mp\sqrt{5})}{2}
\end{align}
With this, at the same tine, all means are obtained, in order to geometrically construct, in the $\zeta$-plane, the polygon of the present principal circle group. Namely, this polygon has a symmetry line in the imaginary $\zeta$-axis. But if we extend with the reflection $\bar{V}$ in that axis, then $V^{\pm 1}_2\bar{V}$, $V^{\pm 1}_3\bar{V}$, $V^{\pm 1}_4\bar{V}$ become reflactions, where symmetry circles are directly the boundaries belonging to $V_2$, $V_3$, $V_4$. On the strength of this form (1), one concludes, as the equations of the boundary curves in question:
\begin{align}
\xi^2+\eta^2\mp 5\xi-2\eta+7=0, \xi^2+\eta^2\mp 4\xi-4\eta+7=0, \xi^2+\eta^2\mp 2\xi-5\eta+7=0.
\end{align}
The circles still remaining, however, are likewise easily determinable; \textcolor{red}{here}, for the individual circle, one can always give three circles, to which it runs orthogonally: the equations of the circle in question are:
\begin{align}
5(\xi^2+\eta^2)\mp 28\xi+35=0, 5(\xi^2+\eta^2)-28\eta+35=0.
\end{align}
On the strength of this, the precise shape of the discontinuity domain “ of the second type” is given in figure 167. One is concerned with a discontinuity domain of the “first kind” consisting of two separately situated circular-arc polygons, symmetric with respect to the principal circle. The coordination of the boundary curves is altered with respect to figure 166, insofar as the side $\overline{e_1e_2}$ is now mapped by the loxodramic substitution $V_5$ onto the side $\overline{e_3'e_4'}$ at the outer polygon, and insofar as the corresponding \textcolor{red}{ring} hold for all other sides, which were of the second kind on the hemisphere.

In order to obtain the discontinuity domain of the reproducing group, we first go back to the representing hemisphere of the principal form, divide the decangle there into two septangles by the symmetry plane  (cf. figure 163) and exercise the substitutions $V_5$ resp. $V_6$ on these two hept-angles. We attach the hept-angles to be thus obtained onto the previous ones, as figure 163 immediately carries rgus out, again in the $\zeta$-plane. There arises a completely rectangular sixteen-angle as discontinuity domain, whereby, it is to be noticed, that at the upper part of figure 163, for the sake of clarity, the boundary curves were drawn imprecisely, namely too large.
%\begin{figure}
%\end{figure}

The coordination of the boundary curves is only first partially brought to expression in the figure. For the sides not considered, the common prescription holds, that they belong together with their sides symmetric with respect to the imaginary $\zeta$-axis (centerline of the figure). The generators let themselves be easily produced from (1); thus, e.g., $V_5\phantom{}'=V_5\phantom{}^2$, $V_6\phantom{}'=V_6\phantom{}^2$. For the sake of brevity, we ignore here the calculation of the complete system of generators.

In schematic arrangement, the discontinuity domain of the reproducing group of the form $(1,0,0,-7)$ is reproduced by figure 169. If we ignore the two fixpoints of the elliptic substitutions $V_1$ and $V_1'$ of period two (cf. figure 163), then the remaining sixteen vertices arrange themselves into six cycles, as one will easily determines by means of the arrow in figure 169. Since it is a question here exclusively of rectangular vertices, then, as one likewise rends off easily from the figure, the angle sum of two of the vertex cycles in equal to $\alpha \pi$, that of the four others, $\pi$. On the strength of this, one immediately determines the theorem: The reproducing groups of the single class of indefinite Hermitian forms entering for $D=7$ belong to the family of the signature:

%\begin{figure}
%
%\end{figure}
\[
(1,6;2,2,2,2,2,2)---
\]

The approach of the immediate arithmetic definition of the reproducing group of the principal form $(1,0,0,-7)$ leads us to entirely similar considerations, as we carried them out above (p.432) for the principal form of the determinant $D=5$. Form (8) on p.453 follows, that the substitutions, which transform the principal form $(1,0,0,-7)$ into itself, are characterizible by:
\[
\alpha\overline{\alpha}-7\gamma\overline{\gamma}=1,\;\alpha\overline{\beta}=7\gamma\overline{\gamma},\;\alpha\delta-\beta\gamma=1.
\]

By discussion of these equations, we get the theorem: The reproducing group of the principal form $(1,0,0,-7)$ consists of all substitutions of the form:
\begin{equation}
\zeta'=\dfrac{\alpha\zeta+7\overline{\gamma}}{\gamma\zeta+\overline{\alpha}}
\end{equation}
contained in the unimodular Picard group. That these substitutions, all together, form a group, is immediately evident. Furthermore, one finds just as easily, that those substitutions, which transform $(1,0,0,-7)$ into $(-1,0,0,7)$, are characterized by $\delta=-\overline{\alpha}$ and $\beta=-7\overline{\gamma}$. The generating system (1) verifies these remarks.---

We would now also be able to treat the possible extensions of out present group, and indeed, as well as reflections, as by substitutions of determinant: However, one affects\textcolor{red}{?} these extensions very easily on the basis of the pertinent general approaches. Furthermore, we shall later obtain, from another aspect, the most comprehensive properly discontinuous principal circle group, in which the groups discussed here are contained.

\section{Theory of the Gaussian forms in projective-geometrical form}

The division of the $\zeta$-halfplane into circular-arc triangles, as it belongs to the modular group, underlies the theory of Gaussian forms sketched in $\S 1$. Now considered above (p.74 ff.) the triangle net of the modular group in the “interior of the elliptic of the hyperbolic plane” and already indicated briefly there, that this “projective form” of the modular group also appears very suitable for the geometric theory of the Gaussian forms. A few further discussions concerning this matter will now be in place.

In order to give the structure of the theory as independently as possible, we proceed in the following manner: We interpret the three coefficients a,b,c of a Gaussian from directly, as homogeneous point coordinates in the plane and work ourselves the conic section, which is given by $D=0$, i.e, in detail, by $b^2-ac=0$. By suitable figure of the coordinate system, we are able to invest this conic section with the form of an ellipse; in the interior of this $D<0$, outside, $D>0$. The individual point with integral coordinates a,b,c will be able to be designed briefly as a “rotational” point of the plane.

Now the following determination yields itself immediately: The individual rotational point with the coordinates a,b,c in the interior of the ellipse id directly the representation of the definite Gaussian form $(a,b,c)$, and similarly, a rotational point in the exterior of the ellipse represents an indefinite Gaussian form $(a,b,c)$.

Two properly or improperly equivalent Gaussian forms $(a,b,c)$ and $(a',b',c')$ are connected with one another by means of the equation system:
\begin{equation}
\left\{
\begin{split}
a'&=\alpha\alpha^2+2b\alpha\gamma+c\gamma^2\\
b'&=a\alpha\beta+b(\alpha\delta+\beta\gamma)+c\gamma\delta\\
c'&=a\beta^2+2b\beta\delta+c\delta^2\\
\end{split}\right.
\end{equation}
where $\alpha$, $\beta$, $\gamma$, $\delta$ are for rational whole numbers of the determinant $1$ resp. $-1$. Thus equation system represents an integral unimodular collineation of the ellipse $D=0$ into itself; and if we collect all the substitutions of this kind, then, as was already determined on p.75, we are led back precisely to the projective form of the extended modular group. It follows: The equivalence of the Gaussian forms coincides precisely with the equivalence of the representing points with respect to the modular group.

Now, to the projective form of the modular group belongs the division of the interior of the ellipse into a net of rectilinear triangles given in figure 13 on p.75. This net is reproduced and \textcolor{red}{here} in figure 170; 
%\begin{figure}
%\end{figure}
and thereby, the usual initial space is first worked heavily for the group of the “first kind, and then also, \textcolor{red}{here} particular symmetry lines of the net, of which one, as joining-line of the points $\zeta=0$ and $\zeta=\infty$, symmetrically divides the initial space just mentioned, while the other represents a side of that double triangle. We shall immediately have to make use of these two symmetry lines for the indefinite forms. Outside the ellipse, in \textcolor{red}{here} distinction to this, as we already concluded on p.76 from the theory of indefinite Gaussian forms, the modular group was improperly discontinuous; here, therefore, there is no division into finitely extended discontinuity domains.

These geometrical relations become decisive for the projective form of the theory of the Gaussian forms. 

The representing point of a definite form lies in the interior of the ellipse. We call the form “reduced” in case the associated point belongs to the initial space (cf. figure 170). The reduction conditions in arithmetic form presented in (3) on p.449, in homogeneous coordinates, immediately yield the definition of the initial space. One only wants to notice, that the sides $a=0$ and $c=0$ of the coordinate triangle are the ellipse tangents at the points $\zeta=\infty$ resp. $\zeta=0$, while the joining-line of these two points furnishes the third side $b=0$. The quotients of the coordinates, however, are to be fixed more closely, so that the sides of the initial space are given by
\begin{equation}
a+\alpha\beta=0,\; a-\alpha b=0,\;a-c=0.
\end{equation}
From here, the arithmetic reduction conditions (3), (4) on p.449 arise directly, if one only yet odds, which boundary points of the initial space are to be counted as part of this space, as well as, that those conditions refer themselves to “positive” forms. The further development of the theory of the definite forms now fashions itself exactly as far the use as the $\zeta$-halfplane.

Things lie completely differently for the indefinite forms, since, outside the ellipse, no division into finite domains exists belonging to the group. Here, from the projective character of the consideration, it appears as the indicated path, to bing in, in place of the points outside the ellipse, its polar in the interior of the ellipse, as the representation of the individual indefinite form. With this, however, we are led back to the projective form of the Smith semicircles. An indefinite form is now called “reduced”, in case its representing line cuts the initial space; and just this requirement finds its expression in the arithmetic reduction condition (5) on p.449, namely, that:
\begin{equation}
a(a\pm b+c)<0
\end{equation}
is to hold either for one or both signs. The further continuation of the study then naturally also fashions itself here essentially as for the use of the $\zeta$-halfplane.

Moreover, it is yet to be added here, that the reduction condition (2) deviated from the original Gaussian reduction condition for indefinite forms. For the geometrical conception of this latter condition, compare the work of Hurwitc “\textcolor{red}{here}”\footnote{*}. There, a synthetic construction of the rectilinear triangle net belonging to the modular group is furnished, and indeed, the development is based on the so-called “elementary chords of the first and second kinds”, which are defined arithmetically, but are moreover identical with the symmetry lines of the net of Fig. 170 consisting of two resp. four triangle sides.

The Gaussian reduction conditions for an indefinite form now requires (stabled geometrically), that the representing line of the form $(a,b,c)$ intersect (in the interior of the ellipse) the two elementary chords heavily accentuated in figure 170, namely, first the chord of the first kind which joins the points $\zeta=0$ and $\zeta=\infty$, then the elementary chord of the second kind given by $a=c$, which joins the points $\zeta=\pm 1$. To this, furthermore, is added one more condition regarding the “direction of the arrow” of the representing lines; we are able, in connection to the positioning in figure 170, to express this condition thus: The direction of the arrow of the representing line is to pass over the elementary chord $a=c$ in the direction from below to above. The analytic expression of this condition is:
\begin{equation}
ac<0,\; \abs{\dfrac{-b+\sqrt{D}}{c}}<1,\;\abs{\dfrac{-b-\sqrt{D}}{c}}>1,
\end{equation}
where in the lost two inequalities, on the left sides are meant the absolute values of the numbers \textcolor{red}{here} in vertical strokes; and these are just the conditions, which Gauss sets up in Article 133 of the “Disquisitions \textcolor{red}{here}”\footnote{*}. For the further development of the geometric theory of the indefinite forms upon this foundation, we refer to the work of named, as well as to the (hand written) lectures of Klein \textcolor{red}{here}\footnote{**}.
\section{the projective form of the Picard group}

In order to also rewrite the theory of the Dirichlet and Hermitian forms in projective-geometric form, we must first present a short study concerning the projective form of the Picard group; we relate this, at the same time, to the Picard group consisting of substitutions of the determinants 1 and \ii.

In order to conceive the substitutions $\left(\begin{array}{cc}\alpha&\beta\\\gamma&\delta\\\end{array}\right)$ of this group as “motions of hyperbolic space”,we could, by means of (5) resp. (6) on p.46, go over from $\zeta$ to the homogenous coordinantes $y_1,y_2,y_3,y_4$ of this space, in order than to let the individual $\zeta$-substitution correspond to the quaternary y-substitution of the determinant 1 given by (10) on p.47. However, the coefficients of the y-substitution would still come at partially complex.

In order to obtain a real quaternary substitution, we could utilize the coordinante system of the $z_i$ defined in (3) on p.46. However, it is more to the purpose at present, to define coordinates $x_i$ from the $y_i$ as follows:
\begin{align}
    y_1=x_1, y_2=x_2+\ii x_3, y_3=x_2-\ii x_3, y_4=x_4   
\end{align}
The individual substitution $\zeta'=\dfrac{\alpha\zeta+\beta}{\gamma\zeta+\delta}$ of the Picard group now furnishes a real unimodular $x_i$-substitution with the following sixteen coefficients:
\begin{align}
    \begin{array}{cccc}
    \alpha\bar{\alpha}&\alpha\bar{\beta}+\beta\bar{\alpha}&\ii(\alpha\bar{\beta}-\beta\bar{\alpha})&\beta\bar{\beta}\\
    \dfrac{1}{2}(\alpha\bar{\gamma}+\gamma\bar{\alpha})&\dfrac{1}{2}(\alpha\bar{\delta}+\delta\bar{\alpha}+\beta\bar{\gamma}+\gamma\bar{\beta})&\dfrac{\ii}{2}(\alpha\bar{\delta}-\delta\bar{\alpha}+\gamma\bar{\beta}-\beta\bar{\gamma})&\dfrac{1}{2}(\beta\bar{\delta}+\delta\bar{\beta})\\
    \dfrac{1}{\alpha\ii}(\alpha\bar{\gamma}-\gamma\bar{\alpha})&\dfrac{1}{\alpha\ii}(\alpha\bar{\delta}-\delta\bar{\alpha}+\beta\bar{\gamma}-\gamma\bar{\beta})&\dfrac{1}{2}(\alpha\bar{\delta}+\delta\bar{\alpha}-\beta\bar{\gamma}-\bar{\beta}\gamma)&\dfrac{1}{\alpha\ii}(\beta\bar{\delta}-\delta\bar{\beta})\\
    \gamma\bar{\gamma}&(\gamma\bar{\delta}+\delta\bar{\gamma})&\ii(\gamma\bar{\delta}-\delta\bar{\gamma})&\delta\bar{\delta}\\   
    \end{array}
\end{align}
The absolute sphere of the hyperbolic space, however, assumes the form:
\begin{align}
    x_2^2+x_3^2-x_1x_4=0.
\end{align}
the “quaternary quadratic form” standing on the left side here, in consequence of p.47, is transformed into itself by the individual $x_i$-substitution,

Notice now, that $\alpha,\beta,\gamma,\delta$ are here whole numbers of the quadratic number field denoted above by $\Omega$. One concludes immediately from this, that the sixteen coefficients of the individual $x_i$-substitution become rational whole numbers. However, the following theorem holds directly: the Picard group of the first kind with substitutions of the determinants 1 and \ii, in its new projective form, is straight quaternary form $(x_2^2+x_3^2-x_1x_4)$ into itself. This latter group we designate briefly as the “reproducing group of the quaternary form $(x_2^2+x_3^2-x_1x_4)$”. It will be the task of the next chapter to investigate more closely, in general, the reproducing groups of indefinite ternary and quaternary forms.

In order to prove the theorem set up, we call $\Gamma$ the reproducing unimodular group of $(x_2^2+x_3^2-x_1x_4)$, while $\Gamma_0$ is the abovementioned Picard group in projective form. $\Gamma_0$ is contained as a subgroup in $\Gamma$, and indeed, one such of a certain finite index $\mu$; for one easily recognizes in $\Gamma$ a properly discontinuous group ( as is discussed even more closely in the next chapter), and the double tetrahedron of  $\Gamma_0$ defined by (6) on p.35. \textcolor{red}{here}, in any case, only be divided into finitely many (belonging to  $\Gamma$) polyhedral of non-vanishing spatial content.

Exactly $\mu$ polyhedral of  $\Gamma$ will now completelyt fill up thje double tetrahedron of the Picard group just mentioned. Since, however, the double tetrahedron has one parabolic cusp situated at $\zeta=\infty$, then the $\mu$ polyhedral of  $\Gamma$ just considered will each reach to $\zeta=\infty$ with a parabolic cusp.

On the strength of this, we separate out, inside $\Gamma$ and $\Gamma_0$, those parabolic rotation subgroups $G$ and $G_0$, which belong to the centrum $\zeta=\infty$. According to what has just been said, $G_0$ will then be a subgroup of index $\mu$ inside $G$. Put the group $G_0$ is known here from p.31; it contains, if we again, at the same time, write it as a group of $\zeta$-substitutions, all substitutions:
\begin{align}
    \zeta'=\pm\zeta+a+\ii b,\zeta'=\pm\ii\zeta+a+\ii b
\end{align}
when $a,b$ are to run through all pairs of rational whole numbers. Let us now, say, not the same time, also set up the $\zeta$-substitutions, which belong to $G$, in their totality.

Since $\zeta=\infty$ is transformed into itself by the substitutions segment, then, in any case, $\gamma$ must be =0 for them. The three remaining coefficients $\alpha,\beta,\delta$ are then to be determined in the most general manner, that the sixteen coefficients of the quaternary schema given above become rational whole numbers of determinant 1.

As we now notice, at the same time, the individual $\zeta$-substitution remains essentially unaltered, the corresponding $x_i$-substitution even formally so, in case we provide $\alpha,\beta,\delta$ with a common factor of absolute value 1. We shall uniquely dispose of such a factor by the determination, that $\delta$, say, is to be real and positive.

For abbreviation, we now call the integral coefficients of the $x_i$-substitution $a_{ik}$. The determinant $|a_{ik}|$, on account of 8=0, becomes equal to $(\alpha\bar{\alpha},\delta\bar{\delta})^2$; and since this is to be equal to 1, there follows further:
\begin{align}
    \alpha\bar{\alpha}\cdot\delta\bar{\delta}=a_{11}\cdot a_{44}=1, a_{11}=1, a_{44}=1.
\end{align}
In consequence of the determination mode concerning $S$, and thus has $\delta=1$ and $\alpha=\pm1$ or $\pm\ii$.

Further, $2\bar{\alpha}\beta=a_{12}+\ii a_{13}$ follows, so that $\alpha\beta=m+\ii n$, understanding by $m$ and $n$ rational whole numbers. The latter satisfy the condition $4a_{14}=m^2+n^2$ and are therefore necessarily both even; we thus have $\beta=a+\ii b$, where a and b are rational whole numbers.

As one sees, we are thus led back exactly to the substitutions (3) and to no others. The groups $G_0$ and $G$ are consequently identical, i.e., one has $\mu=1$. With this, $\Gamma_0$ and $\Gamma$ are also exactly equal, and our assertion above concerning the projective definition of the Picard group is proved.——
The extension of the Picard group to a group of the second kind, we could achieve at that time by addition of the reflection $\zeta'=-\bar{\zeta}$. Since, in consequence of (4) on p.45, as well as of the equations (1) of the present paragraph:
$$\zeta=\frac{y_2}{y_4}=\frac{x_2+\ii x_3}{x_4}$$
holds. The reflection just given corresponds to the integral quaternary substitution:
$$x_1'=x1, x_2'=-x_2, x_3'=x_3, x_4'=x_4$$
of the determinant -1, which likewise transforms the form $(x_2^2+x_3^2-x_1x_4)$ into itself. Form this, the corollary easily yields itself: The Picard group of the second kind under consideration \textcolor{red}{here}, in its projective form. The group of all real integral substitutions of the determinants $+1$ or $-1$, which transforms the quaternary form $(x_2^2+x_3^2-x_1x_4)$ into itself.
\section{Theory of the Hermitian and Dirichlet forms in projective-geometric form}

As, in the last paragraph, we could \textcolor{red}{here} the theory of the Gaussian forms on the basis of the “rectilinear modular figure”, so it is now correspondingly possible to base the theory of the Hermitian and Dirichlet forms, on the projective form of the Picard group. According to the representations of the previous paragraphs, it thereby appears indicated, to \textcolor{red}{here} equivalence simultaneously in the narrower and in the extended sense (cf. p.453). Then the division of the interior of the sphere of the hyperbolic space into plane-surfaced tetrahedral resp. double tetrahedral becomes the proper foundation of the theory.

Hereby, naturally, it is again a question only of a more external metamorphosis of our theory laid out above. But this metamorphosis is remarkable, since the geometric interpretation of the forms in the hyperbolic space fashions itself even more simply and naturally than in the $\zeta$-space.

In fact, we had first interpreted a definite Hermitian form $(a,b_1,b_2,c)$ by that point of trhe $\zeta$-halfspace, which had the coordinates:\textcolor{red}{text}
$$\zeta=\frac{-b_1}{a}, n=\frac{b_2}{a}, \mathcal{G}=\pm \frac{\sqrt{-D}}{a}.$$
Now that mapping between the $\zeta$-halfspace and the interior of the sphere of the hyperbolic space is given by:
$$\zeta=\frac{x_2+\ii x_3}{x_4},=\frac{\sqrt{x_1x_4-x_2^2-x_3^2}}{x_4}.$$
We accordingly arrive at the following conception: We shall interpret a definite Hermitian form $(a,b_1,b_2,c)$ geometrically by that point situated in the interior of the sphere of the hyperbolic space, whose coordinates are the following:
\begin{align}
x_1=c,x_2=-b_1,x_3=b_2,x_4=a.
\end{align}

It is now a great simplification (compared to the $\zeta$-space), that the interpretation hereby given also carries itself over immediately to the indefinite forms too. In fact, we shall also interpret an indefinite Hermitian form $(a,b_1,b_2,c)$ by the point of the hyperbolic space, whose coordinates are given by (1); then, however, this representing point lies outside the absolute space.

By grouping both cases together, one can consequently state the theorem, that the Hermitian forms corresponding to all the “rational points” of the hyperb0olic space, filling it out everywhere densely. Thereby, the points outside the absolute sphere furnish the indefinite forms, those inside the sphere the definite forms, while the rational situated on the spherical surface itself given the Hermitian forms of vanishing determinant (not \textcolor{red}{here} above).
In carrying through the equivalence theory on the basis of the preceding interpretation of the forms, the circumstance is again decisive, that our group is indeed properly discontinuous inside, but not outside the absolute surface of the projective space. Analogously, as for the Gaussian forms, we shall rep;ace the representing point of an indefinite form by its polar plane with respect to the absolute sphere, pushing through the interior of the sphere: 
\begin{align}
ax_1+\alpha b_1x_2-2b_2x_3+cx_4=0.
\end{align}
This polar plane then corresponds exactly to the representing hemisphere of \textcolor{red}{here}, as we utilized it above in the $\zeta$-plane.——

Also, the interpretation of the Dirichlet forms attaches itself here effortlessly: The individual Dirichlet form $(a,b,c)$ is interpreted by that rectilinear secant of the absolute sphere, which joins the two “zeroes” of the form situated on the surface of the sphere. In contrast to the Hermitian forms, however, it is not always a question here of “rational” lines of the hyperbolic space. Namely, if we set $p_{ik}=x_ix_k'-x_i'x_k$ for the introduction of line coordinantes, then the Dirichlet form $(a,b,c)$ of the determinant D is given by the line with the following coordinates:
\begin{align}
\left\{\begin{array}{rclrcl}
p_{12}&=& c\sqrt{\bar{D}}+\bar{c}\sqrt{D},& p_{13}&=& -\ii\left(c\sqrt{\bar{D}}-\bar{c}\sqrt{D}\right),\\
p_{14}&=& -2\left(b\sqrt{\bar{D}}+\bar{b}\sqrt{D}\right),& p_{23}&=& \ii\left(b\sqrt{\bar{D}}-\bar{b}\sqrt{D}\right),\\
p_{24}&=& a\sqrt{\bar{D}}+\bar{a}\sqrt{\bar{D}},& p_{34}&=& \ii\left(a\sqrt{\bar{D}}+\bar{a}\sqrt{\bar{D}}\right)\\
\end{array}\right.
\end{align}
the \textcolor{red}{ratios} of these coordinates, as one sees, always still contain the sphere toot $\sqrt{D\bar{D}}$, which is irrational in general.

Furthermore, in connection with this, one can state the special theorem, that the hyperbolic axes of the Picard group are, in only case, always “rational” lines of the projective space. Namely, a hyperbolic substitution furnishes, in the sense on p.465, the Dirichlet form $(\alpha\gamma,\delta-\alpha,-\alpha\beta)$, whose determinant $D=(\alpha+\delta)^2-4$. Now since, for that hyperbolic substitution, $(\alpha+\delta)$ is real and absolute \textcolor{red}{here}, then one has a real positive D. Here, therefore, $\sqrt{D\bar{D}}$ is a rational whole number, from which our assertion emerges.

From this, we may now easily succeed in proving a supplementary theorem, concerning the indefinite Hermitian forms: Not only to each such form does there belong a hyperbolic rotation subgroup inside the Picard group (as proved above), but also, conversely, “each” such subgroup is the reproducing group of a determinate indefinite Hermitian form associated to it, so that, with the theory of the latter, that of those subgroups is also exhausted at the same time. In the polar plane of the \textcolor{red}{here} of an arbitrarily chosen hyperbolic rotation subgroup of the Picard group, namely, always lie infinitely many hyperbolic axes, i.e. , rational lines. Two of these lines suffice, in order to recognize, in the polar plane in question, a “rational” plane; to it, consequently, actually belongs a Hermitian form.
To the study of the indefinite Hemitian forms, finally, the following important argument also attaches itself.

If $(a,b_1,b_2,c)$ is an arbitrary such form, then the representing “plane” of this is given by (2). The greatest common factor of \textcolor{red}{here}, we imagine removed, whereby, let the equation reduce itself to:
\begin{align}
a_{41}x_1+a_{42}x_2+a_{42}x_3+a_{44}x_4=0.
\end{align}
In consequence of simple considerations, one can then forma quaternary integral unimodular substitution:
\begin{align}
\tau_i=a_{i1}x_1+a_{i2}x_2+a_{i3}x_3+a_{i4}x_4, i=1,2,3,4\ldots
\end{align}
whereby, for $i=4$, the coefficients are exactly the whole numbers coming up on (4).

By means of the transformation (5), the form $(x_2^2+x_3^2-x_1x_4)$ goes over into a new integral \textcolor{red}{here} quadratic form $F(\tau_1,\tau_2,\tau_3,\tau_4)$; the operations of the (projective) Picard group, however, furnish integral unimodular $\tau_i$-substitutions, which will clearly form the “reproducing group” of the form $F(\tau_i)$.

In particular, the substitutions of the reproducing group belonging to $(a,b_1,b_2,c)$ will furnish $\tau_i$-substituions, for which the plane $\tau_i=0$ goes over into itself, and which accordingly were the form:
\begin{align}
\left\{\begin{array}{rcl}
\tau_i'&=&\alpha_{i1}\tau_1+\alpha_{i2}\tau_2+\alpha_{i3}\tau_3+\alpha_{i4}\tau_4, i=1,2,3\ldots,\\
\tau_4'&=&\alpha_{44}\tau_4.
\end{array}\right.
\end{align}

The whole number $\alpha{44}$, as divisor of the determinant of the substitution, i.e., as a divisor of 1, is equal to $\pm1$, and one therefore has:
\begin{align}
\left|a_{ik}\right|=\pm1, i,k=1,2,3\ldots.
\end{align}

If one now sets $\tau_4=0$, then $F(\tau_i)$ goes over in to an integral ternary form \textcolor{red}{here} “representable by the quaternary form $(x_2^2+x_3^2-x_1x_4)$” . The substitution of $\tau_4=0$, with this, $\tau_4'=0$, in the equation $F(\tau_i')=F(\tau_i)$, however, furnishes the result, that the ternary group:
\begin{align}
\tau_i'=\alpha_{i1}\tau_1+\alpha_{i2}\tau_2+\alpha_{i3}\tau_3, i=1,2,3\ldots,
\end{align}
to the obtained in the \textcolor{red}{here} manner from the reproducing group of $(a,b_1,b_2,c)$, represents either the entire reproducing group of the ternary form $f(\tau_1,\tau_2,\tau_3)$ or a subgroup contained in this group.


The theory of the Hermitian forms has led us in such a way to those groups to whose mo0re detailed investigation the following chapter is to be dedicated. Thereby, we also come back to the two examples considered above with $D=5$ and $D=7$; both groups, corresponding to our theorem, will be subgroups inside the reproducing groups of two particular ternary forms.



\end{document}
