\section{the projective form of the Picard group}

In order to also rewrite the theory of the Dirichlet and Hermitian forms in projective-geometric form, we must first present a short study concerning the projective form of the Picard group; we relate this, at the same time, to the Picard group consisting of substitutions of the determinants 1 and \ii.

In order to conceive the substitutions $\left(\begin{array}{cc}\alpha&\beta\\\gamma&\delta\\\end{array}\right)$ of this group as “motions of hyperbolic space”,we could, by means of (5) resp. (6) on p.46, go over from $\zeta$ to the homogenous coordinantes $y_1,y_2,y_3,y_4$ of this space, in order than to let the individual $\zeta$-substitution correspond to the quaternary y-substitution of the determinant 1 given by (10) on p.47. However, the coefficients of the y-substitution would still come at partially complex.

In order to obtain a real quaternary substitution, we could utilize the coordinante system of the $z_i$ defined in (3) on p.46. However, it is more to the purpose at present, to define coordinates $x_i$ from the $y_i$ as follows:
\begin{align}
    y_1=x_1, y_2=x_2+\ii x_3, y_3=x_2-\ii x_3, y_4=x_4   
\end{align}
The individual substitution $\zeta'=\dfrac{\alpha\zeta+\beta}{\gamma\zeta+\delta}$ of the Picard group now furnishes a real unimodular $x_i$-substitution with the following sixteen coefficients:
\begin{align}
    \begin{array}{cccc}
    \alpha\bar{\alpha}&\alpha\bar{\beta}+\beta\bar{\alpha}&\ii(\alpha\bar{\beta}-\beta\bar{\alpha})&\beta\bar{\beta}\\
    \dfrac{1}{2}(\alpha\bar{\gamma}+\gamma\bar{\alpha})&\dfrac{1}{2}(\alpha\bar{\delta}+\delta\bar{\alpha}+\beta\bar{\gamma}+\gamma\bar{\beta})&\dfrac{\ii}{2}(\alpha\bar{\delta}-\delta\bar{\alpha}+\gamma\bar{\beta}-\beta\bar{\gamma})&\dfrac{1}{2}(\beta\bar{\delta}+\delta\bar{\beta})\\
    \dfrac{1}{\alpha\ii}(\alpha\bar{\gamma}-\gamma\bar{\alpha})&\dfrac{1}{\alpha\ii}(\alpha\bar{\delta}-\delta\bar{\alpha}+\beta\bar{\gamma}-\gamma\bar{\beta})&\dfrac{1}{2}(\alpha\bar{\delta}+\delta\bar{\alpha}-\beta\bar{\gamma}-\bar{\beta}\gamma)&\dfrac{1}{\alpha\ii}(\beta\bar{\delta}-\delta\bar{\beta})\\
    \gamma\bar{\gamma}&(\gamma\bar{\delta}+\delta\bar{\gamma})&\ii(\gamma\bar{\delta}-\delta\bar{\gamma})&\delta\bar{\delta}\\   
    \end{array}
\end{align}
The absolute sphere of the hyperbolic space, however, assumes the form:
\begin{align}
    x_2^2+x_3^2-x_1x_4=0.
\end{align}
the “quaternary quadratic form” standing on the left side here, in consequence of p.47, is transformed into itself by the individual $x_i$-substitution,

Notice now, that $\alpha,\beta,\gamma,\delta$ are here whole numbers of the quadratic number field denoted above by $\Omega$. One concludes immediately from this, that the sixteen coefficients of the individual $x_i$-substitution become rational whole numbers. However, the following theorem holds directly: the Picard group of the first kind with substitutions of the determinants 1 and \ii, in its new projective form, is straight quaternary form $(x_2^2+x_3^2-x_1x_4)$ into itself. This latter group we designate briefly as the “reproducing group of the quaternary form $(x_2^2+x_3^2-x_1x_4)$”. It will be the task of the next chapter to investigate more closely, in general, the reproducing groups of indefinite ternary and quaternary forms.

In order to prove the theorem set up, we call $\Gamma$ the reproducing unimodular group of $(x_2^2+x_3^2-x_1x_4)$, while $\Gamma_0$ is the abovementioned Picard group in projective form. $\Gamma_0$ is contained as a subgroup in $\Gamma$, and indeed, one such of a certain finite index $\mu$; for one easily recognizes in $\Gamma$ a properly discontinuous group ( as is discussed even more closely in the next chapter), and the double tetrahedron of  $\Gamma_0$ defined by (6) on p.35. \textcolor{red}{here}, in any case, only be divided into finitely many (belonging to  $\Gamma$) polyhedral of non-vanishing spatial content.

Exactly $\mu$ polyhedral of  $\Gamma$ will now completelyt fill up thje double tetrahedron of the Picard group just mentioned. Since, however, the double tetrahedron has one parabolic cusp situated at $\zeta=\infty$, then the $\mu$ polyhedral of  $\Gamma$ just considered will each reach to $\zeta=\infty$ with a parabolic cusp.

On the strength of this, we separate out, inside $\Gamma$ and $\Gamma_0$, those parabolic rotation subgroups $G$ and $G_0$, which belong to the centrum $\zeta=\infty$. According to what has just been said, $G_0$ will then be a subgroup of index $\mu$ inside $G$. Put the group $G_0$ is known here from p.31; it contains, if we again, at the same time, write it as a group of $\zeta$-substitutions, all substitutions:
\begin{align}
    \zeta'=\pm\zeta+a+\ii b,\zeta'=\pm\ii\zeta+a+\ii b
\end{align}
when $a,b$ are to run through all pairs of rational whole numbers. Let us now, say, not the same time, also set up the $\zeta$-substitutions, which belong to $G$, in their totality.

Since $\zeta=\infty$ is transformed into itself by the substitutions segment, then, in any case, $\gamma$ must be =0 for them. The three remaining coefficients $\alpha,\beta,\delta$ are then to be determined in the most general manner, that the sixteen coefficients of the quaternary schema given above become rational whole numbers of determinant 1.

As we now notice, at the same time, the individual $\zeta$-substitution remains essentially unaltered, the corresponding $x_i$-substitution even formally so, in case we provide $\alpha,\beta,\delta$ with a common factor of absolute value 1. We shall uniquely dispose of such a factor by the determination, that $\delta$, say, is to be real and positive.

For abbreviation, we now call the integral coefficients of the $x_i$-substitution $a_{ik}$. The determinant $|a_{ik}|$, on account of 8=0, becomes equal to $(\alpha\bar{\alpha},\delta\bar{\delta})^2$; and since this is to be equal to 1, there follows further:
\begin{align}
    \alpha\bar{\alpha}\cdot\delta\bar{\delta}=a_{11}\cdot a_{44}=1, a_{11}=1, a_{44}=1.
\end{align}
In consequence of the determination mode concerning $S$, and thus has $\delta=1$ and $\alpha=\pm1$ or $\pm\ii$.

Further, $2\bar{\alpha}\beta=a_{12}+\ii a_{13}$ follows, so that $\alpha\beta=m+\ii n$, understanding by $m$ and $n$ rational whole numbers. The latter satisfy the condition $4a_{14}=m^2+n^2$ and are therefore necessarily both even; we thus have $\beta=a+\ii b$, where a and b are rational whole numbers.

As one sees, we are thus led back exactly to the substitutions (3) and to no others. The groups $G_0$ and $G$ are consequently identical, i.e., one has $\mu=1$. With this, $\Gamma_0$ and $\Gamma$ are also exactly equal, and our assertion above concerning the projective definition of the Picard group is proved.——
The extension of the Picard group to a group of the second kind, we could achieve at that time by addition of the reflection $\zeta'=-\bar{\zeta}$. Since, in consequence of (4) on p.45, as well as of the equations (1) of the present paragraph:
$$\zeta=\frac{y_2}{y_4}=\frac{x_2+\ii x_3}{x_4}$$
holds. The reflection just given corresponds to the integral quaternary substitution:
$$x_1'=x1, x_2'=-x_2, x_3'=x_3, x_4'=x_4$$
of the determinant -1, which likewise transforms the form $(x_2^2+x_3^2-x_1x_4)$ into itself. Form this, the corollary easily yields itself: The Picard group of the second kind under consideration \textcolor{red}{here}, in its projective form. The group of all real integral substitutions of the determinants $+1$ or $-1$, which transforms the quaternary form $(x_2^2+x_3^2-x_1x_4)$ into itself.