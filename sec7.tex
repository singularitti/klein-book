\section{Reduction theory of the indefinite Hermitian forms}

We interpreted on indefinite Hermitian form $(a,b_1,b_2,c)$ where (p.455 ff.) by the hemisphere:
\begin{align}\label{eq:71}
    a(\xi^2+\eta^2+\vartheta)+2b_1\xi-2b_2\eta+c=0.
\end{align}
standing perpendicularly to the \zeta-plane and provided with an appropriate arrow.

In the definition of the “reduced” form, we proceed here analogously, as above: The indefinite Hermitian form $(a,b_1,b_2,c)$ is called “reduced”, in case its representing hemisphere has a finitely extended surface portion in common with the initial space of the unimodular Picard group. The representing hemisphere is thus either one of those facial surfaces of the double pentahedron(1) on p.457, which latter is to be counted, or it intersects this pentahedron in a surface portion of non-vanishing content.

Since, because of $D>0$, it is a question in (1) of a sphere of non-vanishing radius, then one immediately concludes: Each indefinite form $(a,b_1,b_2,c)$ is transformable into a reduced form equivalent to it.

In carving out the reduction theory further, it is often expedient, to treat  by themselves those reduced forms, whose representing hemisphere are facial surfaces of the initial space. There are only two inequivalent hemispheres(halfplanes) of this kind; and we are able to the halfplane $\eta=0$ and the hemisphere of radius $\mathrm{1}$ about $\zeta=0$ as such. The two associated “primitive” forms are $(0,0,1,0)$ and $(1,0,0,-1)$; and self-evidently, all “inprimitive” forms also belong here, which proceed from those two by addition of factors. The determinant of the primitive forms is $D=1$ in both cases.

Now apart form the cases just mentioned, a proper penetration of the initial space is always present for the representing hemisphere of a reduced form $(a,b_1,b_2,c)$. But in order, on the strength of this, to formulate the reduction conditions, arithmetically, one has to distinguish, whether the first coefficient a vanishes or not.

If $a=0$, then \ref{eq:71} represents a plane. In order that this penetrate the initial space, among the four non-parabolic vertices of this domain, two must let themselves be given, which are \textcolor{red}{situable} on different sides of the plane \ref{eq:71} and not on it. If we bring in the coordinates of more vertices, then follows: An indefinite form $(a,b_1,b_2,c)$ with $a=0$ is always and only then reduced, if among the far whole numbers:
\begin{align}\label{eq:72}
    b_1+c, -b_1+c, b_1-b_2+c, -b_1-b_2+c,
\end{align}
at least one is $<0$, and, at the same time, at least one is $>0$.

If, however, $a\gtrless0$, then one has a proper hemisphere in \ref{eq:71}. If this is to penetrate the initial space, then, of the four vertices of the initial space just utilized, at least one must lie in the interior of the hemisphere. There consequently yields itself: An indefinite Hermitian from $(a,b_1,b_2,c)$ with $a\gtrless 0$ is always and only reduced, if, of the four whole numbers:
\begin{align}\label{eq:73}
    a^2\pm ab_1+ac, a^2\pm ab_1-ab_2+ac
\end{align}
at least one is $<0$.

The arithmetic formulation of the reduction conditions play the same role here again, as far the forms considered before. We are able, by means of those conditions, to show, that for given determinant $D$, the number of reduced indefinite forms, and consequently, also the class number, is a bounded one.

If, namely, $a=0$, then an account of $D=b_1^2+b_2^2$, we have, for given $D$, a bounded number pairs $b_1, b_2$. For the individual pair $b_1, b_2$, then, $c$ is included within finite bounds, since, among the expressions \ref{eq:72}, in each case, one must be $>0$ and one $<0$.

If, on the other hand, $a\gtrless 0$ and one of the first two numbers \ref{eq:73} is negative, then one replaces as by $(b_1^2+b_2^2-D)$ and has, for at least one of the two signs in question:
\begin{align}
    a^2\pm ab_1+b_1^2+b_2^2<D.
\end{align}
One easily transforms this inequality into the form:
\begin{align}
    (2a\pm b_1)^2+3b_1^2+4b_2^2<4D,
\end{align}
and recognized form this, that for given $D$, the whole numbers $a, b_1, b_2$ are only able to be chosen in finitely many ways. But, with $D, a, b_1, b_2, c$, is uniquely determined. If, finally, one of the last two numbers \ref{eq:73} is negative, then one of the inequalities holds:
\begin{align}
    (2a-b_2\pm b_1)^2+(b_1\pm b_2)^2+3b_1^2+2b_2^2<4D.
\end{align}
the discussion of this again leads to the desired result, we are now also able to develop an “algorithm of continuous reduction” for the indefinite Hermitian forms.

If here too, for the sake of more fluent language, the forms belonging to the facial surfaces of the initial space remain initially excluded, then one representing semisphere of an individual indefinite form $(a, b_1, b_2, c)$ will intersect infinitely many double pentahedra of the halfspace division. The surface of the hemisphere, for its part, thereby appears covered by a net of infinitely many circular-arc polygons with three, four or five sides, whereby these circular-arc polygons press together infinite number toward each point of the boundary of the hemisphere substituted in the $\zeta$-plane.

Each of these circular-arc polygons furnishes, by means of its transfer backs into the initial space, a substitution, which transforms the proposed form into a reduced form equivalent with it; and we arrive in this way, at the same time, at all reduced forms of the proposed class. But we recognize here, above all, a “netlike connection” of all these reduced forms; for one such is immediately given by the character of the positioning of the circular-arc polygons onto the representing hemisphere of the form $(a,b_1,b_2,c)$.

The process of continuous reduction will now consist in our now actually producing the connection of the reduced forms thus recognized resp. their spherical segments substituted in the initial space. One will obviously continues the individual such spherical surface-portion into the three resp. four or five neighboring double pentahedra, and transfer \textcolor{red}{here} into the initial space the spherical segments to the net here. In this way, one obtains, as “neighboring” worth that first reduced form, three resp. four or five particular additional reduced forms, and, by continuation of this operation, can arrive at each reduced form of the class.

The substitutions to be applied in this process, in general, \textcolor{red}{here}, only the five generators of  of the unimodular Picard group named in \ref{eq:73} on p.461. only in case the hemisphere of an individual reduced form runs through one of the tight edges of the initial space, will one of the following eight substitutions have to be exercised for the calculation of the appropriate neighboring form:
\begin{align}
    \left\{\begin{array}{ccc}
    ST^{\pm 1}\left(\begin{array}{cc}
    \ii & \mp \ii\\
    0 & -\ii\\
    \end{array}\right) & VT^{\pm 1}=\left(\begin{array}{cc}
    \ii & 1\mp \ii\\
    0 & -\ii\\
    \end{array}\right) & SU=\left(\begin{array}{cc}
    0 & \ii\\
    \ii & 0\\
    \end{array}\right)\\
    T^{\pm 1}UT^{\pm 1}=\left(\begin{array}{cc}
    1 & 0\\
    \mp 1 & 1\\
    \end{array}\right) & VUV=\left(\begin{array}{cc}
    \ii & 0\\
    1 & -\ii\\
    \end{array}\right) & 
    \end{array}\right.
\end{align}
these eight are among the twelve substitutions already mentioned at a corner \textcolor{red}{here} place for the Dirichlet forms(p.462).

We now combine the perceptions thus obtained with the theorem of the finiteness of the number of reduced forms for given D. There immediately arises the important theorem: the process of continuous reduction always ends, from a first reduced form, the only “finitely many” further reduced forms of the same class. There thus yields itself, for each class of indefinite Hermitian forms, a determinate, completely \textcolor{red}{here} “net of reduced forms”, which obviously represents the analogy of the “period of reduced Dirichlet forms”.

One now recognizes without difficulty, that the results obtained also remain in force for those forms, which beong to the facial surfaces of the initial space. There arises here too, without further \textcolor{red}{ado}, a division of the individual hemisphere into circular-arc triangle resp. quadrangles, which have the same signature or the continuous reduction of the forms now intended, as we just become acquainted with them in general. Thereby, one has only, in each case, to make use of the precise prescriptions, which decide the coordination of the facial surfaces of the individual double pentahedron.---

The question of the equivalence of two indefinite Hermitian forms of the same determinant now disposes of itself in known fashion, Form each of the two given forms, say by mediation of the halfspace division, one must go over to a reduced form, and form here, by the process of continuous reduction, produce the entire net of reduced forms. The given forms are always and only then equivalent, if these nets are identical.

