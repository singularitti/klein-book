\section{The Gaussian forms and the modular group}
We write the integral binary quadratic forms (by relation of the terminology utilized in "M" I):
\begin{align}\label{eq:11} %第一章第一个公式
ax^2+2bxy+cy^2
\end{align}
and use, as abbreviation for the individual such form, by symbol $(a,b,c)$ .The coefficients $(a,b,c)$ are rational intergral numbers, and $D=b^{2}-ac$ is the determinant of the form $(a,b,c)$. Let it be allowed, to designate the form \ref{eq:11}  as "\textit{Gaussian forms}"; for even if very valuable fundamental theorems concerning these were discarded by Lagrange, a complete theory of the forms in question is first contained in the \textcolor{red}{there}
The foundations of this theory now permit an especially brief and transparent representation, if one avails oneself of the geometric tools, which the modular group and the discussion of the $\zeta$-halfplane into circular-arc triangles associated to it given ready at hand. This geometric treatment of the Gaussian forms is given in "M" I on p.243.; and here are found a few, not unimportant expressions concerning the ambiguous forms as well as the sefl-inverse indefinite forms, in "M" II on p.161 ff.
In order to briefly remember to principal points of view of this subject, above all, the individual form(I) received a geometric interpretation.
For the definite forms, i.e., for those with $D<0$, one may restrict oneself to the so-called “positive” forms, for which $a$ and $c$ are positive numbers. A positive form, however, is interpreted geometrically by those points belonging to the positive $\zeta$-halfplane, which one obtains by solution of the quadratic equation:
\begin{align}\label{eq:12}
a\zeta^2+2b\zeta+c=0
\end{align}
one shows immediately, that by the giving of the representing point and the numerical solve $D$, the positive form $a,b,c$ is uniquely determined.
For an indefinite form, i.e., for one such with $D<0$, one also lets the restriction enter, that $D$ is different from a square. For the geometrical interpretation of the indefinite form, one marks the two points on the $\zeta$-axis, which now correspond to real and distinct roots of the equation(I). One then represents the form by that semicircle of the positive halfplane orthogonal to be the $\zeta$-axis, which has the two planes just marked as feet. The circle was also to be provided with a certain directional arrow, in order that the two forms $(a,b,c)$, $(-a,-b,-c)$, inverse to one another, could be separated from one another. The circle thus outfilled, in conjunction with the numerical solve $D$, furnished the form uniquely.
Now it was the theory of the equivalence and reduction of the Gaussian forms which, on the basis of the triangle division of the $\zeta$-halfpane, assumed a surveyable form by means of geometrical interpretation of the forms. For equivalent forms, the two representing points resp. semicircles are equivalent to the modular group of the “first” kind; a form is called reduced, in case its representing point belongs to the initial space (i.e., to the discontinuity domain continually formed in "M" I) of the group mentioned resp., its representing semicircle cuts through this domain. From this, there yielded itself, as reduction conditions for the definite positive forms:
\begin{align}\label{eq:13}
a\zeta^2+2b\zeta+c=0
\end{align}
with the condition, that for $a=c$, in more detail:
\begin{align}\label{eq:14}
b\geq 0
\end{align}
is to hold; for the reduced indefinite forms, besides this, there is the condition, that:
\begin{align}
a(a\pm b+c)<0
\end{align}
is to hold either for one or for both sighs. Yet it is to be noticed, that this latter inequality wears off from the Gaussian reduction condition for indefinite forms; we come back to this at the end of the chapter.
From the reduction conditions followed, moreover, by the arithmetic consideration, the finiteness od the number of reduced forms and with this, the finiteness of the class number for a given determinant $D$.
For the goal of our present representation, the theory of the indefinite forms is more important than that of the definite arcs. On the representing semicircle of an indefinite form, the double triangle of the modular division cuts off the infinitely termed chain of segments. We then obtain, in each case, a reduced form equivalent with the proposed form, if we transfer any one of those segments into the initial space and exercise thee corresponding transformation on the form. On account of the finiteness of the number or reduced forms, however, we obtain in this way any finitely many segments in the initial space(cf. “M" I on p.257 ff.); and from this the fact yielded itself, that an indefinite Gaussian form can always be transformed into itself by infinitely many substitutions. These substitutions form a cyclic hyperbolic group, which has the representing semicircle of the form as orbit curve.
This group could be extended inside the modular group in two particular cases, namely, first to a hyperbolic dihedral group of the first kind(cf. p346), in case $(a,b,c)$ is equivalent with its inverse form $(-a,-b,-c)$, secondly, to a(?) cyclic group of the second kind , in case $(a,b,c)$ is ambiguous (cf. "M" II p. 161)
The chain of the segments in the initial space, on which we laid at the representing semicircle of an indefinite form, furnished the associated “period of reduced forms”. These latter forms thereby appear arranged into a closed chain; and, in this chain, the transition from one form to a neighboring one, the so-called ?process of continuous reduction(?)?, effects itself in this specific situation by exercise of one of the two substitutions $\left(\begin{array}{cc}1 & 1\\0 & 1\end{array}\right)$ or $\left(\begin{array}{cc}0 & 1\\-1 & 0\end{array}\right)$, i.e., one of the two substitutions, which , for our discontinuity domain, furnish the generators or the modular group.
With this are given the principal points of view, according to which the geometrical treatment of the Gaussian forms was carried through in "M" I, we come back once more below to the ?projective form? of this theory, which one obtains, if one replace. The $\zeta$-halfplane by the interior of the ellipse of the hyperbolic plane. The conception hereby yielding itself attaches itself to the classical theory the Gaussian forms in many respects even directly.