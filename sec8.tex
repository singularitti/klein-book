\section{The reproducing groups of the indefinite Hermitian forms}

Now secondly, we have to give all those substitutions, which transform a proposed indefinite Hermitian form $(a,b_1,b_2,c)$ into itself. Hereby, it is a question of the group of all these substitutions of the unimodular Picard group, which carry the representing hemisphere of the form $(a,b_1,b_2,c)$, together with its directional arrow, over into itself. This subgroup, which in the sequel will be heavily referred to, we designate, in abbreviation, as the “reproducing group” at the Hermitian form $(a,b_1,b_2,c)$. We recognize in it a properly discontinuous principal-circle group which refers itself to the representing hemisphere of the form $(a,b_1,b_2,c)$, and which has the base circle of this hemisphere as principal circle. With this, we have obtained a first one among the approaches to the “arithmetic” definition of principal-circle groups considered in the introduction (p.447).

The more precise theory of rthe reproducing groups of the indefinite forms $(a,b_1,b_2,c)$ is established through the developments of the previous paragraph.

On the representing hemisphere, a net of polygons was cut out by the pentahedral division of the halfspace, \textcolor{red}{here} furnished is the reduced forms of the class. But the number of those latter forms is finite, let us sayt equal to $\gamma$. It follows immediately: The discontinuity domain $P_0$ of the reproducing group lets itself be represented as a complex of $\nu$ circular-arc polygons of the kind just mentioned. We still call the latter the “partial polygon” of the domain $P_0$. The boundary curves of the paritail polygon, and, with this, of the domain $P_0$, are circular-arcs, which are erected perpendicularly to the principal circle. Naturally, $P_0$, as all discontinuity domains of principal-circle groups, represents a simply connected domain\footnote{*}.

From the finiteness of the number $\gamma$ of the partial polygons follows, that $P_0$ has only finitely many sides. Since, further, the partial polygons press themselves together in infinite number toward each point of the principal circle, then the same thing will hold for the discontinuity domains $P_0$, $P_1$, $P_2,\;\ldots$ With this follows the theorem: The reproducing group of an indefinite Hermitian form is a principal-circle group of “finite” character $(p,n)$, which is “properly” discontinuous on the principal circle itself.

Equivalent forms $(a,b_1,b_2,c)$ and $(a',b_1',b_2',c')$ furnish reproducing groups of the same “class”, which, as transformable into one another, do not count as distinct from the standpoint of invariant theory(p.355 ff.). We shall accordingly \textcolor{red}{here} investigate reduced forms for their reproducing groups. The groups of the remaining forms are to be obtained from here simply by transformation.

The latter remark is important for settling up a system of generating substitutions of the individual reproducing group. Here too, the building up of the polygon $P_0$ from its $\gamma$ partial polygons is fundamental. The process of continuous reduction furnishes the substitutions, which furnish the progression from the individual partial polygon to the neighboring ones. By combining these substitutions according to the prescription of the arrangement of the partial polygons, we arrive at the generators of the reproducing group.---

The equivalence theory of the Hermitian forms of positive determinant is hereby brought to an end in its principal points. Nevertheless, one more series of further discussions of interest might be added here.

First, we remark, that we are able to position the polygon net to the reproducing group, instead of on the representing hemisphere of the form, \textcolor{red}{here} in the $\zeta$-plane, and indeed, outside or inside the base circle of that hemisphere. It is a question hereby of a projection of the hemisphere onto the $\zeta$-plane by semicircles, which \textcolor{red}{here} perpendicularly to the hemisphere as well as to the $\zeta$-plane. The individual point of the hemisphere consequently furnishes two points of the $\zeta$-plane, which are symmetric to one another with respect to the base circle of the hemisphere, i.e., principal circle of the group.

It is not entirely immaterial, whether we utilize the polygon net in the original or the new form. A distinction emerges, e.g., for the “self-inverse” form classes. Should an indefinite form $(a,b_1,b_2,c)$ be equivalent with its inverse form $(-a,-b_1,-b_2,-c)$, then there is, in the unimodular Picard group (of the first kind), one, and with this, at the same time, infinitely many, substitutions, which carry the principal circle over into itself with reversal of the direction of the arrow, The individed one of these substitutions will, in the $\zeta$-plane, interchange the interior of the principal circle with the exterior; however, it transforms the representing hemisphere into itself, and indeed, with reversal of the angles.

We have the simplest example belonging here, in case the representing hemisphere runs through an elliptic axis belonging to the period two, Here than, the fact of the axis are the fixpoints of the elliptic substitution in question. For the hemisphere, this substitution obviously obtains the character of operation of the same kind, since it represents a reflection in the axis just named. 

From this study follows the theorem : For an indefinite Hermitian form equivalent with its inverse form, all substitutions contained in the unimodular Picard group of the first kind, which transform the form either into itself or its inverse, form a group, in which the reproducing group of the form is a distinguished subgroup of index two. The more comprehensive group, following the language on p.132, is to be designated as a group of the “second type”; but on the representing hemisphere it has the character of a group of the “second kind.” That these two kinds of groups are not essentially distinct from one another, was directly remarked on p.141.——

Besides the self-inverse classes, we also have here the ambiguous form classes resp. forms. In this respect, the following definition is to be placed in a position of reeminence: If the representing hemisphere of an indefinite definition is to be placed in a position of preeminence: If the representing hemisphere of an indefinite form $(a,b_1,b_2,c)$ runs orthogonally to one (and, with this, to infinitely many) symmetry hemisphere of the pentahedral division of the halfspace, then the form in  question is called “ambiguous”. The special character of the associated reproducing group is immediately evident: The reproducing group of an ambiguous indefinite form $(a,b_1,b_2,c)$ permits an extension by reflections to a group of the second kind\footnote{*}. This character of the group is then present on the representing hemisphere as well as in the $\zeta$-plane.---
 
A similarity important group extension arises itself on the addition of substitutions of the determinant $i$, which furthermore \textcolor{red}{here} the structure of the $\zeta$-substitutions applied so far. The “extended” Picard group of the first kind thus arising, in which the “unimodular” one is a distinguished subgroup of index two, was likewise investigated in detail above(p.77 ff.) and has as(?) discontinuity domain a “double tetrahedron” determined by the conditions (6) on p.35.

The substitution to the added here, according to the agreement on p.453, furnish Hermitian forms equivalent “in the extended sense”; the determinants of two such forms reveal themselves as equal. One now obviously has here the following alternative: An individual form class of the determinant $D$, upon exercise of one of the new transformations, is either permuted with a second class of the same determinant or is hereby transformed into itself. In the first case, the reproducing groups belonging to those two classes are conjugate inside the extended Picard group, in the second case, the reproducing group itself is capable of extension by addition of substitutions of the determinant $i$. We will then distinguish between an “extended” and a “unimodular” reproducing group (of the first kind); naturally, this one is combined in that one as a distinguished subgroup of index two.——

The question of the accurnence of parabolic substitutions inside the individual one of our reproducing groups has a special significance. One calls two groups “commensurable” if, either directly, or often suitable transformation of one group, they have a subgroup in common, which has finite index inside both groups\footnote{**}.  We shall later be able to show, that the reproducing group of an indefinite Hermitian firm is always and only then commensurable with the ordinary modular group, if it contains parabolic substitutions.

Now parabolic substitutions will occur or not, according as the discontinuity domain of the reproducing group positioned on the representing hemisphere \textcolor{red}{here} down with one or more cusps to the $\zeta$-plane or remains entirely distant from the boundary of the hemisphere. But the first or second of these cases enters, according as the base circle passes through parabolic, i.e., rational points of the $\zeta$-plane or not.

On the strength of this, one sets $\xi$ and then, in consequence of (1) on p.454, has as the equation of the base circle:
\begin{equation}\label{eq:81}
ax^2+ay^2+cz^2-2b_2yz+2b_1xz=0.
\end{equation}
Parabolic substitution will always and only then occur, if this equation lets itself by solved by a triple of rational whole numbers $x$,$y$,$z$. For $a=0$, the existence of such a solution is immediately evident. If $a\gtrless0$, carry the last equation, nu multiplication by a,over into the form:
\begin{equation}
(zx+b_1z)^2+(ay-b_2z)^2-Dz^2=0.
\end{equation}
If $d^2$ is now the greatest \textcolor{red}{here} of a whole rational number going into $D$, then write $D=d^2-D_0$ and moreover set:
\[
ax+b_1z=X,\; ay-b_2z=Y,\; dz=Z.
\]
then with $x$, $y$, $z$, $X$, $Y$, $Z$ are also whole rational numbers, and the latter satisfty the equation:
\begin{equation}\label{eq:82}
X^2+Y^2-D_0Z^2=0.
\end{equation}
Conversely, an integral solution of this equation always furnishes a triple of rational, with this, however, also(?) one such of whole rational, numbers $x$, $y$, $z$ which satisfy equation\ref{eq:81}.

The question of the solubility of equation \ref{eq:82} in whole, not simultaneously vanishing, numbers is now answered by a know theorem of number theory\footnote{*}. The condition, easily recognizable as necessary, namely that -1 is a quadratic residue of $D_0$, is also sufficient for the existence of an integral solution. But -1 will always and only then quadratic a quadratic residue of $D_0$, if no prime number of the form $(4n+3)$ is contained in $D_0$. The theorem therefore hold: Whether, in an individual reproducing group, parabolic substitutions occur or not, depend solely on the numerical value of the associated determinant; the group will always, but also only then, be free of parabolic substitutions, if in $D$ at least one prime number of the form $(4n+3)$ is contained in an odd (highest) power.

Furthermore, we notice, that the indefinite Hermitian forms are \textcolor{red}{exhausted} for the theory of the principal-circle groups inside circle group does not only belonging to each indefinite form $(a,b_1,b_2,c)$; but we shall be able to show at the end of the chapter, that also, conversely, to each such subgroup an indefinite Hermitian form.

In closing, we return once more to the historical development of the theory developed here.

Here, first of all,, besides the references to the historical given on p.92 ff., one more paper of Picard’s on indefinite Hermitian forms is to be named*. This work is to be considered  as a direct continuation of Hermite’s original investigations(in Bd.47 of Credle's Journal), which latter learns undisposed of the indefinite forms $(a,b_1,b_2,c)$. The reduction of the indefinite forms is reduced by Picard 1.c.), by means of a principal originating from Hermite, to that of the definite forms. This principle, which comes into force in even more detail in the next chapter, lets itself be force, by using our geometrical language, in the following manner: A positive Hermitian form is to be designated as “associated: to a given indefinite form, it the representing point of that form is situated on the hemisphere of this latter\footnote{ **}. An indefinite form is then called reduced, if there lets itself be given a definite form associated to it, which is reduced. How, on the basis or this approach, the process of continuous reduction lets itself be introduced, we shall likewise have to explain in the next chapter. This process, then, also plays a very important role for Hermite and Picard 1.c.; however, let it be remembered, that the source of the process in question is to be sought for in "Disquisitions arithmetica"\footnote{a}

It is now immediately evident, that the method of the definition of reduced indefinite forms followed here, and going back to St. Smith, lead materially to the same results, as the Hermitian approach (equal reduction conditions of the definite forms assumed). All the same, the Smith principle, of basing the definitions of the reduced indefinite forms, independently of the definite ones, on a developed theory of groups and discontinuity domains, must be considered as a vey important step vis-à-vis Hemite, which furnished a far more transparent structure of the equivalence theory of indefinite forms. Furthermore, the service is due to Bionchi, to have extended the Smith method to the Dirichlet and Hemitian form\footnote{*}.
